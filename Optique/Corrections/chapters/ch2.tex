\documentclass[../main/main.tex]{subfiles}

\begin{document}

\setcounter{chapter}{1}
\chapter{Miroirs et dioptres plans}
\vspace*{-47pt}
\begin{center}
    \Huge Exercices d'application
\end{center}

\section{Miroir plan et tracé des rayons}
\subsection{}
\begin{minipage}{0.75\linewidth}
    On a intersection des rayons incidents avant le miroir, c'est un objet réel.
    L'image de cet objet par le miroir est son symétrique par rapport au plan du
    miroir : elle est virtuelle. Les rayons émergents passent par cette image
    (mais sont en pointillés derrière le miroir).
\end{minipage}
\begin{minipage}[c]{0.25\linewidth}
    \begin{flushright}
        \begin{tikzpicture}[use optics]
            % Miroir
            \node[mirror, rotate=-90] (M) at (0,-1) {};
            \node[right] at (M.north) {M};
            % Objet
            \coordinate (A) at (-0.25,0);
            \node[above, color=Purple!70] at (A) {A};
            \node[color=Purple!70] at (A) {$\times$};
            % Projection
            \coordinate (H) at ($(M.north)!(A)!(M.south)$);
            \node[above right, gray!70] at (H) {H};
            \node[gray!70] at (H) {$\times$};
            \draw[dashed, gray!70] (A) -- (H) --++ ($(H)-(A)$) coordinate (Ap);
            % Image
            \node[below, Red!70] at (Ap) {A$'$};
            \node[Red!70] at (Ap) {$\times$};
            % Rayons
            \draw[brandeisblue, simple]
                (A) -- ([shift={(-0.15,0)}]M.north) coordinate (Id);
            \draw[brandeisblue, double]
                (A) -- ([shift={(0.15,0)}]M.south) coordinate (Ig);
            \draw[dashed, orange!70, simple] (Ap) -- (Id);
            \draw[dashed, orange!70, double] (Ap) -- (Ig);
            \draw[orange!70, simple] (Id) --++ ($(Id)-(Ap)$);
            \draw[orange!70, double] (Ig) --++ ($(Ig)-(Ap)$);
        \end{tikzpicture}
    \end{flushright}
\end{minipage}

\subsection{}
\begin{minipage}{0.75\linewidth}
    Les rayons incidents se croisent après le miroir. On a un objet virtuel. Son
    image par le miroir plan est son symétrique par rapport au plan du miroir.
    C'est donc une image réelle (au-dessus du miroir). Les rayons émergents
    passent par cette image.
\end{minipage}
\begin{minipage}{0.25\linewidth}
    \begin{flushright}
        \begin{tikzpicture}[use optics]
            % Miroir
            \node[mirror, rotate=-90, name path=Mpath] (M) at (0,-1) {};
            \node[right] at (M.north) {M};
            % Point
            \coordinate (A) at (0,-2);
            \node[below, Purple!70] at (A) {A};
            \node[Purple!70] at (A) {$\times$};
            % Symétrique
            \coordinate (H) at ($(M.north)!(A)!(M.south)$);
            \node[above right, gray!70] at (H) {H};
            \node[gray!70] at (H) {$\times$};
            \draw[dashed, gray!70] (A) -- (H) --++ ($(H)-(A)$) coordinate (Ap);
            \node[above, Red!70] at (Ap) {A$'$};
            \node[Red!70] at (Ap) {$\times$};
            % Rayons
            \coordinate (O1) at (-1.5,0);
            \coordinate (O2) at (-0.75,0);
            % Define intersection
            \path[name path=1A] (O1) -- (A);
            \path[name path=2A] (O2) -- (A);
            \path[name intersections={of=1A and Mpath, by=Ig}];
            \path[name intersections={of=2A and Mpath, by=Id}];
            % Draw
            \draw[brandeisblue, simple] (O1) -- (Ig);
            \draw[brandeisblue, simple, dashed] (Ig) -- (A);
            \draw[brandeisblue, double] (O2) -- (Id);
            \draw[brandeisblue, double, dashed] (Id) -- (A);
            % Émergents
            \draw[orange!70, simple] (Ig) -- (Ap);
            \draw[orange!70, double] (Id) -- (Ap);
        \end{tikzpicture}
    \end{flushright}
\end{minipage}

\subsection{}
\begin{minipage}{0.75\linewidth}
    Ici on a des rayons émergents. Leur intersection est, par définition, le
    point image. Elle se fait derrière le miroir : c'est un image virtuelle. On
    obtient l'objet associé en en faisant le symétrique par rapport au plan du
    miroir : il est donc au-dessus, et c'est un objet réel.
\end{minipage}
\begin{minipage}{0.25\linewidth}
    \begin{flushright}
        \begin{tikzpicture}[use optics]
            % Miroir
            \node[mirror, rotate=-90, name path=Mpath] (M) at (0,0) {};
            \node[right] at (M.north) {M};
            % Point
            \coordinate (Ap) at (-0.5,-0.75);
            \node[below, Red!70] at (Ap) {A$'$};
            \node[Red!70] at (Ap) {$\times$};
            % Symétrique
            \coordinate (H) at ($(M.north)!(Ap)!(M.south)$);
            \node[above left, gray!70] at (H) {H};
            \node[gray!70] at (H) {$\times$};
            \draw[dashed, gray!70] (Ap) -- (H) --++ ($(H)-(Ap)$) coordinate (A);
            \node[above, Purple!70] at (A) {A};
            \node[Purple!70] at (A) {$\times$};
            % Rayons
            \coordinate (E1) at (0,1);
            \coordinate (E2) at (0.7,0.5);
            \path[name path=Eg] (Ap) -- (E1);
            \path[name path=Ed] (Ap) -- (E2);
            % Define intersection
            \path[name intersections={of=Eg and Mpath, by=Ig}];
            \path[name intersections={of=Ed and Mpath, by=Id}];
            % Draw
            \draw[orange!70, dashed, simple] (Ap) -- (Ig);
            \draw[orange!70, dashed, double] (Ap) -- (Id);
            \draw[orange!70, simple] (Ig) --++ ($2*(Ig)-2*(Ap)$);
            \draw[orange!70, double] (Id) --++ ($(Id)-(Ap)$);
            \draw[brandeisblue, simple] (A) -- (Ig);
            \draw[brandeisblue, double] (A) -- (Id);
        \end{tikzpicture}
    \end{flushright}
\end{minipage}

\section{Une grenouille intelligente}
\begin{NCdefi}{Données}
    \begin{minipage}{0.45\linewidth}
        Pour une hauteur de grenouille fixée, il y a une taille de
        nénuphar permettant à tous les rayons partant de la grenouille de ne
        pas traverser le dioptre.
    \end{minipage}
    \begin{minipage}{0.55\linewidth}
        \begin{flushright}
            \begin{tikzpicture}[use optics, scale=0.2]
                % Dioptre
                \def\nair{1}
                \def\neau{1.33}
                \node[screen,
                    object height=8cm, rotate=90,
                    name path=Dpath]
                    (dioptre) at (0,0) {};
                \node[right] at (dioptre.south) {dioptre};
                % Milieu n1
                \node[above left] at (dioptre.south) {$n_\mathrm{air}$};
                % Milieu n2
                \def\dh{11}
                \coordinate (D1) at ($(dioptre.north)!\dh cm!-90:(dioptre.south)$);
                \fill[brandeisblue!30]
                    (dioptre.south) -- (dioptre.north) --
                    (D1) -- (D1-|dioptre.south) -- cycle;
                \node[below left] at (dioptre.south) {$n_\mathrm{eau}$};
                % Nénuphar
                \def\rsiz{11.4}
                \def\rh{0.5}
                \node[draw, scale=0.2, anchor=west,
                    minimum width=2*\rsiz cm,
                    minimum height=2*\rh cm,
                    fill=ForestGreen!80] (nenuph) at
                    ([shift={(3em,0)}]dioptre.north) {};
                % Grenouille
                \def\gsiz{10}
                \draw[Circle-{[Purple!70]Circle}]
                    (nenuph.south) coordinate (Atop)
                    --++ (0,-\gsiz) coordinate (A);
                % Grandeurs
                \node[left] (Ag) at (A) {};
                \node[left] (Atopg) at (Atop) {};
                \draw[<->, ForestGreen] (Ag) --
                node[above, midway, sloped]
                {\scriptsize $h=\SI{10}{cm}$} (Atopg);
                % Grandeurs
                \node[above, inner sep=2]
                    (nenuphtopg) at (nenuph.north) {};
                \node[above, inner sep=2]
                    (nenuphrightg) at (nenuph.north east) {};
                \draw[<->, Red!70] (nenuphtopg) --
                node[above, midway, sloped]
                {\scriptsize \textcolor{Red!70}{$R_0$}} (nenuphrightg);
            \end{tikzpicture}   
        \end{flushright}
    \end{minipage}
\end{NCdefi}

\begin{tcbraster}[raster columns=2, raster equal height=rows]
    \begin{NCprop}{But à atteindre}
        Origine physique de ce phénomène et traduction mathématique.
    \end{NCprop}    
    \begin{NCdemo}{Outils du cours}
        Loi de Snell-Descartes :
        \[ n_1\sin i_1 = n_2\sin i_2\]
        et \underline{angle limite} de réfraction, tel que :
        \[ n_1\sin i_\ell = n_2\sin 90\degres = n_2\]
        qui indique que pour $n_1 > n_2$, il y a un angle d'incidence à
        partir duquel il n'y a pas de rayon réfracté (les rayons réfractés font un
        angle de 90° avec la normale et sont donc parallèles au dioptre).
    \end{NCdemo}
\end{tcbraster}

\begin{NCexem}[breakable]{Application}
    Pour que les pieds de la grenouille ne soient pas visibles par un prédateur
    situé en-dehors de l'eau, c'est-à-dire au-dessus du dioptre, il faut
    simplement qu'il n'y ait pas de rayon partant de ses pieds et qui puissent
    sortir de l'eau : il faut que tous les rayons avec un angle d'incidence plus
    faible que cet angle limite soient bloqués par le nénuphar. C'est possible
    puisqu'on est dans une situation où le rayon passe dans un milieu
    \textbf{moins réfringent}, i.e. $n_2 < n_1$. En effet, dans cette situation
    il y a une inclinaison du rayon incident qui implique que le rayon émergent
    est parallèle à la surface, et tous les rayons au-delà de cet angle limite
    sont tous réfléchis. Un beau, grand schéma avec toutes les données reportées
    dessus mène naturellement à l'utilisation de formules trigonométriques de
    4$^\text{ème}$.
    \begin{center}
        \vspace*{-2.5cm}
        \begin{tikzpicture}[use optics, scale=0.3]
            % Dioptre
            \def\nair{1}
            \def\neau{1.33}
            \node[screen,
                object height=12cm, rotate=90,
                name path=Dpath]
                (dioptre) at (0,0) {};
            \node[right] at (dioptre.south) {dioptre};
            % Milieu n1
            \node[above left] at (dioptre.south) {$n_\mathrm{air}$};
            % Milieu n2
            \def\dh{11}
            \coordinate (D1) at ($(dioptre.north)!\dh cm!-90:(dioptre.south)$);
            \fill[brandeisblue!30]
                (dioptre.south) -- (dioptre.north) --
                (D1) -- (D1-|dioptre.south) -- cycle;
            \node[below left] at (dioptre.south) {$n_\mathrm{eau}$};
            % Nénuphar
            \def\rsiz{11.4}
            \def\rh{0.5}
            \node[draw, scale=0.3, anchor=west,
                minimum width=2*\rsiz cm,
                minimum height=2*\rh cm,
                fill=ForestGreen!80] (nenuph) at
                ([shift={(3em,0)}]dioptre.north) {};
            % Grenouille
            \def\gsiz{10}
            \draw[Circle-{[Purple!70]Circle}]
                (nenuph.south) coordinate (Atop)
                --++ (0,-\gsiz) coordinate (A);
            % Grandeurs
            \node[left] (Ag) at (A) {};
            \node[left] (Atopg) at (Atop) {};
            \draw[<->, ForestGreen] (Ag) --
            node[above, midway, sloped]
            {\scriptsize $h=\SI{10}{cm}$} (Atopg);
            % Grandeurs
            \node[above, inner sep=2]
                (nenuphtopg) at (nenuph.north) {};
            \node[above, inner sep=2]
                (nenuphrightg) at (nenuph.north east) {};
            \draw[<->, Red!70] (nenuphtopg) --
            node[above, midway, sloped]
            {\scriptsize \textcolor{Red!70}{$R_0$}} (nenuphrightg);
            % Point
            \node[left, Purple!70] at (A) {A};
            \node[Purple!70] at (A) {$\times$};
            % Rayons
            \foreach \i/\n in {30/1, 50/2, 60/3, 70/4}{
                \path[name path=AI\n] (A) --++ (\i:25cm);
                % Define intersection
                \path[name intersections={of=Dpath and AI\n, by=I\n}];
                \draw[brandeisblue, dotted, simple] (A) -- (I\n);
            }
            \path[name path=AI0] (A) --++ (41.25:30cm);
            % Define intersection
            \path[name intersections={of=AI0 and Dpath, by=I0}];
            \draw[brandeisblue, simple] (A) -- (I0);
            % Snell
            \foreach \i/\n in {50/2, 60/3, 70/4}{
                \draw[orange!70, dotted, simple] (I\n) --++
                    ({90-asin(\neau*sin(90-\i)/\nair)}:7cm);
            }
            \draw[orange!70, simple] (I0) --++ (0:15cm) coordinate (Aend);
            % Réflexion
            \draw[orange, dotted, simple, rotate=-90] (I1) --++ (60:7cm);
            % Normal
            \def\hsiz{5}
            \draw[gray!50, dashed, name path=Hpath]
            ([shift={(0,\hsiz)}]I0) coordinate (Htop) --
            ([shift={(0,-\hsiz)}]I0) coordinate (Hbot);
            % Define intersection
            \path[name intersections={of=Hpath and Dpath, by=H}];
            \node[gray!70, below right] at (H) {H};
            \node[gray!70] at (H) {$\times$};
            % Draw angles
            \pic[scale=0.3,
            angle radius=2cm, angle eccentricity=1.7,
            draw, ->, "$i_1$" alias=i1r]
            {angle=A--H--Hbot};
            \pic[scale=0.3,
            angle radius=2cm, angle eccentricity=1.7,
            draw, ->, "$i_1$" alias=i1]
            {angle=H--A--Atop};
            % Draw angles
            \pic[scale=0.3,
            angle radius=2cm, angle eccentricity=1.7,
            draw, ->,
            "\vspace{-10pt}\hspace{20pt}$i_2 = \SI{90}{\degres}$" alias=i2]
            {angle=Aend--H--Htop};
        \end{tikzpicture}
    \end{center}
    On voit ici qu'une simple fonction $\tan$ permet d'exprimer $R_0$ :
    \begin{empheq}[box=\fbox]{align}\label{eq:tan}
        \tan i_1 = \frac{R_0}{h}
    \end{empheq}
    Seulement on n'a pas encore la valeur de $i_1$. Or, on a déterminé que pour
    fonctionner l'astuce de la grenouille est d'avoir $i_1 = i_\ell$, et d'après
    le cours :
    \begin{empheq}{align}
        n\eau\sin i_\ell       & = n\air\\
        \Leftrightarrow i_\ell & = \asin \frac{n\air}{n\eau}\label{eq:ilim}
    \end{empheq}
    On peut donc écrire, avec \ref{eq:tan} et \ref{eq:ilim} :
    \begin{empheq}[box=\fbox]{equation}
        R_0 = h\times\tan\left(\asin \frac{n\air}{n\eau}\right)
        \quad \text{avec}
        \left\{
            \begin{array}{rcl}
                h & = & \SI{10.0}{cm}\\
                n_\mathrm{air} & = & 1\\
                n_\mathrm{eau} & = & 1.33
            \end{array}
        \right.
    \end{empheq}
    et finalement,
    \begin{empheq}[box=\fbox]{equation}
        R_0 = \SI{11.4}{cm}
    \end{empheq}
\end{NCexem}

\begin{tcbraster}[raster columns=3, raster equal height=rows]
    \begin{NCcoro}[raster multicolumn=2]{Conseil}
        Pour retenir vos formules trigonométriques, un moyen mnémotechnique :
        \begin{center}
            CAH\quad SOH \quad TOA
        \end{center}
        pour \[
            \cos\a = \frac{ \text{adjacent}}{ \mathrm{hypothénuse}} \quad
            \sin\a = \frac{ \text{opposé}}{ \mathrm{hypothénuse}} \quad
            \tan\a = \frac{ \text{opposé}}{ \mathrm{adjacent}} \]
    \end{NCcoro}
    \begin{NCimpo}{Important !}
        La deuxième plus \underline{grosse} erreur facile à faire mais cette
        fois pire que \textbf{tout}, c'est d'oublier que :
        \begin{center}
            \large \bfseries
            Les angles de la relation de Descartes sont définis entre les
            rayons et la \underline{normale} au dioptre !
        \end{center}
        Il ne faut pas prendre la surface du dioptre comme référence pour
        définir des angles.
    \end{NCimpo}
\end{tcbraster}

\section{Le chat et le poisson}
\begin{tcbraster}[raster columns=3, raster equal height=rows]
    \begin{NCdefi}[raster multicolumn=2]{Données}
        \begin{minipage}{0.5\linewidth}
            \begin{enumerate}
                \item Aquarium $\equiv$ dioptre plan ;
                \item $n _\mathrm{eau} = 1.33$, $n _\mathrm{air} = 1$ ;
                \item $\obar{HA} = \SI{-15}{cm}$.
            \end{enumerate}
        \end{minipage}
        \begin{minipage}{0.5\linewidth}
            \begin{center}
                \begin{tikzpicture}[use optics, scale=0.1]
                    % Dioptre
                    \def\na{1.0}
                    \def\nb{1.33}
                    \node[screen, scale=.1,
                    object height=20cm,
                    name path=Dpath] (dioptre) at (0, 10cm) {};
                    \node[above] at (dioptre.north) {dioptre};
                    % Milieu n1
                    \node[below left] at (dioptre.north) {$n\air$};
                    % Milieu n2
                    \def\dh{20}
                    \coordinate (D1) at ($(dioptre.north)!\dh cm!90:(dioptre.south)$);
                    \fill[brandeisblue!30]
                    (dioptre.south) -- (dioptre.north) --
                    (D1) -- (dioptre.south-|D1) -- cycle;
                    \node[below right] at (dioptre.north) {$n_\mathrm{eau}$};
                    % Axe optique
                    \draw[thin, ->, name path=AO](-20cm,0)--(20cm,0)node[below]{A.O.};
                    % Poisson
                    \node[anchor=south] (fish) at (10,0) {\twemoji{fish}};
                    % Chat
                    \node[anchor=south,
                        xscale=-1, scale=2] (cat) at (-15,-1.2) {\twemoji{cat}};
                    % Point
                    \coordinate (A) at ([shift={(0,1.2)}]cat.south);
                    \node[below, Purple!70] at (A) {A};
                    \node[Purple!70] at (A) {$\times$};
                    % Define intersection
                    \path[name intersections={of=Dpath and AO, by=H}];
                    % Point
                    \node[below, gray!70] at (H) {H};
                    \node[gray!70] at (H) {$\times$};
                    % Grandeurs
                    \node[below=.2] (Ag) at (A) {};
                    \node[below=.2] (Hg) at (H) {};
                    \draw[<->, ForestGreen] (Ag) --
                    node[below, midway, sloped]{\scriptsize
                        $\obar{AH}=\SI{15}{cm}$} (Hg);
                \end{tikzpicture}
            \end{center}            
        \end{minipage}
    \end{NCdefi}    
    \begin{NCprop}{Résultats attendus}
        \begin{enumerate}
            \item Sachant que le poisson observe la lumière partant du chat (point
                A), que vaut $\obar{HA'}$ ?
            \item Sachant que le \underline{chat} observe la lumière partant du poisson
                (point A), que vaut $\obar{HA'}$ ?
        \end{enumerate}
    \end{NCprop}
\end{tcbraster}

\begin{tcbraster}[raster columns=4, raster equal height=rows]
    \begin{NCdemo}{Outils du cours}
        Relation de conjugaison pour un objet $A$ dans un milieu d'indice $n_1$,
        dont l'image $A'$ est dans un milieu d'indice $n_2$ :
        \[ \frac{\obar{HA'}}{\obar{HA}} = \frac{n_2}{n_1}\]
        \begin{center}
            ou
        \end{center}
        \[\frac{\obar{HA'}}{n_2} - \frac{\obar{HA}}{n_1} = 0 \]
        \begin{center}
            ou
        \end{center}
        \[ 0 = \frac{n_2}{\obar{HA'}} - \frac{n_1}{\obar{HA}} \]
        qui ressemble « bizarrement » à la relation de conjugaison pour une
        lentille mince...
    \end{NCdemo}    
    \begin{NCexem}[raster multicolumn=3]{Application}
        \begin{enumerate}
            \item Dans ce cas, $A$, le chat, est dans un milieu d'indice $n_1 =
                1$ ; son image par le dioptre plan donne $A'$ dans le milieu
                d'indice $n_2 = n _\mathrm{eau} = 1.33$. On prend le sens de la
                lumière de l'objet à l'observataire, donc ici de gauche à
                droite. On adapte la relation de conjugaison :
                \[ \obar{HA'} = n _\mathrm{eau}\obar{HA} = \SI{-22.5}{cm}\]
                Ainsi, du point de vue du poisson, le chat est plus loin qu'il
                ne l'est vraiment.
            \item Dans ce cas, $A$, le poisson, est dans un milieu d'indice $n_1
                = n _\mathrm{eau} = 1.33$, et son image par le dioptre donne
                $A'$ dans l'air d'indice $n_2 = n _\mathrm{air} = 1$. Ici le
                sens de progagation est de droite à gauche. On réutilise la
                relation de conjugaison :
                \[ \obar{HA'} = \frac{\obar{HA}}{n _\mathrm{eau}} > \obar{HA} \]
                En effet, puisqu'on a posé le sens de propagation de droite à
                gauche, $\obar{HA} < 0$, et donc $\obar{HA'} > \obar{HA}$ comme
                $n\eau > 1$. Cela signifie que du point de vue du chat, le
                poisson est plus près du dioptre qu'il ne l'est vraiment.
        \end{enumerate}
    \end{NCexem}
\end{tcbraster}

\begin{NCexem}{Schémas}
    \begin{minipage}{0.5\linewidth}
        \begin{center}
            \begin{tikzpicture}[use optics, scale=0.15]
                % Dioptre
                \def\na{1.0}
                \def\nb{1.33}
                \node[screen, scale=.15,
                object height=20cm,
                name path=Dpath] (dioptre) at (0, 10cm) {};
                \node[above] at (dioptre.north) {dioptre};
                % Milieu n1
                \node[below left] at (dioptre.north) {$n\air$};
                % Milieu n2
                \def\dh{20}
                \coordinate (D1) at ($(dioptre.north)!\dh cm!90:(dioptre.south)$);
                \fill[brandeisblue!30]
                (dioptre.south) -- (dioptre.north) --
                (D1) -- (dioptre.south-|D1) -- cycle;
                \node[below right] at (dioptre.north) {$n_\mathrm{eau}$};
                % Axe optique
                \draw[thin, ->, name path=AO](-25cm,0)--(25cm,0)node[below]{A.O.};
                % Define intersection
                \path[name intersections={of=Dpath and AO, by=H}];
                % Point
                \node[below, gray!70] at (H) {H};
                \node[gray!70] at (H) {$\times$};
                % Observation
                \oeil[shift={(15cm,0)}, scale=5, rotate=180];
                % Chat
                \node[anchor=south,
                    xscale=-1, scale=2] (cat) at (-15,-1.2) {\twemoji{cat}};
                % Objet
                \coordinate (A) at ([shift={(0,1.2)}]cat.south);
                \coordinate (B) at ([shift={(0,10)}]A);
                \draw[->, Purple!70] (A)
                    node [below] {A}
                    -- (B)
                    node [above] {B};
                % Grandeurs
                \node[below, inner sep=2] (Ag) at (A) {};
                \node[below, inner sep=2] (Hg) at (H) {};
                \draw[|<->|, ForestGreen] (Ag) --
                node[below, midway, sloped]{\scriptsize
                    $\obar{AH}=\SI{15}{cm}$} (Hg);
                % Image
                \def \pos{-22.5cm}
                \len{(A)}{(B)}{\siz}
                \coordinate (Ap) at ($(0,0)+(\pos,0)$);
                \coordinate (Bp) at ($(Ap)+(0,\siz)$);
                \draw[->, Red!70] (Ap) 
                node[below] {A$'$}
                -- (Bp)
                node[above left] {B$'$};
                % Chat bis
                \node[anchor=south,
                    xscale=-1, scale=2, opacity=.3]
                    (cat) at (-22.5,-1.2) {\twemoji{cat}};
                % Grandeurs
                \node[below=0.6, inner sep=0] (Apg) at (Ap) {};
                \node[below=0.6, inner sep=0] (Hg) at (H) {};
                \draw[|<->|, ForestGreen] (Apg) --
                node[below, midway, sloped]{\scriptsize
                $\obar{A'H}=\SI{22.5}{cm}$} (Hg);
            \end{tikzpicture}
        \end{center} 
    \end{minipage}
    \vrule
    \begin{minipage}{0.5\linewidth}
        \begin{center}
            \begin{tikzpicture}[use optics, scale=0.15]
                % Dioptre
                \def\na{1.0}
                \def\nb{1.33}
                \node[screen, scale=.15,
                object height=20cm,
                name path=Dpath] (dioptre) at (0, 10cm) {};
                \node[above] at (dioptre.north) {dioptre};
                % Milieu n1
                \node[below left] at (dioptre.north) {$n\air$};
                % Milieu n2
                \def\dh{20}
                \coordinate (D1) at ($(dioptre.north)!\dh cm!90:(dioptre.south)$);
                \fill[brandeisblue!30]
                (dioptre.south) -- (dioptre.north) --
                (D1) -- (dioptre.south-|D1) -- cycle;
                \node[below right] at (dioptre.north) {$n_\mathrm{eau}$};
                % Axe optique
                \draw[thin, ->, name path=AO](-25cm,0)--(25cm,0)node[below]{A.O.};
                % Define intersection
                \path[name intersections={of=Dpath and AO, by=H}];
                % Point
                \node[below, gray!70] at (H) {H};
                \node[gray!70] at (H) {$\times$};
                % Observation
                \oeil[shift={(-15cm,0)}, scale=5];
                % Poisson
                \node[anchor=south] (fish) at (15,0) {\twemoji{fish}};
                % Objet
                \coordinate (A) at (fish.south);
                \coordinate (B) at ([shift={(0,10)}]A);
                \draw[->, Purple!70] (A)
                    node [below] {A}
                    -- (B)
                    node [above] {B};
                % Grandeurs
                \node[below=.7, inner sep=2] (Ag) at (A) {};
                \node[below=.7, inner sep=2] (Hg) at (H) {};
                \draw[|<->|, ForestGreen] (Ag) --
                node[below, midway, sloped]{\scriptsize $\obar{HA}<0$} (Hg);
                % Image
                \def \pos{10cm}
                \len{(A)}{(B)}{\siz}
                \coordinate (Ap) at ($(0,0)+(\pos,0)$);
                \coordinate (Bp) at ($(Ap)+(0,\siz)$);
                \draw[->, Red!70] (Ap) 
                node[below] {A$'$}
                -- (Bp)
                node[above left] {B$'$};
                % Poison bis
                \node[anchor=south, opacity=.3]
                    (fish) at (10,0) {\twemoji{fish}};
                % Grandeurs
                \node[below=0.1, inner sep=2] (Apg) at (Ap) {};
                \node[below=0.1, inner sep=2] (Hg) at (H) {};
                \draw[|<->|, ForestGreen] (Apg) --
                node[below, midway, sloped]{\scriptsize
                $\obar{HA'}$} (Hg);
            \end{tikzpicture}
        \end{center} 
    \end{minipage}
\end{NCexem}

\begin{tcbraster}[raster columns=5, raster equal height=rows]
    \begin{NCcoro}[raster multicolumn=2]{Conseils}
        Réécrire les formules telles que vous les avez apprises en cours,
        \textbf{puis} inclure les données de l'énoncé une par une pour éviter les
        erreurs...
    \end{NCcoro}    
    \begin{NCimpo}[raster multicolumn=3]{Important}
        \begin{enumerate}
            \item Ne pas se tromper sur le caractère {\huge algébrique} des
                grandeurs ;
            \item Bien identifier quelle est la source, quæl est l'observataire.
        \end{enumerate}
    \end{NCimpo}
\end{tcbraster}

\newpage

\begin{center}
    \Huge Exercices d'entraînement
\end{center}

\setcounter{section}{5}
\section{Lois de Snell-Descartes}
Cet exercice est d'une simplicité absolue. Et pourtant...
\begin{NCdefi}{Données}
    \begin{enumerate}
        \item Rayon incident sur un dioptre entre air et milieu d'indice $n$ :
            angle {\huge avec le dioptre} de \SI{56}{\degres} ;
        \item Différence d'angle entre rayon incident et réfléchi (« déviation
            ») de \SI{13.5}{\degres}.
    \end{enumerate}
\end{NCdefi}

\begin{NCprop}{Résultat attendu}
    Indice du liquide.
\end{NCprop}

\begin{NCdemo}{Outils du cours}
    Loi de Snell-Descartes :
    \[ n_1\sin i_1 = n_2 \sin i_2 \]
    avec $n_1$ l'indice du milieu d'indidence, $n_2$ celui du milieu de
    réfraction, $i_1$ l'angle d'incidence et $i_2$ l'angle de réfraction.
\end{NCdemo}

\begin{NCexem}{Application}
    Un bon schéma fait attentivement est \textbf{nécessaire} ici. En effet,
    les angles donnés ne sont pas ceux qu'on utilise dans la relation de
    Snell-Descartes. \bigbreak
    
    En appelant $i$ l'angle entre le rayon et le dioptre, on a
    \[ i + i_1 = 90\degres\]
    et en appelant $D$ la déviation entre les deux rayons, on a
    \[ i + 90\degres + i_2 + D = 180\degres\]

    On se débrouille pour trouver $n_2$ sachant qu'on a $i_1 = \SI{34}{\degres}$
    et $i_2 = \SI{20.5}{\degres}$. On trouve
    \[ \boxed{n_2 = 1.6} \]
\end{NCexem}

\end{document}
