\documentclass[../main/main.tex]{subfiles}
\begin{document}

\chapter{Instruments d'optique}
\vspace*{-47pt}
\begin{center}
    \Huge Exercices d'application
\end{center}
\section{Tracés de rayons avec association de lentilles}
Les associations de lentilles ne présentent pas de difficultés particulières,
une fois les techniques de construction maîtrisées (cf. chapitre \ref{ch:O1}).

\subsection{}
Dans ce premier cas, on doit construire l'image d'un objet \underline{réel} par
l'association de deux lentilles convergentes. Il suffit pour cela de construire
l'image de l'objet initial $\ABb$ par la lentille $L_1$, image que l'on appellera
$\ABa$. C'est cette image qui servira d'objet à la lentille $L_2$, qui en
formera l'image finale $\ABp$. \bigbreak

On procède donc de la même manière que précédemment, en traçant :
\begin{enumerate}
    \item Le rayon passant par $B$ et par $O_1$ : il ne sera pas dévié ;
    \item Le rayon passant par $B$ et par $F_1$ : il émerge parallèle à
        l'\ed{A.O.}{axe optique} (optionel quand on est sûr-e de ne pas se
        tromper avec les deux autres rayons) ;
    \item Le rayon passant par $B$ et parallèle à l'A.O. : il passe par $F'_1$
        en sortie.
\end{enumerate}

On obtient un faisceau \underline{convergent} en sortie de cette lentille,
l'intersection des rayons se faisant donc dans leur prolongement dans le sens
positif. $\ABa$ est une \underline{image réelle} \underline{\underline{pour
$L_1$}}. \bigbreak

En revanche, cette image, qui est donc l'objet de la lentille $L_2$, est dans
l'espace image de celle-ci : c'est donc un \underline{objet virtuel}
\underline{\underline{pour $L_2$}}. On construit donc son image en trançant :

\begin{enumerate}
    \item Le rayon passant par $B_1$ et par $O_2$ : il ne sera pas dévié ;
    \item Le rayon passant par $B_1$ et par $F_2$ : il émerge parallèle à l'A.O.
        (optionnel) ;
    \item Le rayon passant par $B_1$ et parallèle à l'A.O. : il passe par $F'_2$
        en sortie.
\end{enumerate}

On fait partir les rayons de la gauche du système comme s'ils allaient passer
par $B_1$, mais une fois arrivés à la lentille on continue les traits en
pointillés pour montrer que ce sont des rayons virtuels. Les rayons émergents
suivent les règles de la définition \ref{rconst}, et donnent un faisceau
émergent \underline{convergent} donnant lieu à une \underline{image réelle}.

\begin{tikzpicture}[]
    % Axe optique
    \draw[thin, ->](-7,0)--(7,0)node[below]{A.O.};
    % Lentille 1
    \coordinate (O1) at (-2,0);
    \def \fa{3}
    \def \lsiz{4}
    \coordinate (F1) at ($(O1)+(-\fa,0)$);
    \coordinate (F1p) at ($(O1)+(\fa,0)$);

    \foreach \x/\z in {O1/\fa}{
        \draw[shift={(\x)}] (0,0)
            node[below right] {O$_{1}$};
        \draw[shift={(\x)}] (\z,2pt) --++ (0,-4pt)
            node[below right] {F$'_{1}$};
        \draw[shift={(\x)}] ({-\z},2pt) --++ (0,-4pt)
            node[below left] {F$_{1}$};
    }

    \draw[shift={(O1)}, thick,
        <->, >=latex, name path=L1] (0,-\lsiz)--(0,\lsiz)
        node[above]{$\Lc_{1}$};
    % Lentille 2
    \coordinate (O2) at (2,0);
    \def \fb{4.5}
    \def \lsiz{4}
    \coordinate (F2) at ($(O2)+(-\fb,0)$);
    \coordinate (F2p) at ($(O2)+(\fb,0)$);

    \foreach \x/\z in {O2/\fb}{
        \draw[shift={(\x)}] (0,0)
            node[below right] {O$_{2}$};
        \draw[shift={(\x)}] (\z,2pt) --++ (0,-4pt)
            node[above right] {F$'_{2}$};
        \draw[shift={(\x)}] ({-\z},2pt) --++ (0,-4pt)
            node[above left] {F$_{2}$};
    }

    \draw[shift={(O2)}, thick,
        <->, >=latex, name path=L2] (0,-\lsiz)--(0,\lsiz)
        node[above]{$\Lc_{2}$};
    % Objet
    \def \pos{-5}
    \def \siz{2}
    \coordinate (A) at ($(O1)+(\pos,0)$);
    \coordinate (B) at ($(A)+(0,\siz)$);
    \draw[->, Purple!70] (A) 
        node[below] {A}
        -- (B)
        node[above left] {B};
    % Rayons pour objet avant foyer d'une lentille convergente
        % Lengths
        \len{(A)}{(O1)}{\posi}
        \len{(A)}{(B)}{\size}
        \len{(O1)}{(F1)}{\foc}
        % Rayon 1
        \draw[Purple!70, simple] (B) -- (O1);
        \def\mut{2}
        \path[brandeisblue, simple, name path=OBp] (O1)
        --++ ($\mut*(O1)-\mut*(B)$);
        % Rayon 2
        \draw[Purple!70, double] (B) --++ (\posi,0) coordinate (I);
        \def \mut{3}
        \path[brandeisblue, double, name path=IBp] (I)
        --++ ($\mut*(F1p)-\mut*(I)$);
        \def \intsiz{\size*\foc/(\posi-\foc)}
        % Rayon 3
        \draw[Purple!70, triple] (B) -- (F1)
        --++ (\foc,-{\intsiz}) coordinate (E);
        \def \mut{3}
        \path[brandeisblue, triple, name path=EBp] (E) --++ (\mut*\foc,0);
        % Intersections avec lentille
        \path[name intersections={of=L2 and OBp, by=OBpI}];
        \path[name intersections={of=L2 and IBp, by=IBpI}];
        \path[name intersections={of=L2 and EBp, by=EBpI}];
        % Actual draw
        \draw[brandeisblue, simple, name path=OBpA]
            (O1) --
            (OBpI);
        \draw[brandeisblue, double, name path=IBpA]
            (I) --
            (IBpI);
        \draw[brandeisblue, triple, name path=EBpA]
            (E) --
            (EBpI);
        % Dashed draws
        \def \mut{1}
        \draw[brandeisblue, simple, dashed, name path=OBpB] (OBpI)
        --++ ($\mut*(O1)-\mut*(B)$);
        \def \mut{1.5}
        \draw[brandeisblue, double, dashed, name path=IBpB] (IBpI)
        --++ ($\mut*(F1p)-\mut*(I)$);
        \def \mut{3}
        \draw[brandeisblue, triple, dashed, name path=EBpB] (E)
        --++ (\mut*\foc,0);
        % Image
        \path[name intersections={of=OBpB and IBpB, by=B1}];
        \coordinate (A1) at ($(B1)+(0,{\intsiz})$);
        \draw[->, brandeisblue] (A1) 
        node[above] {A1}
        -- (B1)
        node[below right] {B1};
        % Rayons pour objet virtuel d'une lentille convergente
        % Lengths
        \len{(A1)}{(O2)}{\posi}
        \len{(A1)}{(B1)}{\size}
        \len{(O2)}{(F2)}{\foc}
        % Rayon 1
        \draw[ForestGreen, simple] ($2*(O2)-(B1)$) -- (O2);
        \def \mut{1.8}
        \draw[orange!70, simple, name path=OBp] (O2)
            --++ ($\mut*(B1)-\mut*(O2)$);
        % Rayon 2
        \def \mut{2}
        \draw[ForestGreen, double] ($(B1)-(\mut*\posi,0)$)
        --++ (\mut*\posi-\posi,0) coordinate (I);
        \draw[ForestGreen, dashed, double] (I) --++ (2*\posi,0);
        \def \mut{1.5}
        \draw[orange!70, double, name path=IBp] (I)
            --++ ($\mut*(F2p)-\mut*(I)$);
        % Rayon 3
        \def \intsiz{-\size*\foc/(\posi+\foc)}
        \draw[ForestGreen, triple] (F2) --++ (\foc,{\intsiz}) coordinate (E);
        \def \mut{1.2}
        \draw[ForestGreen, dashed, triple] (E) -- (B1) --++ ($\mut*(B1)-\mut*(E)$);
        \draw[orange!70, triple, name path=EBp] (E) --++ (1.5*\posi,0);
        % Image
        \path[name intersections={of=OBp and IBp, by=Bp}];
        \coordinate (Ap) at ($(Bp)+(0,-{\intsiz})$);
        \draw[->, Red!70] (Ap) 
        node[below] {A'}
        -- (Bp)
        node[below right] {B'};
\end{tikzpicture}

\subsection{}
De la même manière, on construit $\ABa$ à partir de l'action de $L_1$ sur
$\ABb$.  C'est la situation 1) 1- c. pour la lentille convergente : on obtient
une \underline{image virtuelle} \underline{\underline{pour $L_1$}}. $\ABa$ est
cependant dans l'espace objet pour $L_2$, et est donc un \underline{objet réel}
\underline{\underline{pour $L_2$}}. On construit son image comme dans la
situation 1) 1- a. pour la lentille divergente, et on obtient une
\underline{image virtuelle}.

\begin{tikzpicture}[]
    % Axe optique
    \draw[thin, ->](-7,0)--(7,0)node[below]{A.O.};
    % Lentille 1
    \coordinate (O1) at (-3,0);
    \def \fa{3}
    \def \lsiz{4}
    \coordinate (F1) at ($(O1)+(-\fa,0)$);
    \coordinate (F1p) at ($(O1)+(\fa,0)$);

    \foreach \x/\z in {O1/\fa}{
        \draw[shift={(\x)}] (0,0)
            node[below right] {O$_{1}$};
        \draw[shift={(\x)}] (\z,2pt) --++ (0,-4pt)
            node[below right] {F$'_{1}$};
        \draw[shift={(\x)}] ({-\z},2pt) --++ (0,-4pt)
            node[below left] {F$_{1}$};
    }

    \draw[shift={(O1)}, ultra thick,
        <->, >=latex, name path=L1] (0,-\lsiz)--(0,\lsiz)
        node[above]{$\Lc_{1}$};
    % Lentille 2
    \coordinate (O2) at (4,0);
    \def \fb{-1.5}
    \def \lsiz{4}
    \coordinate (F2) at ($(O2)+(-\fb,0)$);
    \coordinate (F2p) at ($(O2)+(\fb,0)$);

    \foreach \x/\z in {O2/\fb}{
        \draw[shift={(\x)}] (0,0)
            node[below right] {O$_{2}$};
        \draw[shift={(\x)}] (\z,2pt) --++ (0,-4pt)
            node[below left] {F$'_{2}$};
        \draw[shift={(\x)}] ({-\z},2pt) --++ (0,-4pt)
            node[below right] {F$_{2}$};
    }

    \draw[shift={(O2)}, ultra thick,
        >-<, >=latex, name path=L2] (0,-\lsiz)--(0,\lsiz)
        node[above]{$\Lc_{2}$};
    % Objet
    \def \pos{-1}
    \def \siz{2}
    \coordinate (A) at ($(O1)+(\pos,0)$);
    \coordinate (B) at ($(A)+(0,\siz)$);
    \draw[->, Purple!70] (A) 
        node[below] {A}
        -- (B)
        node[above left] {B};
    % Rayons pour objet entre le foyer et le centre d'une lentille convergente
        % Lengths
        \len{(A)}{(O1)}{\posi}
        \len{(A)}{(B)}{\size}
        \len{(O1)}{(F1)}{\foc}
        % Rayon 1
        \draw[Purple!70, simple] (B) -- (O1);
        \draw[brandeisblue, simple] (O1) --++ ($(O1)-(B)$);
        \def \mut{1}
        \draw[brandeisblue, simplerev,
        dashed, name path=OBp] (B) --++ ($\mut*(B)-\mut*(O1)$);
        % Rayon 2
        \draw[Purple!70, double] (B) --++ (\posi,0) coordinate (I);
        \def \mut{2}
        \draw[brandeisblue, double] (I) --++ ($\mut*(F1p)-\mut*(I)$);
        \def \mut{1.5}
        \draw[brandeisblue, doublerev,
        dashed, name path=IBp] (I) --++ ($-\mut*(F1p)+\mut*(I)$);
        % Rayon 3
        \def \intsiz{\size*\foc/(\posi-\foc)}
        \draw[Purple!70, triple] (B) -- (F1)
        --++ (\foc,-{\intsiz}) coordinate (E);
        \def \mut{2}
        \draw[brandeisblue, triple] (E) --++ (\mut*\foc,0);
        \def \mut{1}
        \draw[brandeisblue, triplerev,
        dashed, name path=EBp] (E) --++ (-\mut*\foc,0);
        % Image
        \path[name intersections={of=OBp and IBp, by=B1}];
        \coordinate (A1) at ($(B1)+(0,{\intsiz})$);
        \draw[->, brandeisblue] (A1) 
        node[below] {A1}
        -- (B1)
        node[below right] {B1};
        % Rayons pour objet avant foyer image d'une lentille divergente
        % Lengths
        \len{(A1)}{(O2)}{\posi}
        \len{(A1)}{(B1)}{\size}
        \len{(O2)}{(F2)}{\foc}
        % Rayon 1
        \draw[ForestGreen, simple, name path=OBp] (B1) -- (O2);
        \def \mut{0.5}
        \draw[orange!70, simple] (O2) --++ ($\mut*(O2)-\mut*(B1)$);
        % Rayon 2
        \draw[ForestGreen, double] (B1) --++ (\posi,0) coordinate (I);
        \def \mut{1}
        \draw[orange!70, double] (I)
        --++ ($-\mut*(F2p)+\mut*(I)$);
        \draw[orange!70, doublerev,
        dashed, name path=IBp] (I)
        --++ ($\mut*(F2p)-\mut*(I)$);
        % Rayon 3
        \path[name path=tripleB] (B1) -- (F2);
        \path[name intersections={of=L2 and tripleB, by=E}];
        \draw[ForestGreen, triple] (B1) -- (E);
        \def \mut{2}
        \draw[ForestGreen, triple, dashed] (E) --++ ($\mut*(F2)-\mut*(E)$);
        \draw[orange!70, triple] (E) --++ (\mut*\foc,0);
        \draw[orange!70, triplerev,
        dashed, name path=EBp] (E) --++ (-\mut*\foc,0);
        % Image
        \path[name intersections={of=OBp and IBp, by=Bp}];
        \len{(O2)}{(E)}{\intsiz}
        \coordinate (Ap) at ($(Bp)+(0,-{\intsiz})$);
        \draw[->, Red!70] (Ap) 
        node[below] {A'}
        -- (Bp)
        node[above right] {B'};
\end{tikzpicture}

\section{Des lunettes astronomiques}
\begin{center}
    \huge Partie 1
\end{center}

\pagebreak

\begin{NCdefi}{Données}
    Association de deux lentilles :
    \begin{enumerate}
        \item $L_1$ « objectif », vergence $C_1 = \SI{3.125}{\de}$, diamètre $D
            = \SI{30}{mm}$ ;
        \item $L_2$ « oculaire », vergence $C_2 = \SI{25}{\de}$.
    \end{enumerate}
\end{NCdefi}

\subsection{}

\begin{NCprop}{Résultat attendu}
    Focales de lentilles
\end{NCprop}

\begin{NCdemo}{Outil du cours}
    Une lentille de focale $f'$ a pour vergence $V$ :
    \[ V = \frac{1}{f'} \]
\end{NCdemo}

\begin{NCexem}{Application}
    \[ \boxed{\obar{O_1F'_1} = \SI{32}{cm}} \]
    \[ \boxed{\obar{O_2F'_2} = \SI{4}{cm}} \]
\end{NCexem}

\subsection{a.}
\begin{defi}{Système afocal}
    Est afocal un système pour lequel un objet initial à l'infini donne une
    image finale à l'infini.
\end{defi}

\begin{inte}{Intérêt d'un système afocal}
    Un système afocal présente comme intérêt de permettre à un œil emmétrope
    d'observer sans fatigue, étant donné que l'image sortant du système est à
    l'infini (cf chapitre 1 exercice 4).
\end{inte}

\setcounter{subsection}{1}
\subsection{b.}
\begin{NCprop}{Résultat attendu}
    $$\obar{O_1O_2}$$
\end{NCprop}

\begin{NCdemo}{Outils du cours}
    Règles de construction de rayons :
    \begin{enumerate}

        \item Un rayon provenant de l'infini émerge d'une lentille en croisant
            l'axe optique au plan focal image ;

        \item Des rayons se croisant dans le plan focal objet d'une lentille
            émergent parallèles entre eux.
    \end{enumerate}
    Relation de Chasles :
    \[ \obar{O_1O_2} = \obar{O_1F_1'} + \obar{F_1'O_2} \]
\end{NCdemo}

\begin{NCexem}{Application}
    Pour que tous les rayons sortant de la lunette soient parallèles entre eux
    (donnant donc une image à l'infini), il faut que tous les rayons à
    l'intérieur passent par le plan focal objet de son oculaire.\bigbreak

    Or, tous les rayons arrivent dans la lunette parallèles entre eux (objet
    initial à l'infini) ; il se croisent donc dans le plan focal image de
    l'objectif. \bigbreak

    Pour que la condition soit vérifiée, il faut donc simplement que les plans
    focaux image de $L_1$ et objet de $L_2$ soient confondus ; autrement dit :
    \[ \boxed{F_1' = F_2} \]
    
    On a alors $\obar{O_1O_2} = \obar{O_1F_1'} + \obar{F_2O_2}$, et finalement

    \[ \boxed{\obar{O_1O_2} = \SI{+36}{cm}} \]
\end{NCexem}

\subsection{a.}
\begin{defi}{Cercle oculaire}
    On appelle cercle oculaire l'image de la monture de l'objectif donnée par
    l'oculaire.
\end{defi}

\begin{inte}{Utilité du cercle oculaire}
    Il correspond à la section la plus étroite du faisceau sortant de
    l'oculaire, où l'œil reçoit le maximum de lumière.  
\end{inte}

\setcounter{subsection}{2}
\subsection{b.}
\begin{NCprop}{Résultat attendu}
    $$\obar{O_2C_k'}$$
\end{NCprop}

\begin{NCdemo}{Outil du cours}
    Par définition, $C_k'$ est l'image de $O_1$ par $L_2$. On va donc se servir
    de la relation de conjugaison d'une lentille mince :
    \[ \frac{1}{\OF} = \frac{1}{\OAp} - \frac{1}{\OA} \]
\end{NCdemo}

\begin{NCexem}{Application}
    On a ici $O \equiv O_2$, $F' \equiv F_2'$, $A \equiv O_1$ et $A' \equiv
    C_k'$. On a donc :
    \[ \frac{1}{\obar{O_2F_2'}} = \frac{1}{\obar{O_2C_k'}} -
    \frac{1}{\obar{O_2O_1}} \]
    et après calculs :
    \[ \boxed{\obar{O_2C_k'} = \left[ \frac{1}{\obar{O_2O_1}} +
    \frac{1}{\obar{O_2F_2'}}\right]^{-1} = \SI{+4.5}{cm}} \]
\end{NCexem}

\setcounter{subsection}{2}
\subsection{c.}
\begin{NCprop}{Résultat attendu}
    $$D_k'$$
\end{NCprop}

\begin{NCdemo}{Outil du cours}
    Le diamètre du cercle oculaire s'apparente à la taille d'un objet. On peut
    donc utiliser le grandissement :
    \[ \g = \frac{\ABp}{\ABb} = \frac{\OAp}{\OA} \]
\end{NCdemo}

\begin{NCexem}{Application}
    Avec les données de l'énoncé, on obtient :
    \[ \g = \frac{D_k'}{D} = \frac{\obar{O_2C_k'}}{\obar{O_2O_1}}\]
    et finalement
    \[ \boxed{D_k' = D\times \frac{\obar{O_2C_k'}}{\obar{O_2O_1}} =
    \SI{3.75}{mm}} \]
\end{NCexem}

\begin{center}
    \huge Partie 2
\end{center}

\subsection{}
Si l'oculaire est divergent, cela signifie que $C_3 < 0$. On a donc $C_3 = -
C_2$, d'où le résultat demandé.

\subsection{}
On reprend la question 2) b-, avec cette fois des indices « 3 » au lieu des
indices « 2 », et on obtient :

\begin{NCexem}{Application}
    $\obar{O_1O_3} = \obar{O_1F_1'} + \obar{F_3O_3}$ avec $\obar{F_3O_3} =
    \obar{O_3F_3'} = \SI{-4}{cm}$, d'où
    \[ \boxed{\obar{O_1O_3} = \SI{+28}{cm}} \]
\end{NCexem}

\begin{inte}{Intérêt}
    La lunette de Galilée est donc plus compacte que la lunette de Kepler !
\end{inte}

\subsection{Tracé non effectué}

\subsection{a.}
On reprend la question 2) 3- b., avec des indices « 3 » au lieu de « 2 », et on
obtient :
\begin{NCexem}{Application}
    \[ \boxed{\obar{O_3C_k'} = \SI{-3.5}{cm}}\]
\end{NCexem}

\begin{rema}{comparaison}
    On a cette fois un cercle oculaire virtuel. Il faudra placer son œil le plus
    près possible de l'oculaire pour espérer avoir le plus de lumière possible.
\end{rema}

\setcounter{subsection}{6}
\subsection{b.}
On reprend la question 2) 3- c. :
\begin{NCexem}{Application}
    \[ \boxed{D_k' = \SI{3.75}{mm}} \]
\end{NCexem}

\subsection{Comparaison}
\begin{tabularx}{\linewidth}{|Y*{2}{|Y}|}\hline
    \rowcolor{gray!15} & Avantages & Inconvéniants \\\hline
    \cellcolor{gray!15} Lunette Galilée & + compacte\smallbreak image droite &
    cercle oculaire virtuel \\\hline
    \cellcolor{gray!15} Lunette Kepler & Grande clarté \smallbreak Cercle
    oculaire réel & - compacte \smallbreak image renversée \\\hline
\end{tabularx}

\end{document}
