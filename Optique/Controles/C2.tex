\documentclass[french]{article}
\usepackage[T1]{fontenc}
\usepackage[utf8]{inputenc}
\usepackage{lmodern}
\usepackage[a4paper]{geometry}
\usepackage{ntheorem}
\usepackage{babel}
\usepackage{amsmath}
%
\theoremstyle{plain}
\theorembodyfont{\normalfont}
\theoremseparator{~--}
\newtheorem{exo}{Exercice}%[section]
%
\title{Contrôle de TD numéro 2 (Sujet A) - 20 min}
\date{Jeudi 10 octobre}
\author {Nom :\\
\\
Prénom :}

\begin{document}
\maketitle
Tout oubli d'unité ou de chiffres significatifs pourra entrainer la perte de point.
\\Pour les exercices 2-4-5 , il est demandé de fournir une explication détaillée de vos réponses.
\\ Il n'est pas demandé de faire de schémas pour les exercices 4 et 5 sur la feuille. (Mais cela peut être utile sur un brouillon)

\section{TD 1 : Lentilles minces}

\begin{exo}[Formule de grandissement (1 pt)]
 Quelle sont les bonnes égalités de la formule de grandissement ? (une seule réponse possible)
 \\
 \begin{equation*}
 \\
 \vspace{0.5cm}
    $\bigcirc$  $\gamma=  \frac{\overline{A'B'}}{\overline{AB}} = \frac{\overline{OA}}{\overline{OA'}}$
    \\
    \vspace{0.5cm}
    $\bigcirc$  $\gamma =  \frac{\overline{A'B'}}{\overline{AB}} = \frac{\overline{OA'}}{\overline{OA}}$
    \\
    \vspace{0.5cm}
    $\bigcirc$  $\gamma = \frac{\overline{AB}}{\overline{A'B'}} =  \frac{\overline{OA}}{\overline{OA'}}$
    \\
    \vspace{0.5cm}
      $\bigcirc$  $\gamma = \frac{\overline{AB}}{\overline{A'B'}} =  \frac{\overline{OA'}}{\overline{OA}}$
 
 \end{equation*}
\end{exo}

\begin{exo}[Construction optique (3 pts)]
  On forme l'image d'un objet de 1,0 cm placé à 2,0 cm \underline{après} une lentille divergente de distance focale -5,0 cm. 
  \\1) Construire l'image sur un schéma à l'échelle et déterminer la nature de l'objet et de l'image.
  \\ 2) Calculer le grandissement par la méthode de votre choix.
\end{exo}
\vspace{5cm}
\section{TD 3 : Systèmes composés de plusieurs lentilles minces}

\begin{exo}[Formule de grossissement (1 pt)]
Quelle est la bonne formule du grossissement, si $\alpha$ est l'angle sous lequel l'objet est vu par le système optique et $\alpha'$ l'angle sous lequel l'image est vue à travers le système optique ? (une seule réponse possible)
\\
\vspace{0.2cm}
 \begin{equation*}
 \\
 \vspace{0.2cm}
    $\bigcirc$  $G_c = \frac{\alpha}{\alpha'}$
    \\
    \vspace{0.2cm}
    $\bigcirc$  $G_c = \frac{\alpha'}{\alpha}$
 
 \end{equation*}
\end{exo}

\begin{exo}[La lunette de Galilée (5 pts)]
Une lunette astronomique de Galilée est constituée d'un objectif $L_1$ de centre $O_1$ et d'un oculaire $L_2$ de centre $O_2$. L'objectif a une vergence $V_1$ de 4,0 $\delta$ et un diamètre D de 20,0 mm. L'oculaire présente une vergence $V_2$ de \underline{-20,0  $\delta$}. 
\\
1) Calculer les distances focales $f'_1$ et $f'_2$.
\\
2) Définissez le caractère afocal et expliquer son intêret pour un oeil emmétrope (sans défaut).
\\
3) Calculer l'encombrement optique $\overline{O_1O_2}$ de la lunette pour qu'elle soit afocale.
\\
4) Calculer le grossissement $G_c$ de la lunette.
\\
5) Déterminer la nature de l'image $A_1B_1$ pour la lentille $L_1$ et de l'objet $A_1B_1$ pour la lentille $L_2$ .
\end{exo}

\vspace{7cm}


\section{BONUS} 

\begin{exo}[La lunette de Galilée (suite) (2 pts)]
On cherche à déterminer la position et la taille du cercle oculaire de centre $C'_G$. 
\\
6) Rappeler la définition du cercle oculaire et son intérêt.
\\
7) Déterminer la position $\overline{O_2C'_G}$ .
\\
8) Déterminer le diamètre $D_G$ du cercle oculaire.
\end{exo}

\end{document}
