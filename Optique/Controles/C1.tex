\documentclass[french]{article}
\usepackage[T1]{fontenc}
\usepackage[utf8]{inputenc}
\usepackage{lmodern}
\usepackage[a4paper]{geometry}
\usepackage{ntheorem}
\usepackage{babel}
\usepackage{amsmath}
%
\theoremstyle{plain}
\theorembodyfont{\normalfont}
\theoremseparator{~--}
\newtheorem{exo}{Exercice}%[section]
%
\title{Contrôle de TD numéro 1 (Sujet A)}
\date{Jeudi 10 octobre}
\author {Nom :\\
\\
Prénom :}

\begin{document}
\maketitle
Tout oubli d'unité ou de chiffres significatifs pourra entrainer la perte de point, même si la réponse est juste.
\\Pour les exercices 2-4-5 , il est demandé de fournir une explication détaillée de vos réponses.

\section{TD 1 : Lentilles minces}

\begin{exo}[Formule de conjugaison]
 Quelle est la bonne formule de conjugaison ? (Il peut y avoir plusieurs bonnes réponses)
 \\
 \begin{equation*}
 \\
 \vspace{0.5cm}
    $\bigcirc$  $\frac{1}{\overline{OF'}} =  \frac{1}{\overline{A'O}}-
    \frac{1}{\overline{AO}}$
    \\
    \vspace{0.5cm}
    $\bigcirc$  $\frac{1}{\overline{OF'}} =  \frac{1}{\overline{OA}}- \frac{1}{\overline{OA'}}$
    \\
    \vspace{0.5cm}
    $\bigcirc$  $\frac{1}{\overline{OF'}} =  \frac{1}{\overline{OA'}}- \frac{1}{\overline{OA}}$
    \\
    \vspace{0.5cm}
    $\bigcirc$  $\frac{1}{\overline{OF'}} =  \frac{1}{\overline{AO}}- \frac{1}{\overline{A'O}}$
 
 \end{equation*}
\end{exo}

\begin{exo}[Construction optique]
  On forme l'image d'un objet de 1,0 cm placé à 2,0 cm devant une lentille convergente de distance focale 5,0 cm. 
  \\Construire l'image sur un schéma à l'échelle et déterminer la nature de l'objet et de l'image.
  \\ Déterminer le grandissement par la méthode de votre choix.
\end{exo}
\vspace{5cm}
\section{TD 2 : Miroirs et dioptres plans }

\begin{exo}[Relation de Snell-Descartes]
Quelle est la bonne relation de Snell-Descartes ? (Une seule bonne réponse)
\\
\vspace{0.5cm}
 \begin{equation*}
 \\
 \vspace{0.5cm}
    $\bigcirc$  $n_1 \text{cos}(i_2) = n_2 \text{cos}(i_1) $
    \\
    \vspace{0.5cm}
   $\bigcirc$  $n_1 \text{sin}(i_2) = n_2 \text{sin}(i_1) $
    \\
    \vspace{0.5cm}
    $\bigcirc$  $n_1 \text{cos}(i_1) = n_2 \text{cos}(i_2) $
    \\
    \vspace{0.5cm}
   $\bigcirc$  $n_1 \text{sin}(i_1) = n_2 \text{sin}(i_2) $
 
 \end{equation*}
\end{exo}

\begin{exo}[Le pêcheur]
Un pêcheur croit apercevoir un poisson à 20 cm sous sa barque, mais il s'agit en réalité de son image à travers le dioptre air/eau. A quelle distance est réellement le poisson ? (données : $n_{eau}$ = 1,33 et $n_{air}$ = 1,0 )
\end{exo}

\vspace{4cm}


\section{BONUS} 

\begin{exo}[Le bateau de plongée]
Un plongeur plonge dans l'eau ($n_{eau}$= 1,33) exactement sous le $\underline{centre}$ d'un bateau de longueur L = 8,0 m. A cause de l'angle limite atteint par les rayons non-arrêtés par le bateau et issus du plongeur,  un observateur dans l'air ($n_{air}$ = 1) regardant l'eau au niveau de l'avant du bateau ne voit pas le plongeur avant que celui-ci n'atteigne une profondeur $H_0$ pour laquelle les rayons issus du plongeur seront en régime de réfraction.
\\ Calculez la profondeur minimale $H_0$.
\end{exo}

\end{document}


