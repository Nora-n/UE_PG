\documentclass[../main/main.tex]{subfiles}
\begin{document}

\chapter{Grandeurs électriques}\label{ch:O1}

\begin{center}
    \Huge Exercices d'application
\end{center}

\section{Ordres de grandeur}

Cet exercice se concentre sur la notion d'intensité en électricité.

\begin{NCdefi}{Données}
	\begin{enumerate}
        \item « Un générateur délivre une intensité $I = \SI{3.0}{A}$ » :
            $I = \SI{3.0}{A}$ ;
		\item « 1000 électrons » : $N = 1000$ ;
        \item « faire circuler $\SI{1e20}{}$ électrons chaque seconde » : $N =
            \SI{1e20}{}$, $t = \SI{1}{s}$.
	\end{enumerate}
\end{NCdefi}

\begin{NCprop}{Résultats attendus}
    Les trois questions de l'exercice donnent une grandeur électrique et
    attendent de vous le calcul d'une grandeur inconnue. Il va donc falloir
    utiliser les formules précédentes pour exprimer la grandeur inconnue en
    fonction des données du problème.
\end{NCprop}

\begin{NCdemo}{Outil du cours : intensité électrique}
    L'intensité électrique est une grandeur physique décrivant la quantité de
    charges électriques (exprimées en Coulomb, C) passant par un point d'un
    circuit à chaque unité de temps (exprimé en seconde, s) :
	\begin{equation}
		I = \dfrac{Q}{t} \label{eq:1.1.intensite}
	\end{equation}
    L'intensité est ainsi exprimée en Coulomb par seconde, unité que l'on nomme
    l'Ampère (A). Si les charges sont des électrons se déplaçant dans un fil, le
    nombre de charges est :
	\begin{equation}
		Q = N\times e \label{eq:1.1.charge}
	\end{equation}
    où $e = \SI{1.602e-19}{C}$ est la charge de l'électron (en valeur
    absolue).
\end{NCdemo}

Nous voyons donc que le temps, l'intensité et le nombre de charges sont reliées
par les formules \ref{eq:1.1.intensite} et \ref{eq:1.1.charge}.

\begin{NCexem}{Application}
	\begin{enumerate}
		\item Le nombre d'électrons émis chaque seconde est donné par :
		\begin{equation}
			N = \frac{I \times t}{e}
		\end{equation}
		Avec les données du problème, nous avons :
		\begin{equation}
			N = \dfrac{3.0 \times 1}{1.6\times 10^{-19}} = 1.9\times 10^{19}
		\end{equation}
		\item Le temps pour émettre 1000 électrons est donné par :
		\begin{equation}
            t = \dfrac{N\times e}{I} = \dfrac{1000 \times 1.6\times
            10^{-19}}{3.0}\,{\rm s} = \SI{5.3e-17}{s}
		\end{equation}
		\item L'intensité correspondante est :
		\begin{equation}
            I = \dfrac{N\times e}{t} = \dfrac{1.0\times 10^{20} \times 1.6\times
            10^{-19}}{1}\,{\rm A} = \SI{16}{A}
		\end{equation}
	\end{enumerate}
\end{NCexem}

\begin{impo}{Important}
    Dans cet exercice, nous avons dû faire des applications numériques. Il faut
    alors faire attention à deux choses :
	\begin{itemize}
        \item \underline{l'unité} : dès que vous remplacez les grandeurs
            littérales par des valeurs numériques, votre calcul acquiert une
            unité, qui doit apparaître ;
        \item \underline{les chiffres significatifs} : le résultat final doit
            comporter un nombre de chiffres significatifs cohérent avec la
            précision des données utilisées. Par exemple, l'intensité $I =
            \SI{3.0}{A}$ a deux chiffres significatifs, ce qui va limiter la
            précision avec laquelle on va utiliser la charge de l'électron à
            deux chiffres : $e = \SI{1.6e-19}{C}$. Autre cas, quand on vous dit
            « par seconde », le temps $t$ a alors la valeur $t = \SI{1}{s}$, avec
            une précision arbitraire, qui sera limitée par la précision des
            autres données. Il en va de même pour le nombre $N = 1000$
            électrons.
	\end{itemize}
\end{impo}

\section{Potentiels, tensions et courants}

\begin{defi}{Potentiel et tension électrique}
	Dans cet exercice, nous allons appliquer les notions de potentiel et de
    tension électrique, ainsi que celle de sens « conventionnel » du courant.
	\begin{enumerate}
        \item Le potentiel électrique peut être vu comme un équivalent de
            l'altitude en mécanique : si vous êtes en altitude, vous avez le «~
            potentiel~» de tomber et de fournir de l'énergie, emmagasinée
            pendant la chute, en arrivant au sol. Chaque point d'un circuit est
            ainsi à une certaine «~altitude~». Si on considère une particule de
            charge positive, alors cette particule a le comportement intuitif et
            «~tombe~» des potentiels les plus élevés vers les plus bas (le $+$
            repousse le $+$). Si cette particule est chargée négativement, comme
            l'électron, elle «~remonte~» des bas potentiels vers les plus élevés
            (le $+$ attire le $-$).
        \item La tension électrique est la différence de potentiel entre deux
            points d'un circuit. Sa notation est intuitive : $U_{\rm AB}$ est la
            différence de potentiel entre A et B, $V_{\rm A} - V_{\rm B}$. On la
            représente par contre comme une flèche allant de $B$ vers $A$ : la
            flèche suit les potentiels croissants.
        \item Le sens conventionnel du courant positif est alors l'inverse du
            sens de circulation des électrons, car ils sont chargés
            négativement.
	\end{enumerate}
\end{defi}

\subsection{}
Sur le circuit ci-dessous, on a indiqué le sens de circulation des électrons,
donc des potentiels les plus bas vers les plus hauts.

\begin{figure}[h!]
    \begin{center}
        \begin{circuitikz}
            \draw
            (0,2)
            node {$\bullet$}
            node [above left] (A) {A}
            node [below] {$\SI{5.0}{V}$}
                to[R=\raisebox{-.45cm}{$R_1$}, name=R1,
                i<^=$e^-$, !vi]
            (3,2)
            node {$\bullet$}
            node [above left=0.2em] (B) {B}
            node [below left] {$\SI{3.0}{V}$}
                to[R=\raisebox{-0.45cm}{$R$}, name=R,
                i<^=$e^-$, !vi]
            (6,2)
            node {$\bullet$}
            node [above left=0.2em] (C) {C}
            node [below left] {$\SI{2.5}{V}$}
                to[R=\raisebox{-0.45cm}{$R_3$}, name=R3,
                i<^=$e^-$, !vi]
            (9,2)
            node {$\bullet$}
            node [above] (D) {D}
            node [right] {$\SI[parse-numbers=false]{x}{V}$}
                to[R=\shifttext{-1cm}{$R_2$}, name=R2,
                i<^=$e^-$, !vi]
            (9,0)
            node {$\bullet$}
            node [below] (E) {E}
                to[short, name=S1, i<^=$e^-$]
            (6,0)
            node {$\bullet$}
            node [below] (F) {F}
                to[short, name=S2, i<^=$e^-$]
            (3,0)
            node {$\bullet$}
            node [below] (G) {G}
                to[short, name=S3, i<^=$e^-$]
            (0,0)
            node {$\bullet$}
            node [below] (H) {H}
            node [above] {\SI{1.0}{V}};
            \iarronly{R1}\iarronly{R}\iarronly{R3}\iarronly{R2}
            \iarronly{S1}\iarronly{S2}\iarronly{S3}
            \draw
            (3,2) to[R=\shifttext{-0.7cm}{$r$}, name=r, i<^=$e^-$, !vi]
                (3,0);
            \draw
            (6,2) to[R=\shifttext{-0.9cm}{$r_2$}, name=r2, i<^=$e^-$, !vi]
                (6,0);
            \iarronly{r}\iarronly{r2}
        \end{circuitikz}
    \end{center}
\end{figure}
\subsection{}
Sur le circuit suivant, on a indiqué le sens conventionnel du courant positif,
donc des hauts potentiels vers les bas.
\begin{figure}[h!]
    \begin{center}
        \begin{circuitikz}
        \renewcommand{\iarronly}[1]{% name
            \node [currarrow, color=Purple, anchor=center,
            rotate=\ctikzgetdirection{#1-Iarrow}] at (#1-Ipos) {};}
        \ctikzset{bipole current style/.style={color=Purple}}
            \draw
            (0,2)
            node {$\bullet$}
            node [above left] (A) {A}
            node [below] {$\SI{5.0}{V}$}
                to[R=\raisebox{-.45cm}{$R_1$}, name=R1,
                i>^=$I>0$, !vi]
            (3,2)
            node {$\bullet$}
            node [above left=0.2em] (B) {B}
            node [below left] {$\SI{3.0}{V}$}
                to[R=\raisebox{-0.45cm}{$R$}, name=R,
                i>^=$I>0$, !vi]
            (6,2)
            node {$\bullet$}
            node [above left=0.2em] (C) {C}
            node [below left] {$\SI{2.5}{V}$}
                to[R=\raisebox{-0.45cm}{$R_3$}, name=R3,
                i>^=$I>0$, !vi]
            (9,2)
            node {$\bullet$}
            node [above] (D) {D}
            node [right] {$\SI[parse-numbers=false]{x}{V}$}
                to[R=\shifttext{-1cm}{$R_2$}, name=R2,
                i>^=$I>0$, !vi]
            (9,0)
            node {$\bullet$}
            node [below] (E) {E}
                to[short, name=S1, i>^=$I>0$]
            (6,0)
            node {$\bullet$}
            node [below] (F) {F}
                to[short, name=S2, i>^=$I>0$]
            (3,0)
            node {$\bullet$}
            node [below] (G) {G}
                to[short, name=S3, i>^=$I>0$]
            (0,0)
            node {$\bullet$}
            node [below] (H) {H}
            node [above] {\SI{1.0}{V}};
            \iarronly{R1}\iarronly{R}\iarronly{R3}\iarronly{R2}
            \iarronly{S1}\iarronly{S2}\iarronly{S3}
            \draw
            (3,2) to[R=\shifttext{-0.7cm}{$r$}, name=r, i^>=$I>0$, !vi]
                (3,0);
            \draw
            (6,2) to[R=\shifttext{-0.9cm}{$r_2$}, name=r2, i^>=$I>0$, !vi]
                (6,0);
            \iarronly{r}\iarronly{r2}
        \end{circuitikz}
    \end{center}
\end{figure}
\subsection{}
L'analogie de l'altitude pour les potentiels électriques nous donne une
intuition pour la valeur du potentiel au point D : ce potentiel doit être
compris «~entre~» ceux des points C et E, car aucune source extérieure ne peut
permettre de relever ou d'abaisser artificiellement le point D. Par exemple, $x
= \SI{2.0}{V}$.

\subsection{}
Sur le circuit ci-dessous, nous avons fléché les tensions en question.
\begin{figure}[h!]
    \begin{center}
        \begin{circuitikz}
            \draw
            (0,2)
            node [inner sep=0] (A) {$\bullet$}
            node [above left] {A}
            node [below] {\textcolor{orange!70}{$\SI{5.0}{V}$}}
                to[R=\raisebox{-.45cm}{$R_1$}, name=R1,
                i>^=$I>0$, v^<=$U_{\rm AB}$, !vi]
            (3,2)
            node [inner sep=0] (B) {$\bullet$}
            node [above left=0.2em] {B}
            node [below left] {\textcolor{orange!70}{$\SI{3.0}{V}$}}
                to[R=\raisebox{-0.45cm}{$R$}, name=R,
                i>^=$I>0$, v^>=$U_{\rm CB}$, !vi]
            (6,2)
            node [inner sep=0] (C) {$\bullet$}
            node [above left=0.2em] {C}
            node [below left] {\textcolor{orange!70}{$\SI{2.5}{V}$}}
                to[R=\raisebox{-0.45cm}{$R_3$}, name=R3,
                i>^=$I>0$, !vi]
            (9,2)
            node [inner sep=0] (D) {$\bullet$}
            node [above] {D}
            node [right] {\textcolor{orange!70}{
                $\SI[parse-numbers=false]{x}{V}$}}
                to[R=\shifttext{-1cm}{$R_2$}, name=R2,
                i>^=$I>0$, !vi]
            (9,0)
            node [inner sep=0] (E) {$\bullet$}
            node [below] {E}
                to[short, name=S1,
                i>^=$I>0$]
            (6,0)
            node [inner sep=0] (F) {$\bullet$}
            node [below] {F}
                to[short, name=S2,
                i>^=$I>0$, v_<=$U_{\rm FG}$, !vi]
            (3,0)
            node [inner sep=0] (G) {$\bullet$}
            node [below] {G}
                to[short, name=S3, i>^=$I>0$]
            (0,0)
            node [inner sep=0] (H) {$\bullet$}
            node [below] {H}
            node [above] {\textcolor{orange!70}{\SI{1.0}{V}}};
            \iarronly{R1}\iarronly{R}\iarronly{R3}\iarronly{R2}
            \iarronly{S1}\iarronly{S2}\iarronly{S3}
            \varronly{R1}\varronly{R}
            \varronly{S2}
            \draw
            (3,2) to[R=\shifttext{-0.7cm}{$r$}, name=r,
                i>^=$I>0$, v^<=$U_{\rm BG}$, !vi]
            (3,0);
            \draw
            (6,2) to[R=\shifttext{-0.9cm}{$r_2$}, name=r2,
                i>^=$I>0$, v^<=$U_{\rm CF}$, !vi]
            (6,0);
            \iarronly{r}\iarronly{r2}
            \varronly{r}\varronly{r2}
            \draw[-{Latex[length=2mm]}, red!70, dashed]
            (C)
            --++(0.1, 0.7) -- ($(D) + (0.7, 0.7)$)
            --node[midway, sloped, above] {$U_{\rm EC}$}
            ($(E) + (0.7, 0)$)
            -- (E);
            \draw[-{Latex[length=2mm]}, red!70, dashed]
            (H) -- ($(H) + (-0.5, 0)$)
            -- ($(A) + (-0.5, 0.7)$)
            --node[midway, above] {$U_{\rm CH}$}
            ($(C) + (-0.1, 0.7)$)
            -- (C);
            \draw[-{Latex[length=2mm]}, red!70, dashed]
            (E) --++(-0.3, -0.7)
            --node[midway, below] {$U_{\rm AE}$}
            ($(H) - (0.7, 0.7)$) --
            ($(A) - (0.7, 0)$) --
            (A);
        \end{circuitikz}
    \end{center}
\end{figure}
Ces tensions se calculent par différence de potentiel~:

\begin{minipage}{0.5\linewidth}
    \begin{itemize}
        \item $U_{\rm AB} = V_{\rm A} - V_{\rm B} = \SI{2.0}{V}$ ;
        \item $U_{\rm FG} = \SI{0}{V}$ ;
        \item $U_{\rm BG} = \SI{2.0}{V}$ ;
        \item $U_{\rm CB} = \SI{-0.5}{V}$ ;
    \end{itemize}
\end{minipage}
\begin{minipage}{0.5\linewidth}
    \begin{itemize}
        \item $U_{\rm CF} = \SI{1.5}{V}$ ;
        \item $U_{\rm EC} = \SI{-1.5}{V}$ ;
        \item $U_{\rm CH} = \SI{1.5}{V}$ ;
        \item $U_{\rm AE} = \SI{4.0}{V}$.
    \end{itemize}
\end{minipage}

\subsection{}
\begin{NCdemo}{Outils du cours}
    Une tension peut être difficile à calculer au premier coup d'œil. Dans
    ces cas-là, il peut être utile d'utiliser des points intermédiaires dont on
    connaît le potentiel, c'est une forme de composition des vecteurs. Par
    exemple, la tension $U_{\rm AB}$ entre B et A peut être décomposée à l'aide
    d'un point tiers, C, par la formule $U_{\rm AB} = U_{\rm AC} + U_{\rm CB}$.
    On peut le démontrer facilement en remplaçant les tensions par des
    différences de potentiel.
\end{NCdemo}

\begin{NCexem}{Application}
    \begin{itemize}
        \item $U_{\rm AG} = U_{\rm AB} + U_{\rm BG}$
        \item $U_{\rm AD} = U_{\rm AB} + U_{\rm BD}$
    \end{itemize}
\end{NCexem}

\subsection{}
Les points E, F, G et H sont reliés par des fils, sans dipôles intermédiaires.
Ils sont donc au même potentiel ; on appelle de tels points des points
\underline{equipotentiels}.

\begin{impo}{Potentiel dans un circuit}\label{def:potentiel}
    Dans un circuit, tous les points reliés par des fils sans dipôles
    intermédiaires sont au même potentiel. Ils sont en fait considérés comme un
    seul point dans un circuit. Il peut être utile de se servir de cette
    propriété pour redessiner un circuit sous une forme plus simple.
\end{impo}

\section{Schématisation}

\begin{defi}{Résistances équivalentes et associations de résistances}
    Très souvent, un circuit électrique contient de nombreuses résistances. Ces
    résistances peuvent être :
	\begin{itemize}
        \item en série : elles sont sur une même branche, aucune branche tierce
            ne part du point qui les séparent
			\begin{circuitikz}
				\draw (0,0) node {$\bullet$} to[R=\raisebox{-0.45cm}{$R_1$}]
				(2,0) to[R=\raisebox{-0.45cm}{$R_2$}]
				(4,0) node {$\bullet$};
			\end{circuitikz}
        \item en dérivation : elles sont situées sur deux branches connectées à
            leurs extrémités
			\begin{circuitikz}
				\draw (0,0) node {$\bullet$} --
				(1,0) --
				(1,0.5) to[R=\raisebox{-0.45cm}{$R_1$}]
				(3,0.5) --
				(3,0) --
				(4,0) node {$\bullet$};
				\draw (1,0) --
				(1,-0.5) to[R=\raisebox{-0.45cm}{$R_2$}]
				(3,-0.5) --
				(3,0);
			\end{circuitikz}
	\end{itemize}
    On note symboliquement que deux résistances sont en série par $R_1 + R_2$,
    elles sont alors équivalentes à une seule résistance de valeur :
	\begin{equation}
		R_\text{série} = R_1 + R_2
	\end{equation}
    On note symboliquement que deux résistances sont en dérivation par $R_1 \parr 
    R_2$, elles sont alors équivalentes à une seule résistance de valeur :
	\begin{equation}
		R_\text{dérivation} = \dfrac{R_1\times R_2}{R_1 + R_2}
	\end{equation}
\end{defi}

\begin{NCcoro}{Conseils}
    Lors de la transformation d'une formule symbolique d'une association de
    résistances en schéma, il est conseillé de commencer par les parenthèses les
    plus intérieures, puis d'ajouter les éléments en allant vers l'extérieur des
    parenthèses. Comme pour la multiplication $\times$ et l'addition $+$, le
    symbole $\parr$ est prioritaire devant le symbole $+$. N'oubliez pas les
    branches de sortie afin de définir explicitement quel est le dipôle final.
\end{NCcoro}

\begin{NCexem}{Application}
	\begin{itemize}
		\item $[(R+r)\parr (r+r_2)]+r_1$ :\\[-0.5cm]
		\begin{center}
			\begin{circuitikz}
				\draw (0.5,0) node {$\bullet$} --
				(1,0) --
				(1,0.5) to[R=\raisebox{-0.45cm}{$R$}]
				(3,0.5) to[R=\raisebox{-0.45cm}{$r$}]
				(5,0.5) --
				(5,0) to[R=\raisebox{-0.45cm}{$r_1$}]
				(7,0) node {$\bullet$};
				\draw (1,0) --
				(1,-0.5) to[R=\raisebox{-0.45cm}{$r$}]
				(3,-0.5) to[R=\raisebox{-0.45cm}{$r_2$}]
				(5,-0.5) --
				(5,0);
			\end{circuitikz}
		\end{center}
		\item $\{[(R\parr r)+r_1]\parr(r+r_2)\} + (r_1\parr R)$ :\\[-0.5cm]
		\begin{center}
			\begin{circuitikz}
				\draw
                (0,0) node {$\bullet$} --
				(0.5,0) --
				(0.5,1) --
				(1,1) --
				(1,1.5) to[R=\raisebox{-0.45cm}{$R$}]
				(3,1.5) --
				(3,1) to[R=\raisebox{-0.45cm}{$r_1$}]
				(5,1) --
				(5,0) --
				(6,0) --
				(6,0.5) to[R=\raisebox{-0.45cm}{$r_1$}]
				(8,0.5) --
				(8,0) --
				(8.5,0) node {$\bullet$};
				\draw (0.5,0) --
				(0.5,-0.5) to[R=\raisebox{-0.45cm}{$r$}]
				(2.75,-0.5) to[R=\raisebox{-0.45cm}{$r_2$}]
				(5,-0.5) --
				(5,0);
				\draw (1,1) --
				(1,0.5) to[R=\raisebox{-0.45cm}{$r$}]
				(3,0.5) --
				(3,1);
				\draw (6,0) --
				(6,-0.5) to[R=\raisebox{-0.45cm}{$R$}]
				(8,-0.5) --
				(8,0);
			\end{circuitikz}
		\end{center}
	\end{itemize}
\end{NCexem}

\section{Association de résistances}\label{ch1:ex4}

\begin{NCcoro}{Conseils}
    Dans cet exercice, on va exprimer la résistance équivalente d'un circuit,
    symboliquement, à l'aide des signes $\parr $ et $+$. Le plus simple est de
    procéder par étapes afin d'identifier les couples de résistances en série et
    en dérivation, et de les remplacer par leur résistance équivalente. Pour ce
    faire, on se rappelle que d'après la définition \ref{def:potentiel}, on peut
    déplacer les points le long de fils tant qu'on ne traverse pas de dipôle.
\end{NCcoro}

\subsection{}

\begin{itemize}
    \item Dans le schéma 1, $R_1$ et $R$ ne sont ni en série ni en dérivation,
        il y a un nœud entre les deux, qui part sur une autre branche. $r$
        et $r_2$ ne sont pas en parallèle, il y a une résistance ($R$) sur la
        branche transverse. $R_2$ et $R$ : aucun. $R_3$ et $r_2$ : aucun. $R_3$
        et $R_2$ sont en série.
    \item Dans le schéma 2, $R_1$ et $R$ : aucun. $r$ et $r_2$ : aucun. $R_3$ et
        $r_2$ sont en parallèle, ce qui est bien visible si on déplace le point
        B en bas à gauche du schéma.
    \item Dans le schéma 3, $R_1$ et $R$ sont en série. $R_2$ et $R$ : aucun.
        $r$ et $r^\prime$ sont en parallèle, si on prend les deux résistances en
        bas à gauche du schéma.
\end{itemize}

\subsection{Schéma 1}
On va faire en détails le cas du schéma 1.
	\begin{center}
		\begin{circuitikz}
			\draw
            (0,0)
            node {$\bullet$}
            node [left] (A) {A}
                to[R=\raisebox{-0.45cm}{$R_1$}]
			(2,0)
                to[R=\raisebox{-0.45cm}{$R$}]
			(4,0)
            coordinate (C)
                to[R=\raisebox{-0.45cm}{$R_3$}]
			(6,0)
                to[R=\shifttext{-1cm}{$R_2$}]
			(6,-2)
            coordinate (E) --
			(0,-2)
            node {$\bullet$}
            node [left] (B) {B};
			\draw (2,0) to[R=\shifttext{-0.7cm}{$r$}]
			(2,-2);
			\draw (4,0) to[R=\shifttext{-0.9cm}{$r_2$}]
			(4,-2);
            \draw[dashed, ForestGreen]
            ([shift={(0.25,0.25)}]C.center) rectangle
            ([shift={(0.25,-0.25)}]E.center);
		\end{circuitikz}
	\end{center}
Sur ce circuit, $R_3$ et $R_2$ sont en série. Nous les remplaçons par une
résistance équivalente $R_{\rm eq,1} = R_3 + R_2$. Le nouveau schéma est~:
	\begin{center}
		\begin{circuitikz}
			\draw
            (0,0)
            node {$\bullet$}
            node [left] (A) {A}
                to[R=\raisebox{-0.45cm}{$R_1$}]
			(2,0)
            to[R=\raisebox{-0.45cm}{$R$}]
			(4,0)
            coordinate (C) --
			(6,0)
            to[R=\shifttext{-1.2cm}{$R_{\rm eq,1}$}]
			(6,-2) coordinate (D) --
			(0,-2)
            node {$\bullet$}
            node [left] (B) {B};
			\draw
            (2,0)
                to[R=\shifttext{-0.7cm}{$r$}]
			(2,-2);
			\draw
            (4,0)
                to[R=\shifttext{-0.9cm}{$r_2$}]
			(4,-2);
            \draw[dashed, ForestGreen]
            ([shift={(-0.25,0.25)}]C.center) rectangle
            ([shift={(0.25,-0.25)}]D.center);
		\end{circuitikz}
	\end{center}
Maintenant, on voit que $r_2$ et $R_{\rm eq,1}$ sont en parallèle. On les
remplace par la résistance équivalente $R_{\rm eq,2} = r_2 \parr R_{\rm eq,1}$.
Le nouveau schéma est :
	\begin{center}
		\begin{circuitikz}
			\draw
            (0,0)
            node {$\bullet$}
            node [left] (A) {A}
                to[R=\raisebox{-0.45cm}{$R_1$}]
			(2,0)
            coordinate (C)
                to[R=\raisebox{-0.45cm}{$R$}]
			(4,0)
                to[R=\shifttext{-1.2cm}{$R_{\rm eq,2}$}]
            (4,-2)
            coordinate (D) --
			(0,-2)
            node {$\bullet$}
            node [left] (B) {B};
			\draw
            (2,0)
                to[R=\shifttext{-0.7cm}{$r$}]
			(2,-2);
            \draw[dashed, ForestGreen]
            ([shift={(0.25,0.25)}]C.center) rectangle
            ([shift={(0.25,-0.25)}]D.center);
		\end{circuitikz}
	\end{center}
$R$ et $R_{\rm eq,2}$ sont en série, on les remplace par la résistance
équivalente $R_{\rm eq,3} = R + R_{\rm eq,2}$ :
	\begin{center}
		\begin{circuitikz}
			\draw
            (0,0)
            node {$\bullet$}
            node [left] (A) {A}
                to[R=\raisebox{-0.45cm}{$R_1$}]
			(2,0)
            coordinate (C) --
			(4,0)
                to[R=\shifttext{-1.2cm}{$R_{\rm eq,3}$}]
			(4,-2)
            coordinate (D) --
			(0,-2)
            node {$\bullet$}
            node [left] (B) {B};
			\draw (2,0) to[R=\shifttext{-0.7cm}{$r$}]
			(2,-2);
            \draw[dashed, ForestGreen]
            ([shift={(-0.25,0.25)}]C.center) rectangle
            ([shift={(0.25,-0.25)}]D.center);
		\end{circuitikz}
	\end{center}
$r$ et $R_{\rm eq,3}$ sont en parallèle, on les remplace par $R_{\rm eq,4} = r
\parr R_{\rm eq,3}$ :
	\begin{center}
		\begin{circuitikz}
			\draw
            (0,0)
            node {$\bullet$}
            node [left] (A) {A}
                to[R=\raisebox{-0.45cm}{$R_1$}]
			(2,0)
                to[R=\raisebox{-0.45cm}{$R_{\rm eq,4}$}]
			(4,0)
            node {$\bullet$}
            node [right] (B) {B};
            \draw[dashed, ForestGreen]
            ([shift={(0.25,0.25)}]A.east) rectangle
            ([shift={(-0.25,-0.25)}]B.west);
		\end{circuitikz}
	\end{center}
	$R_1$ et $R_{\rm eq,4}$ sont en série. La résistance totale entre A et B est :
	\begin{align}
		R_{\rm eq} &= R_1 + R_{\rm eq,4} \nonumber \\ &= R_1 + (r\parr R_{\rm eq,3}) \nonumber \\
		&= R_1 + (r\parr [R + R_{\rm eq,2}]) \nonumber \\
		&= R_1 + (r\parr [R + \{ r_2 \parr  R_{\rm eq,1} \}]) \nonumber \\
		&= R_1 + (r\parr [R + \{ r_2 \parr  (R_3 + R_2) \}]) \label{eq:1.4.1}
	\end{align}

Une fois ce calcul terminé, on peut vérifier par le chemin inverse que la
schématisation de la résistance équivalente \ref{eq:1.4.1} donne bien le schéma
de départ.

\setcounter{subsection}{1}
\subsection{Schéma 2}

On part du schéma suivant :
\begin{center}
    \begin{circuitikz}[scale=1]
        \draw
        (0,0)
        coordinate (A)
        node [left] {A}
        node {$\bullet$}
            to[R, name=R1]
        (2,0)
            to[R, name=R]
        (4,0)
            to[R, name=R3]
        (6,0)
        coordinate (B)
        node {$\bullet$}
        node [right] {B} --
        (6,-2) --
        (2,-2)
            to[R, name=Rp]
        (0,-2) --
        (0,0);
        \node[] at (R1.center) {$R_1$};
        \node[] at (R.center) {$R$};
        \node[] at (Rp.center) {$R'$};
        \node[] at (R3.center) {$R_3$};
        \draw
        (2,0)
            to[R, name=r]
        (2,-2);
        \node[] at (r.center) {$r$};
        \draw
        (4,0)
            to[R, name=r2]
        (4,-2);
        \node[] at (r2.center) {$r_2$};
    \end{circuitikz}
\end{center}
Comme on l'a vu, en déplaçant le point B en bas à droite du schéma (ce qu'on
peut faire parce que c'est le même fil, ils sont équipotentiels), on fait
apparaître que $R_3$ et $r_2$ sont en parallèle :

\begin{center}
    \begin{circuitikz}[scale=1]
        \draw
        (0,0)
        coordinate (A)
        node [left] {A}
        node {$\bullet$}
            to[R, name=R1]
        (2,0)
            to[R, name=R]
        (4,0)
        coordinate (C) --
        (6,0)
            to[R, name=R3]
        (6,-2)
        coordinate (B)
        node {$\bullet$}
        node [right] {B} --
        (2,-2)
            to[R, name=Rp]
        (0,-2) --
        (0,0);
        \node[] at (R1.center) {$R_1$};
        \node[] at (R.center) {$R$};
        \node[] at (Rp.center) {$R'$};
        \node[] at (R3.center) {$R_3$};
        \draw
        (2,0)
            to[R, name=r]
        (2,-2);
        \node[] at (r.center) {$r$};
        \draw
        (4,0)
            to[R, name=r2]
        (4,-2);
        \node[] at (r2.center) {$r_2$};
        \draw[dashed, ForestGreen]
        ([shift={(-0.25,0.25)}]C.center) rectangle
        ([shift={(0.25,-0.25)}]B.center);
    \end{circuitikz}
\end{center}
On peut donc les remplacer par la résistance équivalente $R_{\rm eq,1}$ telle
que $R_{\rm eq,1} = r_2\parr R_3$ :

\begin{center}
    \begin{circuitikz}[scale=1]
        \draw
        (0,0)
        coordinate (A)
        node [left] {A}
        node {$\bullet$}
            to[R, name=R1]
        (2,0)
        coordinate (C)
            to[R, name=R]
        (4,0)
            to[R, name=Rq1]
        (4,-2)
        coordinate (B)
        node {$\bullet$}
        node [right] {B} --
        (2,-2)
            to[R, name=Rp]
        (0,-2) --
        (0,0);
        \node[] at (R1.center) {$R_1$};
        \node[] at (R.center) {$R$};
        \node[] at (Rp.center) {$R'$};
        \node[rotate=90] at (Rq1.center) {$R_{\rm eq,1}$};
        \draw
        (2,0)
            to[R, name=r]
        (2,-2);
        \node[] at (r.center) {$r$};
        \draw[dashed, ForestGreen]
        ([shift={(0.25,0.25)}]C.center) rectangle
        ([shift={(0.25,-0.25)}]B.center);
    \end{circuitikz}
\end{center}

Pendant ce procédé, on fait bien attention à \underline{conserver le point B}.
On a alors un schéma où $R$ et $R_{eq,1}$ sont en série, et on peut les
remplacer par $R_{eq,2} = R + R_{eq,1}$ :

\begin{center}
    \begin{circuitikz}[scale=1]
        \draw
        (0,0)
        coordinate (A)
        node [left] {A}
        node {$\bullet$}
            to[R, name=R1]
        (2,0)
        coordinate (C) --
        (4,0)
            to[R, name=Rq2]
        (4,-2)
        coordinate (B)
        node {$\bullet$}
        node [right] {B} --
        (2,-2)
            to[R, name=Rp]
        (0,-2) --
        (0,0);
        \node[] at (R1.center) {$R_1$};
        \node[] at (Rp.center) {$R'$};
        \node[rotate=90] at (Rq2.center) {$R_{\rm eq,2}$};
        \draw
        (2,0)
            to[R, name=r]
        (2,-2);
        \node[] at (r.center) {$r$};
        \draw[dashed, ForestGreen]
        ([shift={(-0.25,0.25)}]C.center) rectangle
        ([shift={(0.25,-0.25)}]B.center);
    \end{circuitikz}
\end{center}

On a encore une fois un association en parallèle, et on définit $R_{\rm eq,3} =
r \parr R_{eq,2}$ :

\begin{center}
    \begin{circuitikz}[scale=1]
        \draw
        (0,0)
        coordinate (A)
        node [left] {A}
        node {$\bullet$}
            to[R, name=R1]
        (2,0)
            to[R, name=Rq3]
        (2,-2)
        coordinate (B)
        node {$\bullet$}
        node [right] {B}
            to[R, name=Rp]
        (0,-2) --
        (0,0);
        \node[] at (R1.center) {$R_1$};
        \node[] at (Rp.center) {$R'$};
        \node[rotate=90] at (Rq3.center) {$R_{\rm eq,3}$};
    \end{circuitikz}
\end{center}

C'est à partir de là que l'on voit l'importance de conserver le point B sur le
schéma. En effet, l'énoncé nous demande de déterminer la résistance équivalente
entre les points A et B ; si l'on faisait la mesure avec un Ohmmètre, on
mettrait bien un fil de A à B, sans pouvoir faire disparaître l'un des deux
points à l'intérieur d'une résistance équivalente. En l'occurrence, bien que
\underline{techniquement} $R'$ et $R_{\rm eq,3}$ soient en série, la présence du
point B \textit{quand on calcule la résistance équivalente} revient à ce que B
soit un nœud (duquel partirait un câble qui va à un Ohmmètre). On ne peut donc
que simplifier le schéma pour faire apparaître $R_{\rm eq,4} = R_1 + R_{\rm
eq,3}$ :

\begin{center}
    \begin{circuitikz}
        \draw
        (0,0)
        coordinate (A)
        node [left] {A}
        node {$\bullet$} --
        (2,0)
        to[R, name=Rq4]
        (2,-2)
        coordinate (B)
        node [right] {B}
        node {$\bullet$} --
        (0,-2)
        to[R, name=Rp ]
        (A);
        \node[] at (Rp.center) {$R'$};
        \node[rotate=90] at (Rq4.center) {$ R_{\rm eq,4}$};
    \end{circuitikz}
\end{center}

On a ainsi déterminé que la dernière association à faire est celle de $R'$ avec
$R_{eq,4}$ pour obtenir la résistance équivalente finale $R_{eq}$ :
\begin{circuitikz}
    \draw (0,0) node {$\bullet$} to[R=\raisebox{-.45cm}{$R_{eq}$}]
    (2,0) node {$\bullet$}
    ;
    \node[left] (A) at (0,0) {A};
    \node[right] (B) at (2,0) {B};
\end{circuitikz}

On a alors

\begin{align*}
    R_{eq} &= R' \parr R_{\rm eq,4} \\
           &= R' \parr (R_1 + R_{\rm eq,3}) \\
           &= R' \parr (R_1 + [r \parr R_{\rm eq,2}]) \\
           &= R' \parr (R_1 + [r \parr \{R + R_{\rm eq,1}\}]) \\
           &= R' \parr (R_1 + [r \parr \{R + (r_2 \parr R_3)\}]) \\
\end{align*}

\setcounter{subsection}{1}
\subsection{Schéma 3}

On part du schéma suivant :

\begin{center}
    \begin{circuitikz}
        \draw
        (0,0)
        coordinate (A)
        node [left] {A}
        node {$\bullet$}
            to[R, name=ru]
        (2,0)
        node [] (C) {} --
        (2,-3)
        node [] (F) {}
            to[R, name=rp]
        (0,-3)
        coordinate (B)
        node [left] {B}
        node {$\bullet$} --
        (0,-1.5)
            to[R, name=rd]
        (2,-1.5);
        \draw
        (2,0)
            to[R, name=Rp]
        (4,0)
            to[R, name=R]
        (4,-1.5)
            to[R, name=R1]
        (4,-3);
        \draw
        (4,0) --
        (6,0)
        coordinate (D)
           to[R, name=R2]
        (6,-1.5)
            to[R, name=rr]
        (6,-3)
        coordinate (E) --
        (2,-3);
        \node[] at (ru.center) {$r$};
        \node[] at (rd.center) {$r$};
        \node[] at (rp.center) {$r'$};
        \node[] at (Rp.center) {$R'$};
        \node[] at (R.center) {$R$};
        \node[] at (R1.center) {$R_1$};
        \node[] at (R2.center) {$R_2$};
        \node[] at (rr.center) {$r$};
        \draw[dashed, ForestGreen]
        (C.north west) rectangle
        (F.south east);
        \draw[dashed, orange!70]
        ([shift={(0.25,0.25)}]C.center) rectangle
        ([shift={(0.25,-0.25)}]E.center);
    \end{circuitikz}
\end{center}

Or, la partie encadrée en tirets oranges est court-circuitée par la partie
encadrée en tirets vert : le schéma équivalent se réduit directement à

\begin{minipage}[c]{0.3\linewidth}
    \begin{center}
        \begin{circuitikz}
            \draw
            (0,0)
            coordinate (A)
            node [left] {A}
            node {$\bullet$}
                to[R, name=ru]
            (2,0) --
            (2,-3)
                to[R, name=rp]
            (0,-3)
            coordinate (B)
            node [left] {B}
            node {$\bullet$} --
            (0,-1.5)
                to[R, name=rd]
            (2,-1.5);
            \node[] at (ru.center) {$r$};
            \node[] at (rd.center) {$r$};
            \node[] at (rp.center) {$r'$};
        \end{circuitikz}
    \end{center}    
\end{minipage}
\hfill
\begin{minipage}[c]{0.3\linewidth}
    On remarque qu'on peut faire l'association en parallèle de $r$ et $r'$ :
\end{minipage}
\hfill
\begin{minipage}[c]{0.3\linewidth}
    \begin{center}
        \begin{circuitikz}
            \draw
            (0,0)
            coordinate (A)
            node [left] {A}
            node {$\bullet$}
                to[R, name=ru]
            (2,0) --
            (2,-1.5)
                to[R, name=Rq1]
            (0,-1.5)
            coordinate (B)
            node [left] {B}
            node {$\bullet$};
            \node[] at (ru.center) {$r$};
            \node[] at (Rq1.center) {$ R_{\rm eq,1}$};
        \end{circuitikz}
    \end{center}    
\end{minipage}

Et on obtient donc
\begin{circuitikz}
    \draw
    (0,0)
    node [left] {A}
    node {$\bullet$}
        to[R, name=Rq]
    (2,0)
    node[right] {B}
    node {$\bullet$};
    \node[] at (Rq.center) {$ R_{\rm eq}$};
\end{circuitikz}
avec $\DS R_{\rm eq} = r + ( r \parr r' )$

\section{Problème : puissance et énergie}
La philosophie de la résolution de problème, c'est de soi-même établir la
réflexion « Données/Résultats attendus/Outils » pour la résolution d'un exercice
de physique. On commence par paramétrer le problème. En l'occurrence, on étudie
le fonctionnement d'un ascenseur dans un immeuble : on peut naturellement faire
un schéma. Il faut ensuite déterminer les données pertinentes du sujet, en
l'occurrence la puissance du système électrique qui doit faire fonctionner
l'ascenseur doit permettre de soulever à la fois des êtres humains et la cabine
en elle-même sur plusieurs étages. « Puissance » voulant dire « énergie par
unité de temps », les données pertinentes vont être l'énergie totale à dépenser
pour son fonctionnement, et le temps total de fonctionnement.

Pour la partie énergie, on considère donc la masse totale du système à déplacer.
C'est là que vient la partie résolution de problème : il faut faire une
proposition. Il n'y a pas de réponse unique à un problème. On peut avoir un
grand ascenseur qui peut accueillir 6 personnes et qui a donc une cabine plus
lourde, ou avoir un petit ascenseur avec 2 personnes mais une cage légère. Ce
qui est important c'est la pertinence de la réflexion établie.

Dans l'exemple corrigé sur Claroline, on a considéré un ascenseur qui déplace
deux personnes, de poids de $\SI{75}{kg}$, ainsi que la cabine qui a été estimée
à $\SI{150}{kg}$. Ça paraît faible, mais peu importe, la réflexion est établie ;
si on veut une mesure plus réaliste, on peut chercher un cahier des charges ou
des valeurs classiques en ligne.

Ensuite, l'énergie à dépenser dépend de la hauteur sur laquelle on veut déplacer
cette masse totale. Là il est bon de savoir que dans un immeuble classique, un
étage fait 3 mètres de haut. Si on veut se déplacer de 4 étages, on va compter
12 mètres. Dans la formule de l'EPP, la constante est en effet fixe, mais c'est
nous qui la fixons. Par exemple, je peux dire qu'au rez-de-chaussé, je suis à
une altitude nulle, et que j'ai une énergie potentielle nulle, donc la constante
sera 0 tout au long de l'expérience. Au 4ème étage, on aura une EPP de

\[\mathrm{EPP}_\mathrm{4ème} = mgh_\mathrm{immeuble} + 0\]

avec $m= \SI{300}{kg}$, $g= \SI{9.81}{m.s^-2}$ (ou 10.0 si on fait une
approximation) et $h_\mathrm{immeuble}= \SI{12}{m}$. On aurait pu définir l'EPP
au rez-de-chaussé en considérant que comme nous ne sommes pas au niveau de la
mer, on n'a pas une énergie potentielle nulle. À Lyon on a une altitude moyenne
par rapport à la mer de \SI{237}{m} : on peut considérer qu'au rez-de-chaussé,
même si on considère l'altitude nulle, on a déjà une EPP de $mgh_\mathrm{mer}$.
Si c'est le cas, au 4ème étage on aura une EPP de

\[\mathrm{EPP}_\mathrm{4ème} = mgh_\mathrm{immeuble} + mgh_\mathrm{mer}\]

On se permet cela parce que ce qui nous intéresse, c'est la différence d'énergie
potentielle : dans les deux cas, la différence d'EPP fait bien
$mgh_\mathrm{immeuble}$. La constante ne sert que de point de référence.

Il reste à déterminer le temps de fonctionnement $t_\mathrm{montée}$ : pour
cela, on peut chronométrer soi-même le temps de montée dans un ascenseur. Pour 4
étages, on trouve environ 16 secondes. On se dit que 4 secondes pour un étage
n'est pas aberrant, compte tenu du temps de démarrage et de décélération, mais
encore une fois on peut avoir un ascenseur plus rapide ou plus lent. L'important
c'est d'oser proposer une valeur plausible.

Ainsi, au minimum, la puissance électrique fournie par le système doit être
capable de fournir exactement la puissance nécessaire à déplacer cette masse
pendant ce laps de temps, soit

\[P_\mathrm{elec} \geq P_\mathrm{meca} =
\frac{mgh_\mathrm{immeuble}}{t_\mathrm{montée}}\]

En réalité, il y a toujours des pertes dans les câbles, la dissipation, le
rendement, et globalement un système électrique (moteur) ne débite mécaniquement
que 30\% de sa puissance. Autrement dit, si $P_\mathrm{elec}$ est la puissance
électrique totale de mon système, il n'y a que $0.30\times P_\mathrm{elec}$ qui
est convertie en énergie mécanique. Encore une fois, ce 30\% est assez
classique, mais pour le raisonnement on aurait pu dire 80\% (ça n'arrive jamais
mais on ne peut pas vous en vouloir de ne pas connaître le rendement d'un moteur
en L1), ou 50\%. En tout cas, c'est cette portion de la puissance électrique qui
doit permettre de soulever le système, c'est-à-dire $P_\mathrm{meca}$. On
corrige donc l'équation précédente et on a

\[0.30\times P_\mathrm{elec} = P_\mathrm{meca}\]

Soit, pour conclure,

\[P_\mathrm{elec} = P_\mathrm{meca}/0.30 =
\frac{mgh_\mathrm{immeuble}}{0.30\times t_\mathrm{montée}}\]

Il ne reste plus qu'à faire l'application numérique :

\[ \boxed{P\mathrm{elec} = \SI{2300}{W}} \]

La source de toute la réflexion est partie de notre capacité à déterminer à la
fois les données et le résultat attendu. En l'occurrence, c'est la question que
vous m'avez posée : « comment est on censé trouver une puissance à partir de
l’énergie potentielle de pesanteur ». La réponse se trouve dans la traduction en
« maths » du mot français « puissance » : une puissance c'est une énergie
divisée par un temps. L'intensité électrique est quelque part une puissance :
c'est une énergie électrique (en coulomb, qui ne sont pas des joules) divisée
par un temps.

\end{document}
