\documentclass[../main/main.tex]{subfiles}
\begin{document}

\chapter{Lois de Kirchhoff}\label{ch:O3}
\vspace*{-47pt}

\begin{center}
    \Huge Exercices d'application
\end{center}

\section{Tensions et courants}

\begin{tcbraster}[raster columns=6, raster equal height=rows]
    \begin{NCdefi}[raster multicolumn=4]{Schéma}
        \begin{center}
            \begin{circuitikz}
                \draw
                (0,0)
                    to[R, name=R1, i>_=$I(R_1)$, v^<=$U(R_1)$, !vi]
                (3,0) coordinate (Itl)
                    node {\color{ForestGreen}$\bullet$}
                    node [above] {\color{ForestGreen}1}
                    node [above left=0.4] (M) {}
                    to[R, name=R, i^>=$I(R)$, v^<=$U(R)$, !vi]
                (6,0) coordinate (Itr)
                    node {\color{ForestGreen}$\bullet$}
                    node [above] {\color{ForestGreen}2}
                    to[R, name=R3, i_>=$I(R_3)$, v^<=$U(R_3)$, !vi]
                (9,0)
                    to[R, name=R2, i>^=$I(R_2)$, v^<=$U(R_2)$, !vi]
                (9,-2)
                    node [below right=0.4] (N) {}
                    to[short]
                (6,-2) coordinate (Ibr)
                    node {\color{ForestGreen}$\bullet$}
                    node [below] {\color{ForestGreen}3}
                    to[R, name=Rp, i>_=$I(R')$, v^<=$U(R')$, !vi]
                (3,-2) coordinate (Ibl)
                    node {\color{ForestGreen!40}$\bullet$}
                    to[short]
                (0,-2);
                \draw[]
                (Itl)
                to[R, name=r, i>^=$I(r)$, v^<=$U(r)$, !vi]
                (Ibl);
                \draw[]
                (Itr)
                to[R, name=r2, i>^=$I(r_2)$, v^<=$U(r_2)$, !vi]
                (Ibr);
                \varronly{R} \varronly{R1} \varronly{R2} \varronly{R3}
                \varronly{Rp} \varronly{r} \varronly{r2}
                \iarronly{R} \iarronly{R1} \iarronly{R2} \iarronly{R3}
                \iarronly{Rp} \iarronly{r} \iarronly{r2}
                \node[] at (R.center) {$R$};
                \node[] at (Rp.center) {$R'$};
                \node[] at (R1.center) {$R_1$};
                \node[] at (R2.center) {$R_2$};
                \node[] at (R3.center) {$R_3$};
                \node[] at (r.center) {$r$};
                \node[] at (r2.center) {$r_2$};
                \node[Orchid] (LM1) at (4.5,-1) {1};
                \circledarrow{Orchid}{(LM1)}{0.3}
                \node[Orchid] (LM2) at (7.7,-1) {2};
                \circledarrow{Orchid}{(LM2)}{0.3}
                \draw[dashed, Orchid]
                (M) rectangle
                (N);
                \node[Orchid] (LM3) at ([shift={(0.5,0.3)}]N) {3};
                \circledarrow{Orchid}{(LM3)}{0.3}
            \end{circuitikz}
        \end{center}
    \end{NCdefi}
    \begin{NCrapp}[raster multicolumn=2]{Rappel}
        \begin{itemize}
            \item \textit{Nœud} : jonction entre au moins 3 câbles ;
            \item \textit{Branche} : section entre 2 nœuds ;
            \item \textit{Maille} : ensemble de branches fermées.
        \end{itemize}
    \end{NCrapp}
\end{tcbraster}

\begin{tcbraster}[raster columns=2, raster equal height=rows]
    \begin{NCexem}{Liens entre courants}
        On établit les liens entre les différents courants avec la loi des nœuds
        ou avec l'unicité de l'intensité dans une branche. Ici, on a :
        \begin{itemize}
            \item $I(R_2) = I(R_3)$ par unicité à droite ;
            \item $I(R_1) = I(R) + I(r)$ par LdN 1 ;
            \item $I(R) = I(R_3) + I(r_2)$ par LdN 2 ;
            \item $I(R_2) + I(r_2) = I(R')$ par LdN 3.
        \end{itemize}
        Le dernier nœud, non numéroté, donne une relation redondante avec les
        autres.
    \end{NCexem}
    \begin{NCcexe}{Liens entre tensions}
        Avec les lois des mailles, on a :
        \begin{itemize}
            \item $U(R') + U(r_2) + U(R) = U(r)$ par LdM 1 ;
            \item $U(R_2) + U(R_3) = U(r_2)$ par LdM 2 ;
        \end{itemize}
        La LdM 3 donne une relation redondante avec les deux premières : $U(R')
        + U(R_2) + U(R_3) + U(R) = U(r)$ est la somme des deux.
    \end{NCcexe}
\end{tcbraster}

\section{Calcul d'intensité}
\begin{tcbraster}[raster columns=5, raster equal height=rows]
    \begin{NCdefi}[raster multicolumn=3]{Schéma}
        \begin{center}
            \begin{circuitikz}
                \draw
                (0,0)
                    to[R, name=r1, i>_=$I$, v^<=$U(r_1)$, !vi]
                (3,0)
                    coordinate (It)
                    node {\color{ForestGreen}$\bullet$}
                    node [above] {\color{ForestGreen}1}
                    to[R, name=R, i_>=$I_2$, v^<=$U(R)$, !vi]
                (6,0)
                    to[R, name=R2,
                        v^<={{{{\rotatebox{90}{$U(R_2)$}}}}}, !vi]
                (6,-3)
                    to[short]
                (3,-3)
                    coordinate (Ib)
                    node {\color{ForestGreen!70}$\bullet$}
                    to[short]
                (0,-3)
                    to[V, name=E1, V^>=$E_{1}$, !vi]
                (0,0);
                \draw[]
                (It)
                to[R, name=R1, i>_=$I_1$,
                        v^<={{{{\rotatebox{90}{$U(R_1)$}}}}}, !vi]
                (Ib);
                \varronly{E1} \varronly{R} \varronly{R1} \varronly{R2}
                \varronly{r1}
                \iarronly{R} \iarronly{R1} \iarronly{r1}
                \node[] at (r1.center) {$r_1$};
                \node[] at (R.center) {$R$};
                \node[] at (R1.center) {$R_1$};
                \node[] at (R2.center) {$R_2$};
                \node[Orchid] (LM1) at (1.5,-1.5) {1};
                \circledarrow{Orchid}{(LM1)}{0.3}
                \node[Orchid] (LM2) at (4.5,-1.5) {2};
                \circledarrow{Orchid}{(LM2)}{0.3}
            \end{circuitikz}
        \end{center}
    \end{NCdefi}
    \begin{tcolorbox}[blankest, raster multicolumn=2, space to=\myspace]
        \begin{tcbraster}[raster columns=1]
            \begin{NCprop}[add to natural height=\myspace]{Résultat attendu}
                On cherche à exprimer $I_2$.
            \end{NCprop}
            \begin{NCdemo}{{LdN, LdM}}
                \begin{itemize}
                    \item $I = I_1 + I_2 \color{ForestGreen}(1)$ (LdN) ;
                    \item $I_1R_1 + Ir_1 = E_1 \color{ForestGreen}(2)$ (LdM 1) ;
                    \item $I_2(R+R_2) = I_1R_1 \color{ForestGreen}(3)$ (LdM 2).
                \end{itemize}
            \end{NCdemo}
        \end{tcbraster}
    \end{tcolorbox}
\end{tcbraster}
\begin{tcbraster}[raster columns=2, raster equal height=rows]
    \begin{NCimpo}{Conseil}
        À partir de cet exercice, la résolution des devient assez
        mathématique et devient de la résolution de systèmes d'équations. Il y a
        de bonnes pratiques à cet égard : numéroter les équations qu'on veut
        réutiliser en premier lieu, à l'aide des $\color{ForestGreen}(1)$ par
        exemple, savoir qu'un système de 3 équations (indépendantes) à 3
        inconnues \underline{est} résolvable ensuite, et comprendre comment s'y
        prendre enfin. Cette dernière partie est bien sûr la vraie étape
        difficile et passe par la pratique, mais elle s'apprend. Je donne
        quelques outils dans l'encadré suivant.
    \end{NCimpo}
    \begin{NCrema}{Exemple}
        En ayant simplement utilisé les lois de Kirchhoff dont on dispose, $I_2$
        apparaît dans l'équation $ \color{ForestGreen}(3)$, mais s'exprime en
        fonction de $I_1$ que l'on ne connaît pas non plus. On doit donc
        commencer par trouver une expression de $I_1$ qui puisse nous faire
        avancer. $I_1$ fait partie de l'équation $\color{ForestGreen}(2)$ qui,
        elle, dépend de I mais en utilisant $\color{ForestGreen}(1)$ on peut
        facilement changer $ \color{ForestGreen}(2)$ en une nouvelle équation
        reliant $I_1$ à $I_2$ \underline{et qui n'est pas $\color{ForestGreen}
        (3)$} et qu'on appellera brillamment $ \color{ForestGreen}(4)$. Ainsi, 
        en réinjectant \textcolor{ForestGreen}{(4)} dans
        \textcolor{ForestGreen}{(3)}, on aura une expression de $I_2$ en
        fonction uniquement des paramètres du circuit ($E, R$).
    \end{NCrema}
\end{tcbraster}

\begin{center}
    \begin{NCexem}[width=.7\linewidth]{Application}
        Injecter \textcolor{ForestGreen}{(1)} dans \textcolor{ForestGreen}{(2)}
        donne :
        \begin{align*}
            I_1R_1 + (I_1-I_2)r_1 &= E_1\\
            I_1(R_1+r_1) &= E_1-I_2r_1\\
            I_1 &= \frac{E_1 - I_2r_1}{R_1 + r_1} \quad \color{ForestGreen}(4)
        \end{align*}
        Ainsi, il suffit de réinjecter \textcolor{ForestGreen}{(4)} dans
        \textcolor{ForestGreen}{(3)} pour avoir :
        \begin{align*}
            I_2(R_2+R) &= \frac{E_1 - I_2r_1}{R_1 + r_1}\times R_1\\
            I_2(R_2+R)\times(R_1+r) &= (E_1-I_2r)\times R_1\\
            I_2 \left[ (R_2+R)(R_1+r_1)+r_1R_1 \right] &= E_1R_1
        \end{align*}
        et finalement
        \[\boxed{I_2 = \frac{E_1R_1}{\left[ (R_2+R)(R_1+r_1)+r_1R_1 \right]}}\]
    \end{NCexem}
\end{center}
\vfill

\section{Diviseur de tension}
\begin{tcbraster}[raster columns=3, raster equal height=rows]
    \begin{NCdefi}{Schéma}
        \begin{center}
            \begin{circuitikz}
                \draw
                (0,0)
                to[V, name=E, V^>=$E_{}$, i^>=$I_{}$, !vi]
                (0,4) --
                (1,4) coordinate (It)
                to[R, name=R2, i^>=$I$, v^<=$U_{AB}$, !vi]
                (1,2) coordinate (Im)
                to[R, name=R1, v^<=$U_{BC}$, !vi]
                (1,0) coordinate (Ib) --
                (0,0);
                \varronly{E} \varronly{R1} \varronly{R2}
                \iarronly{E} \iarronly{R2}
                \node[] at (R1.center) {$R_1$};
                \node[] at (R2.center) {$R_2$};
                \draw[]
                (It) --++
                (0.5,0) node [right] {A};
                \draw[]
                (Im) --++
                (0.5,0) node [right] {B};
                \draw[]
                (Ib) --++
                (0.5,0) node [right] {C};
            \end{circuitikz} 
        \end{center}
    \end{NCdefi}
    \begin{tcolorbox}[blankest, raster multicolumn=1, space to=\myspace]
        \begin{tcbraster}[raster columns=1]
            \begin{NCprop}{Résultat attendu}
                On cherche $I$ puis $U_{BC}$.
            \end{NCprop}
            \begin{NCdemo}[add to natural height=\myspace]{Outils}
                \begin{itemize}
                    \item Loi des mailles pour $I$ ;
                    \item Loi d'Ohm pour $U_{BC}$.
                \end{itemize}
            \end{NCdemo}
        \end{tcbraster}
    \end{tcolorbox}
    \begin{NCexem}{Application}
        Il suffit d'une loi des mailles pour trouver
        \[I = \frac{E}{R_1+R_2}\]
        Puis trivialement
        \[U_{BC} = R_1I = \frac{R_1}{R_1+R_2}E\]
    \end{NCexem}
\end{tcbraster}
\begin{tcbraster}[raster columns=2, raster equal height=rows]
    \begin{NCrema}{Remarque}
        On remarque donc que deux dipôles de résistances $R_1$ et $R_2$ se
        partageant une tension totale $E$ vont se la répartir
        en respectant la fraction de résistance à laquelle chaque diôle
        participe. C'est également une simple moyenne pondérée.
    \end{NCrema}
    \begin{NCror}{Important}
        Ce résultat est bien plus général que pour deux dipôles et fonctionne
        avec $n$ dipôles \textbf{\textit{en série}} sur une branche. Il faut
        pouvoir se ramener à ce schéma précis pour appliquer la formule du pont
        diviseur de tension – que vous pouvez maintenant utiliser sans loi des
        mailles : $\DS U_x = E\times \frac{R_x}{R_\mathrm{tot}}$
    \end{NCror}
\end{tcbraster}

\begin{tcbraster}[raster columns=3, raster equal height=rows]
    \begin{NCdefi}{Schéma}
        \begin{center}
            \begin{circuitikz}
                \draw
                (0,0)
                to[V, name=E, V^>=$E_{}$, i^>=$I_{}$, !vi]
                (0,4) --
                (1,4) coordinate (It)
                to[R, name=R2, i^>=$I$, v^<=$U_{AB}$, !vi]
                (1,2) coordinate (im)
                to[R, name=R1, i>^=$I_1$, v_<=$U_{BC}$, !vi]
                (1,0) coordinate (ib) --
                (0,0);
                \draw[]
                (im)
                to[short]
                ($(im)+(1,0)$) coordinate (Im)
                to[R, name=R3, i>_=$I_3$, v^<=$U_{BC}$, !vi]
                ($(ib)+(1,0)$) coordinate (Ib) --
                (ib);
                \varronly{E} \varronly{R1} \varronly{R2} \varronly{R3}
                \iarronly{E} \iarronly{R2} \iarronly{R3} \iarronly{R1}
                \node[] at (R1.center) {$R_1$};
                \node[] at (R2.center) {$R_2$};
                \node[] at (R3.center) {$R_3$};
                \draw[]
                (It) --++
                (0.5,0) node [right] {A};
                \draw[]
                (Im) --++
                (0.5,0) node [right] {B};
                \draw[]
                (Ib) --++
                (0.5,0) node [right] {C};
            \end{circuitikz} 
        \end{center}
    \end{NCdefi}
    \begin{tcolorbox}[blankest, raster multicolumn=1, space to=\myspace]
        \begin{tcbraster}[raster columns=1]
            \begin{NCexem}{Réponse}
                Oui, elle va changer puisqu'on a branché un nouveau dipôle.
            \end{NCexem}
            \begin{NCprop}{Résultat attendu}
                On cherche $I$ et $U_{BC}$.
            \end{NCprop}
            \begin{NCdemo}[add to natural height=\myspace]{Outils}
                \begin{itemize}
                    \item Loi des mailles pour $I$ ;
                    \item Loi d'Ohm pour $U_{BC}$.
                \end{itemize}
            \end{NCdemo}
        \end{tcbraster}
    \end{tcolorbox}
    \begin{NCinte}{Schéma simplifié}
        \begin{center}
            \begin{circuitikz}
                \draw
                (0,0)
                to[V, name=E, V^>=$E_{}$, i^>=$I_{}$, !vi]
                (0,4) --
                (1,4) coordinate (It)
                to[R, name=R2, i^>=$I$, v^<=$U_{AB}$, !vi]
                (1,2) coordinate (Im)
                to[R, name=Req, v^<=$U_{BC}$, !vi]
                (1,0) coordinate (Ib) --
                (0,0);
                \varronly{E} \varronly{Req} \varronly{R2}
                \iarronly{E} \iarronly{R2}
                \node[] at (Req.center) {$R_{\rm eq}$};
                \node[] at (R2.center) {$R_2$};
                \draw[]
                (It) --++
                (0.5,0) node [right] {A};
                \draw[]
                (Im) --++
                (0.5,0) node [right] {B};
                \draw[]
                (Ib) --++
                (0.5,0) node [right] {C};
            \end{circuitikz} 
        \end{center}
    \end{NCinte}
\end{tcbraster}
\begin{center}
    \begin{NCexem}[width=0.8\linewidth]{Application}
        On peut envisager ce calcul de deux manières :
        \begin{itemize}

            \item D'une part, $U_{BC} = R_1 I_1$ par exemple. Il nous faudrait
                déterminer $I_1$ en fonction de $I$ avec la loi des nœuds pour
                ça. Or, la meilleure manière de déterminer $I$ c'est de se
                ramener à une seule maille en calculant la résistance
                équivalente comme on vient de faire, et dans ce cas :

            \item On voit immédiatement que $U_{BC} = R_{\rm eq}I$. Autant ne
                pas se compliquer la tâche et partir là-dessus.
        \end{itemize}
        On obtient ainsi \[ R_{\rm eq} = \frac{R_1R_3}{R_1+R_3} \quad \text{et}
        \quad I = \frac{E}{R_2 + \frac{R_1R_3}{R_1+R_3}}\] d'où après calcul
        \[\boxed{U_{BC} = \frac{ER_1R_3}{R_2(R_1+R_3t)+R_1R_3}} \quad \text{ou}
        \quad \boxed{U_{BC} = \frac{E}{\frac{R_2}{R_3} + \frac{R_2}{R_1} + 1}}\]
    \end{NCexem}
\end{center}

\section{Diviseur de courant}
\begin{tcbraster}[raster columns=3, raster equal height=rows]
    \begin{NCdefi}{Schéma}
        \begin{center}
            \begin{circuitikz}
                \draw
                (0,0)
                to[I, name=I, i^>=$I_{}$, !vi]
                (0,2) --
                (1,2) coordinate (Itl)
                to[R, name=R1, i>_=$I_1$,
                    v^<={{{{\rotatebox[origin=c]{90}{$U_{R_1}$}}}}}, !vi]
                (1,0) coordinate (Ibl) --
                (0,0);
                \draw[]
                (Itl) --
                (2,2) coordinate (Itr) --
                (3,2) node [right] (A) {A};
                \draw[]
                (Ibl) --
                (2,0) coordinate (Ibr) --
                (3,0) node [right] (B) {B};
                \draw[red, -Triangle]
                (B) -- node[below, midway, sloped] {$U_{AB}$}
                (A);
                \draw[]
                (Itr)
                to[R, name=R2, i>_=$I_2$,
                    v^<={{{{\rotatebox[origin=c]{90}{$U_{R_2}$}}}}}, !vi]
                (Ibr);
                \varronly{R1} \varronly{R2}
                \iarronly{I} \iarronly{R1} \iarronly{R2}
                \node[] at (R1.center) {$R_1$};
                \node[] at (R2.center) {$R_2$};
            \end{circuitikz} 
        \end{center}
    \end{NCdefi}
    \begin{tcolorbox}[blankest, raster multicolumn=1, space to=\myspace]
        \begin{tcbraster}[raster columns=1]
            \begin{NCprop}{Résultat attendu}
                On cherche $U_{R_1}$ et $U_{R_2}$.
            \end{NCprop}
            \begin{NCdemo}[add to natural height=\myspace]{Outils}
                \begin{itemize}
                    \item Unicité de la tension en parallèle ;
                    \item Expression résistance $\parr$.
                \end{itemize}
            \end{NCdemo}
        \end{tcbraster}
    \end{tcolorbox}
    \begin{NCinte}{Schéma simplifié}
        \begin{center}
            \begin{circuitikz}
                \draw
                (0,0)
                to[I, name=I, i^>=$I_{}$, !vi]
                (0,2) --
                (1.5,2) coordinate (Itl)
                to[R, name=Req, i>_=$I$,
                    v^<={{{{\rotatebox[origin=c]{90}{$U_{AB}$}}}}}, !vi]
                (1.5,0) coordinate (Ibl) --
                (0,0);
                \draw[]
                (Itl) --
                (3,2) node [right] (A) {A};
                \draw[]
                (Ibl) --
                (3,0) node [right] (B) {B};
                (Ibr);
                \varronly{Req}
                \iarronly{I} \iarronly{Req}
                \node[rotate=90] at (Req.center) {$R_{\rm eq}$};
            \end{circuitikz} 
        \end{center}
    \end{NCinte}
\end{tcbraster}
\begin{tcbraster}[raster columns=2, raster equal height=rows]
    \begin{NCexem}{Application}
        On a certes $U_{R_1} = I_1R_1$ et $U_{R_2} = I_2R_2$, mais comme on a
        $U_{R_1} = U_{R_2} = U_{AB}$, le plus simple est de déterminer $U_{AB}$.
        Une résistance équivalente $R_{\rm eq} = \frac{R_1R_2}{R_1+R_2}$ avec
        l'intensité $I$ qui est connue (car imposée par le générateur de
        courant) donne facilement \[U_{R_1} = U_{R_2} = R_{\rm eq}I =
        \frac{R_1R_2}{R_1+R_2}I\]
    \end{NCexem}
    \begin{NCror}{Important}
        Ce résultat est la base de la réflexion menant à l'expression du
        diviseur de courant qui donne l'expression de $I_x$ : on voit
        directement apparaître que $\DS I_x = I\times\frac{R_{\rm eq}}{R_x}$ de par
        l'unicité de la tension. Souvenez-vous de cette simplicité.
    \end{NCror}
\end{tcbraster}

\begin{tcbraster}[raster columns=3, raster equal height=rows]
    \begin{NCprop}{Résultat attendu}
        On cherche $I_2$ en fonction de $I, R_1, R_2$ \textbf{à partir de la loi
        des mailles}.
    \end{NCprop}
    \begin{NCdemo}{Outils}
        -- LdM : $I_1R_1 = I_2R_2 \quad \color{ForestGreen}(1)$ ;
        \smallbreak
        -- LdN : $I = I_1 + I_2 \quad \color{ForestGreen}(2)$.
    \end{NCdemo}
    \begin{NCexem}{Application}
        En utilisant \textcolor{ForestGreen}{(2)} dans
        \textcolor{ForestGreen}{(1)}, on a $I_2R_2 = (I-I_2)R_1$, donc en
        isolant $I_2$ on obtient facilement \[\boxed{I_2 = I
        \frac{R_1}{R_1+R_2}}\]
    \end{NCexem}
\end{tcbraster}
\begin{tcbraster}[raster columns=3, raster equal height=rows]
    \begin{NCdefi}{Schéma}
        \begin{center}
            \begin{circuitikz}
                \draw
                (0,0)
                to[I, name=I, i^>=$I_{}$, !vi]
                (0,2) --
                (1,2) coordinate (Itl)
                to[R, name=R1, i>_=$I_1$,
                    v^<={{{{\rotatebox[origin=c]{90}{$U_{R_1}$}}}}}, !vi]
                (1,0) coordinate (Ibl) --
                (0,0);
                \draw[]
                (Itl) --
                (2,2) coordinate (Itm)
                to[R, name=R2, i>_=$I_2$,
                    v^<={{{{\rotatebox[origin=c]{90}{$U_{R_2}$}}}}}, !vi]
                (2,0) coordinate (Ibm) --
                (0,0);
                \draw[]
                (Itm) --
                (3,2) coordinate (Itr)
                to[R, name=R3, i>_=$I_3$,
                    v^<={{{{\rotatebox[origin=c]{90}{$U_{R_3}$}}}}}, !vi]
                (3,0) coordinate (Ibr) --
                (Ibm);
                \draw[]
                (Itr) --
                (4,2) node [right] (A) {A};
                \draw[]
                (Ibr) --
                (4,0) node [right] (B) {B};
                \draw[red, -Triangle]
                (B) -- node[below, midway, sloped] {$U_{AB}$}
                (A);
                \varronly{R1} \varronly{R2} \varronly{R3}
                \iarronly{I} \iarronly{R1} \iarronly{R2} \iarronly{R3}
                \node[] at (R1.center) {$R_1$};
                \node[] at (R2.center) {$R_2$};
                \node[] at (R3.center) {$R_3$};
            \end{circuitikz}
        \end{center}
    \end{NCdefi}
    \begin{NCprop}{Résultat attendu}
        Évidemment, $I_2$ va changer puisqu'on branche un nouveau dipôle en
        parallèle. Une rivière qui se divise en 3 plutôt qu'en 2 va avoir des
        débits différents dans les deux situations. Donc on cherche $I_2$ en
        fonction de $I, R_1, R_2, R_3$ \textbf{sans méthode imposée}.
    \end{NCprop}
    \begin{NCexem}{Application}
        Avec la réflexion de la question 1 ou la relation du pont diviseur de
        courant qui est maintenant utilisable à volonté, on a facilement $I_2 =
        I\times \frac{R_{\rm eq}}{R_2}$. Avec $R_{\rm eq} =
        \frac{R_1R_2R_3}{R_1R_2+R_1R_3+R_2R_3}$, on a finalement
        \[\boxed{I_2 = I \times \frac{R_1R_3}{R_1R_2+R_1R_3+R_2R_3}}\]
    \end{NCexem}
\end{tcbraster}
\begin{tcbraster}[raster columns=2, raster equal height=rows]
    \begin{NCrema}{Remarque}
        L'intensité $I$ ne va pas changer, puisque c'est celle que l'on fixe
        avec le générateur.
    \end{NCrema}
    \begin{NCror}{Important}
        Bien que la loi des mailles soit l'origine de nombreuses relations, ici
        c'est la simple unicité de la tension qui amène au diviseur de courant.
    \end{NCror}
\end{tcbraster}
\vfill

\section{Association de générateurs}
\begin{tcbraster}[raster columns=2, raster equal height=rows]
    \begin{NCdefi}{Schéma}
        \begin{center}
            \begin{circuitikz}
                \draw
                (0,0)
                to[V, name=E1, V^>=$E_{1}$, i^>=$I_{1}$, !vi]
                (0,2)
                to[R, name=r1, i_>=$I_1$, v^<=$U_{r_1}$, !vi]
                (0,4) --
                (2,4) coordinate (N1)
                to[R, name=r2, i<^=$I_2$, v^>=$U_{r_2}$, !vi]
                (2,2)
                to[V, name=E2, V^<=$E_{2}$, i<_=$I_{2}$, !vi]
                (2,0) coordinate (N2) --
                (0,0);
                \draw[]
                (N1)
                    node {\color{ForestGreen}$\bullet$}
                    node [above] {\color{ForestGreen}1} --++
                (2,0) --++
                (0,-1)
                to[R, name=R4, i>_=$I_4$, v^<=$U_{R_4}$, !vi]
                (4,1) --
                (4,0) --
                (N2);
                \varronly{E1} \varronly{r1} \varronly{r2} \varronly{E2}
                \varronly{R4}
                \iarronly{E1} \iarronly{r1} \iarronly{r2} \iarronly{E2}
                \iarronly{R4}
                \node[] at (r1.center) {$r_1$};
                \node[] at (r2.center) {$r_2$};
                \node[] at (R4.center) {$R_4$};
                \node[Orchid] (LM1) at (1,2) {1};
                \circledarrow{Orchid}{(LM1)}{0.4}
                \node[Orchid] (LM2) at (3,2) {2};
                \circledarrow{Orchid}{(LM2)}{0.4}
            \end{circuitikz} 
        \end{center}
    \end{NCdefi}
    \begin{tcolorbox}[blankest, space to=\myspace]
        \begin{tcbraster}[raster columns=1]
            \begin{NCprop}[add to natural height=\myspace]{Résultat attendu}
                On cherche $I_4$ puis $U_4 = R_4I_4$.
            \end{NCprop}
            \begin{NCdemo}{Outils}
                \begin{itemize}
                    \item LdM 1 : $I_4R_4 + I_1r_1 = E_1 \quad \color{ForestGreen}(1)$ ;
                    \item LdM 2 : $I_4R_4 + I_2r_2 = E_2 \quad \color{ForestGreen}(2)$ ;
                    \item LdN 1 : $I_1 + I_2 = I_4 \quad \color{ForestGreen}(3)$.
                \end{itemize}
            \end{NCdemo} 
        \end{tcbraster}
    \end{tcolorbox}
    \begin{NCror}{Approche méthodique}
        Notre but est de trouver une équation contenant $I_4$ et des valeurs
        connues, c'est-à-dire tout sauf $I_1, I_2$.
        \bigbreak
        L'équation \textcolor{ForestGreen}{(1)} peut nous aider ; on peut la
        transformer en remplaçant $I_1$ par $I_4-I_2$ grâce à
        \textcolor{ForestGreen}{(3)} pour avoir une équation
        \textcolor{ForestGreen}{(4)} avec $I_4$ et $I_2$.
        \bigbreak
        Mais comme \textcolor{ForestGreen}{(2)} nous permet d'isoler $I_2$ et de
        l'exprimer en fonction de $I_4$, en injectant cette expression dans
        \textcolor{ForestGreen}{(4)} on obtient une équation entre $I_4$ et les
        éléments du circuit. Question résolue !
    \end{NCror}
    \begin{NCexem}{Application}
        Avec \textcolor{ForestGreen}{(3)} dans \textcolor{ForestGreen}{(1)} :
        \[I_4R_4 + (I_4-I_2)r_1 = E_1 \quad \color{ForestGreen}(4)\]
        En réexprimant \textcolor{ForestGreen}{(2)} :
        \[I_2 = (E_2 - I_4R_4)/r_2\]
        En injectant \textcolor{ForestGreen}{(2)} dans
        \textcolor{ForestGreen}{(4)} :
        \begin{align*}
            I_4(R_4+r_1) - (E_2-I_4R_4) \frac{r_1}{\color{brandeisblue}r_2}
                &= E_1\\
                \Leftrightarrow I_4(\textcolor{orange}{R_4} +
                                    \textcolor{red}{r_1})
                                    {\color{brandeisblue}r_2}
                                    -
                                    (E_2-I_4\textcolor{orange}{R_4})
                                    \textcolor{red}{r_1}
                &= E_1{\color{brandeisblue}r_2}\\
            \Leftrightarrow I_4(\textcolor{red}{r_1}
                                \textcolor{brandeisblue}{r_2} +
                                \textcolor{red}{r_1}\textcolor{orange}{R_4} +
                                \textcolor{brandeisblue}{r_2}\textcolor{orange}{R_4})
                &= E_1r_2 +E_2r_1
        \end{align*}
        Soit
        \[\boxed{I_4 = \frac{E_1r_2 + E_2r_1}{r_1r_2+r_1R_4+r_2R_4}} \quad
        \text{et} \quad \boxed{U_{R_4} = R_4\times I_4}\]
    \end{NCexem}
\end{tcbraster}
\vfill
\newpage

\section{Diviseurs de tension vus en TP}
\begin{NCdefi}[tabularx={Y|Y|Y}]{Schémas}
    \begin{center}
        \begin{circuitikz}
            \draw
            (0,0)
            to[V, name=E, V^>=$E_{}$, i^>=$I_{}$, !vi]
            (0,4) --
            (1,4) coordinate (It)
            to[R, name=R2, i^>=$I$, v^<=$U_{AB}$, !vi]
            (1,2) coordinate (Im)
            to[R, name=R1, v^<=$U_{BC}$, !vi]
            (1,0) coordinate (Ib) --
            (0,0);
            \varronly{E} \varronly{R1} \varronly{R2}
            \iarronly{E} \iarronly{R2}
            \node[] at (R1.center) {$R_1$};
            \node[] at (R2.center) {$R_2$};
            \draw[]
            (It) --++
            (0.5,0) node [right] {A};
            \draw[]
            (Im) --++
            (0.5,0) node [right] {B};
            \draw[]
            (Ib) --++
            (0.5,0) node [right] {C};
        \end{circuitikz} 
    \end{center}
    &
    \begin{center}
        \begin{circuitikz}
            \draw
            (0,0)
            to[V, name=E, V^>=$E$, i^>=$I_{}$, !vi]
            (0,2) --
            (1,2) coordinate (Itl)
            to[R, name=R1, i>_=$I_1$,
                v^<={{{{\rotatebox[origin=c]{90}{$U_{R_1}$}}}}}, !vi]
            (1,0) coordinate (Ibl) --
            (0,0);
            \draw[]
            (Itl) --
            (2,2) coordinate (Itr) --
            (3,2) node [right] (A) {A};
            \draw[]
            (Ibl) --
            (2,0) coordinate (Ibr) --
            (3,0) node [right] (B) {B};
            \draw[red, -Triangle]
            (B) -- node[below, midway, sloped] {$U_{AB}$}
            (A);
            \draw[]
            (Itr)
            to[R, name=R2, i>_=$I_2$,
                v^<={{{{\rotatebox[origin=c]{90}{$U_{R_2}$}}}}}, !vi]
            (Ibr);
            \varronly{R1} \varronly{R2}
            \iarronly{I} \iarronly{R1} \iarronly{R2}
            \node[] at (R1.center) {$R_1$};
            \node[] at (R2.center) {$R_2$};
        \end{circuitikz} 
    \end{center}
    &
    \begin{center}
        \begin{circuitikz}
            \draw
            (0,0)
            to[V, name=E, V^>=$E_{}$, i^>=$I_{}$, !vi]
            (0,4) --
            (1,4) coordinate (It)
            to[R, name=R, i^>=$I$, v^<=$U_{AB}$, !vi]
            (1,2) coordinate (im)
            to[R, name=R1, i>_=$I_1$,
                v^<={{{{\rotatebox{90}{$U_{BC}$}}}}}, !vi]
            (1,0) coordinate (ib) --
            (0,0);
            \draw[]
            (im)
            to[short]
            ($(im)+(1,0)$) coordinate (Im)
            to[R, name=R2, i>_=$I_2$,
                v^<={{{{\rotatebox{90}{$U_{BC}$}}}}}, !vi]
            ($(ib)+(1,0)$) coordinate (Ib) --
            (ib);
            \varronly{E} \varronly{R} \varronly{R1} \varronly{R2} 
            \iarronly{E} \iarronly{R} \iarronly{R1} \iarronly{R2}
            \node[] at (R.center) {$R$};
            \node[] at (R1.center) {$R_1$};
            \node[] at (R2.center) {$R_2$};
            \draw[]
            (It) --++
            (0.5,0) node [right] {A};
            \draw[]
            (Im) --++
            (0.5,0) node [right] (B) {B};
            \draw[]
            (Ib) --++
            (0.5,0) node [right] (C) {C};
            \draw[dashed, ForestGreen]
            ([shift={(-0.3,0.1)}]im) rectangle
            ([shift={(0,-0.1)}]C.west);
            \node[right=2em] (E) at (R2) {$\equiv$};
            \draw[shift={($(E)+(2em,1)$)}]
            (0,0)
            to[R, name=Req, color=ForestGreen, i>^=$I$,
                v^<={{{{\rotatebox{90}{$U_{BC}$}}}}}, !vi]
            (0,-2);
            \varronly{Req} \iarronly{Req}
            \node[rotate=90] at (Req.center) {\color{ForestGreen}$R_{\rm eq}$};
        \end{circuitikz} 
    \end{center}
\end{NCdefi}
\begin{NCexem}[tabularx={Y|Y|Y}]{Application}
    \begin{enumerate}
        \item Avec le diviseur de tension :
            \fbox{$\begin{aligned}
                U_{AB} &= E\times \frac{R_1}{R_1+R_2}\\
                U_{BC} &= E\times \frac{R_2}{R_1+R_2}
            \end{aligned}$}
        \item Avec une loi des mailles simple :
            \[\boxed{I = \frac{E}{R_1+R_2}}\]
    \end{enumerate}
    &
    \begin{enumerate}
        \item On a directement \fbox{$U_{AB} = E$}.
        \item Avec le diviseur de courant :
            \[I_1 = I\times \frac{R_{\rm eq}}{R_1} = \boxed{I\times
            \frac{R_2}{R_1+R_2}}\]
    \end{enumerate}
    &
    \begin{enumerate}
        \item Diviseur de tension :\smallbreak
            $\begin{aligned}
                U_{AB} & = E \frac{R}{R+R_{\rm eq}}\\
                       & = \boxed{E \frac{R}{R + \frac{R_1R_2}{R_1+R_2}}}\\
                U_{BC} & = E \frac{R_{\rm eq}}{R + R_{\rm eq}}\\
                       & = \boxed{E \frac{R_1R_2}{R(R_1+R_2)+R_1R_2}}
            \end{aligned}$
        \item Diviseur de courant :
            \[I_1 = I \frac{R_{\rm eq}}{R_1} = \boxed{I \frac{R_2}{R_1+R_2}}\]
    \end{enumerate}
    ~
\end{NCexem}

\setcounter{section}{7}
\section{Transistor base commune}
\begin{tcbraster}[raster columns=2, raster equal height=rows]
    \begin{NCdefi}{Schéma}
        \begin{center}
            \begin{circuitikz}
                \draw
                (0,0)
                to[V, name=E, V^>=$E_{}$, i^>=$I_{}$, !vi]
                (0,2) --
                (1,2) coordinate (Ita) --
                (2,2) coordinate (Itb)
                to[I, name=I, I<=$I_{c}$, !vi]
                (4,2) coordinate (Itc)
                    node {\color{ForestGreen}$\bullet$}
                    node [above] {\color{ForestGreen}1} --
                (5,2)
                to[R, name=Rch, i<^=$I_{ch}$, !vi]
                (5,0) --
                (4,0) coordinate (Ibc) --
                (2,0) coordinate (Ibb) --
                (1,0) coordinate (Iba) --
                (0,0);
                \draw[]
                (Ita)
                to[R, name=Rb, i>^=$I_B$, !vi]
                (Iba);
                \draw[]
                (Itb)
                to[R, name=R1, i>^=$I_1$, !vi]
                (Ibb);
                \draw[]
                (Itc)
                to[R, name=Rc, i<^=$I_i$, !vi]
                (Ibc);
                \iarronly{I} \iarronly{E}
                \iarronly{Rch} \iarronly{Rb} \iarronly{R1} \iarronly{Rc}
                \varronly{E}
                \node[] at (Rch.center) {$R_{ch}$};
                \node[] at (R1.center) {$R_{1}$};
                \node[] at (Rb.center) {$R_{B}$};
                \node[] at (Rc.center) {$R_{C}$};
            \end{circuitikz}
        \end{center}
        \tcblower
        Ici, on note les intensités comme bon nous semble ; mais pour être
        cohérent-e avec les lois des nœuds et les intensités générées, on posera
        $I_B$ et $I_1$ dans le même sens vers le bas, alors que $I_i$ et
        $I_{ch}$ se rejoignent en \textcolor{ForestGreen}{1} pour donner $I_c$.
    \end{NCdefi}
    \begin{NCexem}{Application}
        Avec une loi des nœuds en \textcolor{ForestGreen}{1}, $I_i + I_{ch} =
        I$, d'où avec la relation du diviseur de courant :
        \begin{align*}
            R_{\rm eq} & = \frac{R_CR_{ch}}{R_C+R_{ch}}\\
            I_{ch} & = I \frac{R_{\rm eq}}{R_C+R_{ch}}\\
            \Leftrightarrow \Aboxed{I_{ch} & = I \frac{R_C}{R_C+R_{ch}}}
        \end{align*}
        Et évidemment, $\DS \boxed{U(R_{ch}) = I_C \times
        \frac{R_{ch}R_C}{R_C+R_{ch}}}$
    \end{NCexem}
\end{tcbraster}

\newpage
\section{Pont de Wheatstone}

\begin{tcbraster}[raster columns=6, raster equal height=rows]
    \begin{NCdefi}[raster multicolumn=2]{Schéma}
        \begin{center}
            \begin{circuitikz}
                \draw
                (0,0)
                to[V, name=E, V^>=$E_{}$, i^>=$I_{}$, !vi]
                (0,4) --
                (1,4) coordinate (Ilt)
                    node {\color{ForestGreen}$\bullet$}
                    node [above] {\color{ForestGreen}A}
                to[vR, name=R, i>^=$I_1$, v^<=$U_{AB}$, !vi]
                (1,2) coordinate (Ilm)
                    node {\color{ForestGreen}$\bullet$}
                    node [left] {\color{ForestGreen}B}
                    to[R, name=R1, v^<=$U_{BC}$, !vi]
                (1,0) coordinate (Ilb)
                    node {\color{ForestGreen}$\bullet$}
                    node [below] {C} --
                (0,0);
                \draw[]
                (Ilt) --
                (3,4)
                    node {\color{ForestGreen}$\bullet$}
                    node [above] {\color{ForestGreen}A}
                    to[R, name=R2, i>^=$I_2$, v^<=$U_{AD}$, !vi]
                (3,2) coordinate (Irm)
                    node {\color{ForestGreen}$\bullet$}
                    node [right] {\color{ForestGreen}D}
                    to[R, name=Ri, v^<=$U_{DC}$, !vi]
                (3,0) coordinate (Irb)
                    node {\color{ForestGreen}$\bullet$}
                    node [below] {\color{ForestGreen}C};
                \draw[]
                (Ilm)
                to[rmeter, name=V]
                (Irm);
                \draw[] 
                (Irb) --
                (Ilb);
                \iarronly{E} \iarronly{R} \iarronly{R2}
                \varronly{E} \varronly{R} \varronly{R2} \varronly{Ri}
                \varronly{R2} \varronly{R1}
                \node[] at (V.center) {V};
                \node[] at (R.center) {$R$};
                \node[] at (R1.center) {$R_1$};
                \node[] at (R2.center) {$R_2$};
                \node[] at (Ri.center) {$R_i$};
            \end{circuitikz}
        \end{center}
    \end{NCdefi}
    \begin{tcolorbox}[blankest, raster multicolumn=1, space to=\myspace]
        \begin{tcbraster}[raster columns=1]
            \begin{NCprop}[add to natural height=\myspace]{Résultat attendu}
                On cherche $R_i$, ou $U_{DC}$ quand «~le pont est équilibré~».
            \end{NCprop}
            \begin{NCdemo}{Outil}
                D'après l'énoncé, le pont est équilibré quand $V = 0$, soit
                quand $V_B = V_D$.
            \end{NCdemo}
        \end{tcbraster}
    \end{tcolorbox}
    \begin{NCexem}[raster multicolumn=3]{Application}
        Si le pont est équilibré, alors $U_{AB} = U_{AD}$ et $U_{BC} = U_{DC}$.
        Or, avec le pont diviseur de tension, on a à la fois
        \begin{align*}
            U_{BC} & = E \frac{R_1}{R_1+R}\\
            U_{DC} & = E \frac{R_i}{R_i+R_2}
        \end{align*}
        Donc
        \begin{align*}
            U_{BC} &= U_{DC}\\
            \Leftrightarrow \cancel{E} \frac{R_1}{R_1+R}
                   & = \cancel{E} \frac{R_i}{R_i+R_2}\\
            \Leftrightarrow R_1(\cancel{R_i}+R_2) & = R_i(\cancel{R_1}+R)\\
            \Leftrightarrow \Aboxed{R_i & = \frac{R_1R_2}{R}}
        \end{align*}
    \end{NCexem}
\end{tcbraster}

\end{document}
