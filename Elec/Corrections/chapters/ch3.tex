\documentclass[../main/main.tex]{subfiles}
\begin{document}

\chapter{Lois de Kirchhoff}\label{ch:O3}
\vspace*{-47pt}

\begin{center}
    \Huge Exercices d'application
\end{center}

\section{Tensions et courants}

\begin{tcbraster}[raster columns=6, raster equal height=rows]
    \begin{NCdefi}[raster multicolumn=4]{Schéma}
        \begin{center}
            \begin{circuitikz}
                \draw
                (0,0)
                    to[R, name=R1, i>_=$I(R_1)$, v^<=$U(R_1)$, !vi]
                (3,0) coordinate (Itl)
                    node {\color{ForestGreen}$\bullet$}
                    node [above] {\color{ForestGreen}1}
                    node [above left=0.4] (M) {}
                    to[R, name=R, i^>=$I(R)$, v^<=$U(R)$, !vi]
                (6,0) coordinate (Itr)
                    node {\color{ForestGreen}$\bullet$}
                    node [above] {\color{ForestGreen}2}
                    to[R, name=R3, i_>=$I(R_3)$, v^<=$U(R_3)$, !vi]
                (9,0)
                    to[R, name=R2, i>^=$I(R_2)$, v^<=$U(R_2)$, !vi]
                (9,-2)
                    node [below right=0.4] (N) {}
                    to[short]
                (6,-2) coordinate (Ibr)
                    node {\color{ForestGreen}$\bullet$}
                    node [below] {\color{ForestGreen}3}
                    to[R, name=Rp, i>_=$I(R')$, v^<=$U(R')$, !vi]
                (3,-2) coordinate (Ibl)
                    node {\color{ForestGreen!40}$\bullet$}
                    to[short]
                (0,-2);
                \draw[]
                (Itl)
                to[R, name=r, i>^=$I(r)$, v^<=$U(r)$, !vi]
                (Ibl);
                \draw[]
                (Itr)
                to[R, name=r2, i>^=$I(r_2)$, v^<=$U(r_2)$, !vi]
                (Ibr);
                \varronly{R} \varronly{R1} \varronly{R2} \varronly{R3}
                \varronly{Rp} \varronly{r} \varronly{r2}
                \iarronly{R} \iarronly{R1} \iarronly{R2} \iarronly{R3}
                \iarronly{Rp} \iarronly{r} \iarronly{r2}
                \node[] at (R.center) {$R$};
                \node[] at (Rp.center) {$R'$};
                \node[] at (R1.center) {$R_1$};
                \node[] at (R2.center) {$R_2$};
                \node[] at (R3.center) {$R_3$};
                \node[] at (r.center) {$r$};
                \node[] at (r2.center) {$r_2$};
                \node[Orchid] (LM1) at (4.5,-1) {1};
                \circledarrow{Orchid}{(LM1)}{0.3}
                \node[Orchid] (LM2) at (7.7,-1) {2};
                \circledarrow{Orchid}{(LM2)}{0.3}
                \draw[dashed, Orchid]
                (M) rectangle
                (N);
                \node[Orchid] (LM3) at (2,-1) {3};
                \circledarrow{Orchid}{(LM3)}{0.3}
            \end{circuitikz}
        \end{center}
    \end{NCdefi}
    \begin{NCrapp}[raster multicolumn=2]{Rappel}
        \begin{itemize}
            \item \textit{Nœud} : jonction entre au moins 3 câbles ;
            \item \textit{Branche} : section entre 2 nœuds ;
            \item \textit{Maille} : ensemble de branches fermées.
        \end{itemize}
    \end{NCrapp}
\end{tcbraster}

\begin{tcbraster}[raster columns=2, raster equal height=rows]
    \begin{NCexem}{Liens entre courants}
        On établit les liens entre les différents courants avec la loi des nœuds
        ou avec l'unicité de l'intensité dans une branche. Ici, on a :
        \begin{itemize}
            \item $I(R_2) = I(R_3)$ par unicité à droite ;
            \item $I(R_1) = I(R) + I(r)$ par LdN 1 ;
            \item $I(R) = I(R_3) + I(r_2)$ par LdN 2 ;
            \item $I(R_2) + I(r_2) = I(R')$ par LdN 3.
        \end{itemize}
        Le dernier nœud, non numéroté, donne une relation redondante avec les
        autres.
    \end{NCexem}
    \begin{NCcexe}{Liens entre tensions}
        Avec les lois des mailles, on a :
        \begin{itemize}
            \item $U(R') + U(r_2) + U(R) = U(r)$ par LdM 1 ;
            \item $U(R_2) + U(R_3) = U(r_2)$ par LdM 2 ;
        \end{itemize}
        La LdM 3 donne une relation redondante avec les deux premières : $U(R')
        + U(R_2) + U(R_3) + U(R) = U(r)$ est la somme des deux.
    \end{NCcexe}
\end{tcbraster}

\section{Calcul d'ntensité}
\begin{tcbraster}[raster columns=5, raster equal height=rows]
    \begin{NCdefi}[raster multicolumn=3]{Schéma}
        \begin{center}
            \begin{circuitikz}
                \draw
                (0,0)
                    to[R, name=r1, i>_=$I$, v^<=$U(r_1)$, !vi]
                (3,0)
                    coordinate (It)
                    node {\color{ForestGreen}$\bullet$}
                    node [above] {\color{ForestGreen}1}
                    to[R, name=R, i_>=$I_2$, v^<=$U(R)$, !vi]
                (6,0)
                    to[R, name=R2, v^<=$U(R_2)$, !vi]
                (6,-3)
                    to[short]
                (3,-3)
                    coordinate (Ib)
                    node {\color{ForestGreen!70}$\bullet$}
                    to[short]
                (0,-3)
                    to[V, name=E1, V^>=$E_{1}$, !vi]
                (0,0);
                \draw[]
                (It)
                to[R, name=R1, i>_=$I_1$, v^<=$U(R_1)$, !vi]
                (Ib);
                \varronly{E1} \varronly{R} \varronly{R1} \varronly{R2}
                \varronly{r1}
                \iarronly{R} \iarronly{R1} \iarronly{r1}
                \node[] at (r1.center) {$r_1$};
                \node[] at (R.center) {$R$};
                \node[] at (R1.center) {$R_1$};
                \node[] at (R2.center) {$R_2$};
                \node[Orchid] (LM1) at (1.5,-1.5) {1};
                \circledarrow{Orchid}{(LM1)}{0.3}
                \node[Orchid] (LM2) at (4.5,-1.5) {2};
                \circledarrow{Orchid}{(LM2)}{0.3}
            \end{circuitikz}
        \end{center}
    \end{NCdefi}
    \begin{tcolorbox}[blankest, raster multicolumn=2, space to=\myspace]
        \begin{tcbraster}[raster columns=1]
            \begin{NCprop}[add to natural height=\myspace]{Résultat attendu}
                On cherche à exprimer $I_2$.
            \end{NCprop}
            \begin{NCdemo}{{LdN, LdM}}
                \begin{itemize}
                    \item $I = I_1 + I_2 \color{ForestGreen}(1)$ (LdN) ;
                    \item $I_1R_1 + Ir_1 = E_1 \color{ForestGreen}(2)$ (LdM 1) ;
                    \item $I_2(R+R_2) = I_1R_1 \color{ForestGreen}(3)$ (LdM 2).
                \end{itemize}
            \end{NCdemo}
        \end{tcbraster}
    \end{tcolorbox}
\end{tcbraster}
\begin{tcbraster}[raster columns=2, raster equal height=rows]
    \begin{NCimpo}{Conseil}
        À partir de cet exercice, la résolution des devient assez
        mathématique et devient de la résolution de systèmes d'équations. Il y a
        de bonnes pratiques à cet égard : numéroter les équations qu'on veut
        réutiliser en premier lieu, à l'aide des $\color{ForestGreen}(1)$ par
        exemple, savoir qu'un système de 3 équations (indépendantes) à 3
        inconnues \underline{est} résolvable ensuite, et comprendre comment s'y
        prendre enfin. Cette dernière partie est bien sûr la vraie étape
        difficile et passe par la pratique, mais elle s'apprend. Je donne
        quelques outils dans l'encadré suivant.
    \end{NCimpo}
    \begin{NCrema}{Exemple}
        En ayant simplement utilisé les lois de Kirchhoff dont on dispose, $I_2$
        apparaît dans l'équation $ \color{ForestGreen}(3)$, mais s'exprime en
        fonction de $I_1$ que l'on ne connaît pas non plus. On doit donc
        commencer par trouver une expression de $I_1$ qui puisse nous faire
        avancer. $I_1$ fait partie de l'équation $\color{ForestGreen}(2)$ qui,
        elle, dépend de I mais en utilisant $\color{ForestGreen}(1)$ on peut
        facilement changer $ \color{ForestGreen}(2)$ en une nouvelle équation
        reliant $I_1$ à $I_2$ \underline{et qui n'est pas $\color{ForestGreen}
        (3)$} et qu'on appellera brillamment $ \color{ForestGreen}(4)$. Ainsi, 
        en réinjectant \textcolor{ForestGreen}{(4)} dans
        \textcolor{ForestGreen}{(3)}, on aura une expression de $I_2$ en
        fonction uniquement des paramètres du circuit ($E, R$).
    \end{NCrema}
\end{tcbraster}

\begin{center}
    \begin{NCexem}[width=.7\linewidth]{Application}
        Injecter \textcolor{ForestGreen}{(1)} dans \textcolor{ForestGreen}{(2)}
        donne :
        \begin{align*}
            I_1R_1 + (I_1-I_2)r_1 &= E_1\\
            I_1(R_1+r_1) &= E_1-I_2r_1\\
            I_1 &= \frac{E_1 - I_2r_1}{R_1 + r_1} \quad \color{ForestGreen}(4)
        \end{align*}
        Ainsi, il suffit de réinjecter \textcolor{ForestGreen}{(4)} dans
        \textcolor{ForestGreen}{(3)} pour avoir :
        \begin{align*}
            I_2(R_2+R) &= \frac{E_1 - I_2r_1}{R_1 + r_1}\times R_1\\
            I_2(R_2+R)\times(R_1+r) &= (E_1-I_2r)\times R_1\\
            I_2 \left[ (R_2+R)(R_1+r_1)+r_1R_1 \right] &= E_1R_1
        \end{align*}
        et finalement
        \[\boxed{I_2 = \frac{E_1R_1}{\left[ (R_2+R)(R_1+r_1)+r_1R_1 \right]}}\]
    \end{NCexem}
\end{center}

\section{Diviseur de tension}
\begin{tcbraster}[raster columns=3, raster equal height=rows]
    \begin{NCdefi}{Schéma}
        \begin{center}
            \begin{circuitikz}
                \draw
                (0,0)
                to[V, name=E, V^>=$E_{}$, i^>=$I_{}$, !vi]
                (0,4) --
                (1,4) coordinate (It)
                to[R, name=R2, i^>=$I$, v^<=$U_{AB}$, !vi]
                (1,2) coordinate (Im)
                to[R, name=R1, v^<=$U_{BC}$, !vi]
                (1,0) coordinate (Ib) --
                (0,0);
                \varronly{E} \varronly{R1} \varronly{R2}
                \iarronly{E} \iarronly{R2}
                \node[] at (R1.center) {$R_1$};
                \node[] at (R2.center) {$R_2$};
                \draw[]
                (It) --++
                (0.5,0) node [right] {A};
                \draw[]
                (Im) --++
                (0.5,0) node [right] {B};
                \draw[]
                (Ib) --++
                (0.5,0) node [right] {C};
            \end{circuitikz} 
        \end{center}
    \end{NCdefi}
    \begin{tcolorbox}[blankest, raster multicolumn=1, space to=\myspace]
        \begin{tcbraster}[raster columns=1]
            \begin{NCprop}{Résultat attendu}
                On cherche $I$ puis $U_{BC}$.
            \end{NCprop}
            \begin{NCdemo}[add to natural height=\myspace]{Outils}
                \begin{itemize}
                    \item Loi des mailles pour $I$ ;
                    \item Loi d'Ohm pour $U_{BC}$.
                \end{itemize}
            \end{NCdemo}
        \end{tcbraster}
    \end{tcolorbox}
    \begin{NCexem}{Application}
        Il suffit d'une loi des mailles pour trouver
        \[I = \frac{E}{R_1+R_2}\]
        Puis trivialement
        \[U_{BC} = R_1I = \frac{R_1}{R_1+R_2}E\]
    \end{NCexem}
\end{tcbraster}
\begin{tcbraster}[raster columns=2, raster equal height=rows]
    \begin{NCrema}{Remarque}
        On remarque donc que deux dipôles de résistances $R_1$ et $R_2$ se
        partageant une tension totale $E$ vont se la répartir
        en respectant la fraction de résistance à laquelle chaque diôle
        participe. C'est également une simple moyenne pondérée.
    \end{NCrema}
    \begin{NCror}{Important}
        Ce résultat est bien plus général que pour deux dipôles et fonctionne
        avec $n$ dipôles \textbf{\textit{en série}} sur une branche. Il faut
        pouvoir se ramener à ce schéma précis pour appliquer la formule du pont
        diviseur de tension – que vous pouvez maintenant utiliser sans loi des
        mailles : $\DS U_x = E\times \frac{R_x}{R_\mathrm{tot}}$
    \end{NCror}
\end{tcbraster}

\begin{tcbraster}[raster columns=3, raster equal height=rows]
    \begin{NCdefi}{Schéma}
        \begin{center}
            \begin{circuitikz}
                \draw
                (0,0)
                to[V, name=E, V^>=$E_{}$, i^>=$I_{}$, !vi]
                (0,4) --
                (1,4) coordinate (It)
                to[R, name=R2, i^>=$I$, v^<=$U_{AB}$, !vi]
                (1,2) coordinate (im)
                to[R, name=R1, v^<=$U_{BC}$, !vi]
                (1,0) coordinate (ib) --
                (0,0);
                \draw[]
                (im)
                to[short]
                ($(im)+(1,0)$) coordinate (Im)
                to[R, name=R3, i>^=$I_3$, v^<=$U_{BC}$, !vi]
                ($(ib)+(1,0)$) coordinate (Ib);
                \varronly{E} \varronly{R1} \varronly{R2} \varronly{R3}
                \iarronly{E} \iarronly{R2} \iarronly{R3}
                \node[] at (R1.center) {$R_1$};
                \node[] at (R2.center) {$R_2$};
                \node[] at (R3.center) {$R_3$};
                \draw[]
                (It) --++
                (0.5,0) node [right] {A};
                \draw[]
                (Im) --++
                (0.5,0) node [right] {B};
                \draw[]
                (Ib) --++
                (0.5,0) node [right] {C};
            \end{circuitikz} 
        \end{center}
    \end{NCdefi}
    \begin{tcolorbox}[blankest, raster multicolumn=1, space to=\myspace]
        \begin{tcbraster}[raster columns=1]
            \begin{NCprop}{Résultat attendu}
                On cherche $I$ puis $U_{BC}$.
            \end{NCprop}
            \begin{NCdemo}[add to natural height=\myspace]{Outils}
                \begin{itemize}
                    \item Loi des mailles pour $I$ ;
                    \item Loi d'Ohm pour $U_{BC}$.
                \end{itemize}
            \end{NCdemo}
        \end{tcbraster}
    \end{tcolorbox}
    \begin{NCexem}{Application}
        Il suffit d'une loi des mailles pour trouver
        \[I = \frac{E}{R_1+R_2}\]
        Puis trivialement
        \[U_{BC} = R_1I = \frac{R_1}{R_1+R_2}E\]
    \end{NCexem}
\end{tcbraster}
\begin{tcbraster}[raster columns=2, raster equal height=rows]
    \begin{NCrema}{Remarque}
        On remarque donc que deux dipôles de résistances $R_1$ et $R_2$ se
        partageant une tension totale $E$ vont se la répartir
        en respectant la fraction de résistance à laquelle chaque diôle
        participe. C'est également une simple moyenne pondérée.
    \end{NCrema}
    \begin{NCror}{Important}
        Ce résultat est bien plus général que pour deux dipôles et fonctionne
        avec $n$ dipôles \textbf{\textit{en série}} sur une branche. Il faut
        pouvoir se ramener à ce schéma précis pour appliquer la formule du pont
        diviseur de tension – que vous pouvez maintenant utiliser sans loi des
        mailles : $\DS U_x = E\times \frac{R_x}{R_\mathrm{tot}}$
    \end{NCror}
\end{tcbraster}

\end{document}
