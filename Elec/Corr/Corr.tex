\documentclass[10pt,a5paper,notitlepage]{book}

\usepackage{preambule}
\usepackage[Rejne]{fncychap}
\fancyhead[L]{\scriptsize Nora \textsc{Nicolas}}
\fancyhead[R]{\scriptsize \textsc{UE PG -- Fiche de travaux dirigés}}

\titleformat{\section}{\color{blue}\bfseries\Large}{
\hspace{-.85em}}{.5em}{}{}
\titleformat{\subsection}{\color{brandeisblue}\bfseries\large}{Exercice
\arabic{subsection})}{.5em}{}{}
\titleformat{\subsubsection}{\color{capri}\bfseries}{
\arabic{subsection}) \arabic{subsubsection}-}{.5em}{}{}

\newlength\tindent
\setlength{\tindent}{\parindent}
\setlength{\parindent}{0pt}
\renewcommand{\indent}{\hspace*{\tindent}}

\begin{document}

\begin{center}
\Huge Physique générale : Électricité\smallbreak\vspace*{-14pt}
\rule[11pt]{5cm}{0.5pt}\smallbreak\vspace*{-14pt}
\huge Chapitres 1 à 4
\end{center}

\toccontents

\setcounter{chapter}{-1}
\chapter{Conseils généraux}
\vspace*{-47pt}
Ce document à pour but de rappeler et résumer les conseils, arguments et astuces
qui ont pu être vues et énoncées durant les TDs. Il ne remplace ni les séances
en elles-mêmes, où votre participation active est nécessaire (c'est en se
trompant qu'on sait comment ne pas faire, et donc comment bien faire), ni les CM
de votre professeur-e. J'espère néanmoins qu'il saura vous être utile.
\smallbreak

La première partie comporte quelques éléments généraux sur l'optique. D'autres
conseils et éléments importants sont mis en valeur quand ils sont pertinents :
le code couleur reste le même, dans le but d'avoir une structure facilement
navigable. Les bases de réflexion, données ou définitions, sont en vert. Les
résultats importants, propriétés ou résultats à trouver, sont en rouge. Les
points pivots de réflexion, démonstration ou outils à choisir judicieusement,
sont en bleu. Les côtés pratiques, exemples et applications, sont en gris.
\smallbreak

Les premiers exercice du chapitre 1 sont intégralement corrigés, et certains
mots importants (comme « divergent ») ont une note de fin du chapitre 1 avec une
brève définition. Ces exercices représentent la base de comment construire sa
réflexion face à un exercice de pĥysique (d'optique particulièrement), mais ils
ne sont pas tous corrigés ainsi. Ainsi, vous verrez qu'après quelques exemples,
je vous renvoie aux corrigés que vous avez à disposition sur \textit{Claroline}.
Les schémas y sont clairs et j'espère que ma retranscription écrite du
raisonnement derrière ces schémas suffiront à vous guider. \smallbreak

Bonne lecture, \hfill Nora NICOLAS --
\href{mailto:n.nicolas@ipnl.in2p3.fr}{n.nicolas@ipnl.in2p3.fr}\\

\begin{rtcb}{Principe des exercices de physique}
    Tout exercice de physique suit le schéma suivant :
    \begin{enumerate}
        \item Lecture de l'énoncé en français et relevé des données ;
        \item Traduction des données en schéma si pertinent, et en expression
            mathématique si pertinent ;
        \item Compréhension de la réponse attendue ;
        \item Traduction de la réponse attendue en schéma si pertinent, et en
            expression mathématique si pertinent ;
        \item Détermination d'un ou de plusieurs outils (relation mathématique,
            règle de construction...) du cours faisant le lien entre les données
            et la réponse : répéter si besoin d'une réponse intermédiaire ;
        \item Application.
    \end{enumerate}
    Un exemple est donné partie \ref{ssec:vp}.
\end{rtcb}


\begin{ptcb}{Conseils}
    Avant d'encadrer un résultat :
    \begin{enumerate}
        \item Vérifer la cohérence mathématique avec la ligne précédente : les
            signes devant les grandeurs, le nombre de grandeurs, ne pas oublier
            les fonctions inverses... ;
        \item Vérifier l'homogénéité de part et d'autre de l'équation pour les
            résultats littéraux ;
        \item Vérifier la cohérence physique de la valeur numérique, notamment à
            l'aide d'un schéma
    \end{enumerate}
\end{ptcb}

\begin{bvtcb}{Important}
    L'erreur la plus simple mais la plus grave à faire est de se tromper sur une
    grandeur algébrique.
    \begin{center}
        \Ul{Toujours vérifier le sens des grandeurs algébriques} 
    \end{center}
\end{bvtcb}

\chapter{Grandeurs électriques}\label{ch:O1}
\vspace*{-24pt}
\section{Exercices d'application}
\subsection{Ordres de grandeur}

\theendnotes

\end{document}
