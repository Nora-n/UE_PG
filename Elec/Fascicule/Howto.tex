\documentclass[10pt,a5paper,notitlepage]{book}

\usepackage{preambule}
\usepackage[Rejne]{fncychap}
\fancyhead[L]{\scriptsize Nora \textsc{Nicolas}}
\fancyhead[R]{\scriptsize \textsc{UE PG -- Fiche de travaux dirigés}}

\titleformat{\section}{\color{blue}\bfseries\large}{\hspace{-0.3em}
Exercice~\arabic{section})}{.2em}{}{}
\titleformat{\subsection}{\color{brandeisblue}\bfseries}{\hspace{-0.3em}
\arabic{section}) \arabic{subsection}-}{.5em}{}{}
\titleformat{\subsubsection}{\color{capri}\bfseries}{\hspace{-0.3em}
\arabic{section}) \arabic{subsection}- \alph{subsubsection}.}{.5em}{}{}

\newlength\tindent
\setlength{\tindent}{\parindent}
\setlength{\parindent}{0pt}
\renewcommand{\indent}{\hspace*{\tindent}}

\begin{document}

\begin{center}
\Huge Physique générale : Électricité\smallbreak\vspace*{-14pt}
\rule[11pt]{5cm}{0.5pt}\smallbreak\vspace*{-14pt}
\huge Chapitres 1 à 4
\end{center}

\toccontents

\chapter{Style de page et tcolorboxes}
\vspace*{-47pt}

En plus de charger le package \verb=\usepackage{preambule}=, on doit charger
\verb=\usepackage[Rejne]{fncychap}= dans le préambule du fascicule pour qu'il
soit correctement utilisé. Il définit le style des chapitres. Les
\verb=\fancyhead[L]= et \verb=\fancyhead[R]= qui sont chargés écrasent ceux de
\verb=preambule.sty= qui est plus général et détaille les chapitres, ce qui
n'est pas nécessaire ici. \bigbreak

Ensuite vient le choix de style des sections, sous-sections et
sous-sous-sections. Elles sont définies par \verb=\titleformat= et son
relativement explicites dans leur fonctionnement. Sinon, d'autres exemples sont
présents dans \verb=preambule.sty= (ils se font aussi écraser ici). Étant donné
que le fascicule d'exercices de TD sont divisés entre exercices d'application et
d'entraînement mais ne rompent pas le numéro des exercices, on les indiquera par

\begin{verbatim}
    \begin{center}
        \Huge Exercices d'application
    \end{center}
\end{verbatim}

pour donner

\begin{center}
    \Huge Exercices d'application
\end{center}

On notera qu'il a été décidé que l'indentation ne se fasse pas par défaut en
mettant la grandeur \verb=\parindent= à 0 en utilisant
\verb=\setlength{\parindent}{0pt}=, mais qu'on peut forcer l'indentation
manuellement en utilisant la commande \verb=\indent=. \bigbreak

Les sections correspondent aux exercices. On écrira donc

\begin{verbatim}
    \section{Ordres de grandeur}
\end{verbatim}

pour avoir

\section{Ordres de grandeur}

et évidemment, les questions sont \verb=\subsection{}= et les sous-questions
sont \verb=\subsubsection{}=. Il ne paraît pas pertinent de réécrire les
énoncés dans les \verb={}=.

La mise en valeur des étapes au sein des réponses sont faites avec des
\textit{tcolorboxes}. Il en existe plein, codées dans preambule.sty. Pour chaque
\textit{tcolorbox}, il y a la version avec le \textit{counter} qui indique le
nombre de fois que cette \textit{tcolorbox} a été appelée, et la version sans
qui est appelée de la même manière en rajoutant « NC » devant, pour « No counter
».  Par exemple,

\begin{verbatim}
    \begin{prop}[label = propa]{Propriété}
        Ceci est le texte d'une boîte « Propriété »
        avec un \textit{counter}
    \end{prop}
\end{verbatim}

donne

\begin{prop}[label = propa]{Propriété}
    Ceci est le texte d'une boîte « Propriété » avec un \textit{counter}
\end{prop}

et

\begin{verbatim}
    \begin{NCprop}{Propriété}
        Ceci est le texte d'une boîte « Propriété »
        sans \textit{counter}.
    \end{NCprop}
\end{verbatim}

donne

\begin{NCprop}{Propriété}
    Ceci est le texte d'une boîte « Propriété » sans \textit{counter}.
\end{NCprop}

Loi
Theo
prop
coro = corolaire,
demo
inte = interprétation
Impl = implication

\begin{NCcoro}{Conseils}
    Avant d'encadrer un résultat :
    \begin{enumerate}
        \item Vérifer la cohérence mathématique avec la ligne précédente : les
            signes devant les grandeurs, le nombre de grandeurs, ne pas oublier
            les fonctions inverses... ;
        \item Vérifier l'homogénéité de part et d'autre de l'équation pour les
            résultats littéraux ;
        \item Vérifier la cohérence physique de la valeur numérique, notamment à
            l'aide d'un schéma
    \end{enumerate}
\end{NCcoro}

\begin{NCimpo}{Important}
    L'erreur la plus simple mais la plus grave à faire est de se tromper sur une
    grandeur algébrique.
    \begin{center}
        \Ul{Toujours vérifier le sens des grandeurs algébriques} 
    \end{center}
\end{NCimpo}

\chapter{Grandeurs électriques}\label{ch:O1}
\vspace*{-24pt}
\section{Exercices d'application}
\subsection{Ordres de grandeur}

%\theendnotes

\begin{circuitikz}
    \draw (0,0) node {$\bullet$} to[R=\shifttext{1.15cm}{$R_1$}, i>^=$i_{11}$]
          (0,4) node {$\bullet$} --
          (2,4) node {$\bullet$} to[R=\shifttext{-1.15cm}{$R_2$},
                                    i>^=$i_2$,
                                    v^<=$U_{R_2}$]
          (2,0) node {$\bullet$} to[R=\raisebox{-.5cm}{$R_{40}$}]
          (4,0) node {$\bullet$} to[V_=$E_3$, i^>=$i_3$] 
          (4,2) to[R=\shifttext{1.15cm}{$R_3$},
                   v_<=$U_{R_3}$]
          (4,4) node {$\bullet$} --
          (2,4)
          (0,0) -- (2,0) ;
    \node[above left ] (A) at (0,4) {A};
    \node[above] (B) at (2,4) {B};
    \node[above right] (C) at (4,4) {C};
    \node[below left] (E) at (0,0) {E};
    \node[below] (F) at (2,0) {F};
    \node[below right] (G) at (4,0) {G};
    \draw[->] (E) -- (F) node[midway, below] {$U \St{FE}$};
    \draw[->] (B) -- (A) node[midway, above] {$U \St{AB}$};
    
\end{circuitikz}

\begin{circuitikz}
    \draw (0,0) to[R=$R_1$]
    (0,4) to[V=$E_3$, i^>=$i_1$]
    (4,4) to[R=\shifttext{-1.15cm}{$R_2$}, v^<=$U_{R_2}$]
    (4,0)
    ;
\end{circuitikz}

\begin{circuitikz}
    \draw (0,0) to[I=$I$]
    (0,2) -- (3,2) node[right] {A}
    (1,2) to[R=\shifttext{-1.15cm}{$R_1$}]
    (1,0)
    (2,2) to[R=\shifttext{-1.15cm}{$R_2$}]
    (2,0)
    (0,0) -- (3,0) node[right] {B}
    ;
\end{circuitikz}

\end{document}
