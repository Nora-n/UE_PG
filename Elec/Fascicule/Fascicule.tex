\documentclass[10pt,a5paper,notitlepage]{book}

\usepackage{preambule}
\usepackage[Rejne]{fncychap}
\fancyhead[L]{\scriptsize Nora \textsc{Nicolas}}
\fancyhead[R]{\scriptsize \textsc{UE PG -- Fiche de travaux dirigés}}

\titleformat{\section}{\color{blue}\bfseries\Large}{
\hspace{-.85em}}{.5em}{}{}
\titleformat{\subsection}{\color{brandeisblue}\bfseries\large}{Exercice
\arabic{subsection})}{.5em}{}{}
\titleformat{\subsubsection}{\color{capri}\bfseries}{
\arabic{subsection}) \arabic{subsubsection}-}{.5em}{}{}

\newlength\tindent
\setlength{\tindent}{\parindent}
\setlength{\parindent}{0pt}
\renewcommand{\indent}{\hspace*{\tindent}}

\begin{document}

\begin{center}
\Huge Physique générale : Électricité\smallbreak\vspace*{-14pt}
\rule[11pt]{5cm}{0.5pt}\smallbreak\vspace*{-14pt}
\huge Chapitres 1 à 4
\end{center}

\toccontents

\setcounter{chapter}{-1}
\chapter{Conseils généraux}
\vspace*{-47pt}
Ce document à pour but de rappeler et résumer les conseils, arguments et astuces
qui ont pu être vues et énoncées durant les TDs. Il ne remplace ni les séances
en elles-mêmes, où votre participation active est nécessaire (c'est en se
trompant qu'on sait comment ne pas faire, et donc comment bien faire), ni les CM
de votre professeur-e. J'espère néanmoins qu'il saura vous être utile.
\smallbreak

La première partie comporte quelques éléments généraux sur l'électricité.
D'autres conseils et éléments importants sont mis en valeur quand ils sont
pertinents : le code couleur reste le même, dans le but d'avoir une structure
facilement navigable. Les bases de réflexion, données ou définitions, sont en
vert. Les résultats importants, propriétés ou résultats à trouver, sont en
rouge. Les points pivots de réflexion, démonstration ou outils à choisir
judicieusement, sont en bleu. Les côtés pratiques, exemples et applications,
sont en gris. \smallbreak

Les premiers exercice du chapitre 1 sont intégralement corrigés, et certains
mots importants (comme « divergent ») ont une note de fin du chapitre 1 avec une
brève définition. Ces exercices représentent la base de comment construire sa
réflexion face à un exercice de pĥysique (d'optique particulièrement), mais ils
ne sont pas tous corrigés ainsi. Ainsi, vous verrez qu'après quelques exemples,
je vous renvoie aux corrigés que vous avez à disposition sur \textit{Claroline}.
Les schémas y sont clairs et j'espère que ma retranscription écrite du
raisonnement derrière ces schémas suffiront à vous guider. \smallbreak

Bonne lecture, \hfill Nora NICOLAS --
\href{mailto:n.nicolas@ipnl.in2p3.fr}{n.nicolas@ipnl.in2p3.fr}\\

\begin{NCprop}{Principe des exercices de physique}
    Tout exercice de physique suit le schéma suivant :
    \begin{enumerate}
        \item Lecture de l'énoncé en français et relevé des données ;
        \item Traduction des données en schéma si pertinent, et en expression
            mathématique si pertinent ;
        \item Compréhension de la réponse attendue ;
        \item Traduction de la réponse attendue en schéma si pertinent, et en
            expression mathématique si pertinent ;
        \item Détermination d'un ou de plusieurs outils (relation mathématique,
            règle de construction...) du cours faisant le lien entre les données
            et la réponse : répéter si besoin d'une réponse intermédiaire ;
        \item Application.
    \end{enumerate}
    Un exemple est donné partie .
\end{NCprop}


\begin{NCcoro}{Conseils}
    Avant d'encadrer un résultat :
    \begin{enumerate}
        \item Vérifer la cohérence mathématique avec la ligne précédente : les
            signes devant les grandeurs, le nombre de grandeurs, ne pas oublier
            les fonctions inverses... ;
        \item Vérifier l'homogénéité de part et d'autre de l'équation pour les
            résultats littéraux ;
        \item Vérifier la cohérence physique de la valeur numérique, notamment à
            l'aide d'un schéma
    \end{enumerate}
\end{NCcoro}

\begin{NCimpo}{Important}
    L'erreur la plus simple mais la plus grave à faire est de se tromper sur une
    grandeur algébrique.
    \begin{center}
        \Ul{Toujours vérifier le sens des grandeurs algébriques} 
    \end{center}
\end{NCimpo}

\chapter{Grandeurs électriques}\label{ch:O1}
\vspace*{-24pt}
\section{Exercices d'application}
\subsection{Ordres de grandeur}

Cet exercice ce concentre sur la notion d'intensité en électricité. Faisons tout
d'abord un petit rappel du cours.

\begin{NCrapp}{Outil du cours : intensité électrique}
    L'intensité électrique est une grandeur physique décrivant la quantité de
    charges électriques (exprimées en Coulomb, C) passant par un point d'un
    circuit à chaque unité de temps (exprimé en seconde, s) :
	\begin{equation}
		I = \dfrac{Q}{t} \label{eq:1.1.intensite}
	\end{equation}
    L'intensité est ainsi exprimée en Coulomb par seconde, unité que l'on nomme
    l'Ampère (A). Si les charges sont des électrons se déplaçant dans un fil, le
    nombre de charges est :
	\begin{equation}
		Q = N\times e \label{eq:1.1.charge}
	\end{equation}
    où $e = 1.602\times 10^{-19}\,$C est la charge de l'électron (en valeur
    absolue).
\end{NCrapp}

Nous voyons donc que le temps, l'intensité et le nombre de charges sont reliées
par les formules \ref{eq:1.1.intensite} et \ref{eq:1.1.charge}.

\begin{NCprop}{Résultats attendus}
    Les trois questions de l'exercice donnent une grandeur électrique et
    attendent de vous le calcul d'une grandeur inconnue. Il va donc falloir
    utiliser les formules précédentes pour exprimer la grandeur inconnue en
    fonction des données du problème.
\end{NCprop}

\begin{NCdefi}{Données}
	\begin{enumerate}
        \item "Un générateur délivre une intensité $I = 3,0\,$A." : $I = 3\,$A ;
		\item "1000 électrons" : $N = 1000$ ;
        \item "faire circuler $1.10^{20}$ électrons chaque seconde" : $N =
            1\times 10^{20}$, $t = 1\,$s.
	\end{enumerate}
\end{NCdefi}

\begin{NCexem}{Application}
	\begin{enumerate}
		\item Le nombre d'électrons émis chaque seconde est donné par :
		\begin{equation}
			N = \dfrac{I \times t}{e}
		\end{equation}
		Avec les données du problème, nous avons :
		\begin{equation}
			N = \dfrac{3.0 \times 1}{1.6\times 10^{-19}} = 1.9\times 10^{19}
		\end{equation}
		\item Le temps pour émettre 1000 électrons est donné par :
		\begin{equation}
			t = \dfrac{N\times e}{I} = \dfrac{1000 \times 1.6\times 10^{-19}}{3.0}\,{\rm s} = 5.3\times 10^{-17}\,{\rm s}
		\end{equation}
		\item L'intensité correspondante est :
		\begin{equation}
			I = \dfrac{N\times e}{t} = \dfrac{1.0\times 10^{20} \times 1.6\times 10^{-19}}{1}\,{\rm A} = 16\,{\rm A}
		\end{equation}
	\end{enumerate}
\end{NCexem}

\begin{impo}{Important}
	Dans cet exercice, nous avons dû faire des applications numériques. Il faut alors faire attention à deux choses :
	\begin{itemize}
		\item l'unité : dès que vous remplacez les grandeurs littérales par des valeurs numériques, votre calcul acquiert une unité, qui doit apparaître ;
		\item les chiffres significatifs : le résultat final doit comporter un nombre de chiffres significatifs cohérent avec la précision des données utilisées. Par exemple, l'intensité $I = 3.0\,$A a deux chiffres significatifs, ce qui va limiter la précision avec laquelle on va utiliser la charge de l'électron à deux chiffres : $e = 1.6\times 10^{-19}\,$C. Autre cas, quand on vous dit "par seconde", le temps $t$ a alors la valeur $t = 1\,$s, avec une précision arbitraire, qui sera limitée par la précision des autres données. Il en va de même pour le nombre $N = 1000$ électrons.
	\end{itemize}
\end{impo}

\subsection{Potentiels, tensions et courants}

\begin{defi}{Potentiel et tension électrique}
	Dans cet exercice, nous allons appliquer les notions de potentiel et de tension électrique, ainsi que celle de sens "conventionnel" du courant.
	\begin{enumerate}
		\item Le potentiel électrique peut être vu comme un équivalent de l'altitude en mécanique : si vous êtes en altitude, vous avez le "potentiel" de tomber et de fournir de l'énergie, emmagasinée pendant la chute, en arrivant au sol. Chaque point d'un circuit est ainsi à une certaine "altitude". Si on considère une particule de charge positive, alors cette particule a le comportement intuitif et "tombe" des potentiels les plus élevés vers les plus bas (le $+$ repousse le $+$). Si cette particule est chargée négativement, comme l'électron, elle "remonte" des bas potentiels vers les plus élevés (le $+$ attire le $-$).
		\item La tension électrique est la différence de potentiel entre deux points d'un circuit. Sa notation est intuitive : $U_{\rm AB}$ est la différence de potentiel entre A et B, $V_{\rm A} - V_{\rm B}$. On la représente par contre comme une flèche allant de $B$ vers $A$ : la flèche suit les potentiels croissants.
		\item Le sens conventionnel du courant positif est alors l'inverse du sens de circulation des électrons, car ils sont chargés négativement.
	\end{enumerate}
\end{defi}

\begin{enumerate}
	\item Sur le circuit ci-dessous, on a indiqué le sens de circulation des électrons, donc des potentiels les plus bas vers les plus hauts.
	\begin{figure}[h!]
		\centering{
		\begin{circuitikz}
			\draw (0,2) node {$\bullet$} to[R = \raisebox{-.5cm}{$R_1$}, i<^=$e^-$]
			(3,2) node {$\bullet$} to[R = \raisebox{-0.5cm}{$R$}, i<_=$e^-$]
			(6,2) node {$\bullet$} to[R = \raisebox{-0.5cm}{$R_3$}, i<_=$e^-$]
			(9,2) node {$\bullet$} to[R = \shifttext{-1.15cm}{$R_2$}, i<^=$e^-$]
			(9,0) node {$\bullet$} to[short, i<^=$e^-$]
			(6,0) node {$\bullet$} to[short, i<^=$e^-$]
			(3,0) node {$\bullet$} to[short, i<^=$e^-$]
			(0,0) node {$\bullet$}
			;
			\draw (3,2) to[R = \shifttext{-0.9cm}{$r$}, i^<=$e^-$] (3,0);
			\draw (6,2) to[R = \shifttext{-1.cm}{$r_2$}, i^<=$e^-$] (6,0);
			\node[above left] (A) at (0,2) {A, $5.0\,$V};
			\node[above] (B) at (3,2) {B, $3.0\,$V};
			\node[above] (C) at (6,2) {C, $2.5\,$V};
			\node[above right] (D) at (9,2) {D, $x\,$V};
			\node[below right] (E) at (9,0) {E};
			\node[below] (F) at (6,0) {F};
			\node[below] (G) at (3,0) {G};
			\node[below left] (H) at (0,0) {H, $1.0\,$V};
		\end{circuitikz}
		}
	\end{figure}
	\item Sur le circuit suivant, on a indiqué le sens conventionnel du courant positif, donc des hauts potentiels vers les bas.
		\begin{figure}[h!]
		\centering{
			\begin{circuitikz}
				\draw (0,2) node {$\bullet$} to[R = \raisebox{-.5cm}{$R_1$}, i>^=$I>0$]
				(3,2) node {$\bullet$} to[R = \raisebox{-0.5cm}{$R$}, i>_=$I>0$]
				(6,2) node {$\bullet$} to[R = \raisebox{-0.5cm}{$R_3$}, i>_=$I>0$]
				(9,2) node {$\bullet$} to[R = \shifttext{-1.15cm}{$R_2$}, i>^=$I>0$]
				(9,0) node {$\bullet$} to[short, i>^=$I>0$]
				(6,0) node {$\bullet$} to[short, i>^=$I>0$]
				(3,0) node {$\bullet$} to[short, i>^=$I>0$]
				(0,0) node {$\bullet$}
				;
				\draw (3,2) to[R = \shifttext{-0.9cm}{$r$}, i^>=$I>0$] (3,0);
				\draw (6,2) to[R = \shifttext{-1.cm}{$r_2$}, i^>=$I>0$] (6,0);
				\node[above left] (A) at (0,2) {A, $5.0\,$V};
				\node[above] (B) at (3,2) {B, $3.0\,$V};
				\node[above] (C) at (6,2) {C, $2.5\,$V};
				\node[above right] (D) at (9,2) {D, $x\,$V};
				\node[below right] (E) at (9,0) {E};
				\node[below] (F) at (6,0) {F};
				\node[below] (G) at (3,0) {G};
				\node[below left] (H) at (0,0) {H, $1.0\,$V};
			\end{circuitikz}
		}
	\end{figure}
	\item L'analogie de l'altitude pour les potentiels électriques nous donne une intuition pour la valeur du potentiel au point D : ce potentiel doit être compris "entre" ceux des points C et E, car aucune source extérieure ne peut permettre de relever ou d'abaisser artificiellement le point D. Par exemple, $x = 2.0\,$V.
	\item Sur le circuit ci-dessous, nous avons fléché les tensions en question.
	\begin{figure}[h!]
		\centering{
			\begin{circuitikz}
				\draw (0,2) node {$\bullet$} to[R = \raisebox{-.5cm}{$R_1$}, v_<=$U_{\rm AB}$]
				(3,2) node {$\bullet$} to[R = \raisebox{-0.5cm}{$R$}, v_>=$U_{\rm CB}$]
				(6,2) node {$\bullet$} to[R = \raisebox{-0.5cm}{$R_3$}]
				(9,2) node {$\bullet$} to[R = \shifttext{-1.15cm}{$R_2$}]
				(9,0) node {$\bullet$} to[short]
				(6,0) node {$\bullet$} to[short, v^<=$U_{\rm FG}$]
				(3,0) node {$\bullet$} to[short]
				(0,0) node {$\bullet$}
				;
				\draw (3,2) to[R = \shifttext{-0.9cm}{$r$}, v^<=$U_{\rm BG}$] (3,0);
				\draw (6,2) to[R = \shifttext{-1.cm}{$r_2$}, v^<=$U_{\rm CF}$] (6,0);
				\node[above left] (A) at (0,2) {A, $5.0\,$V};
				\node[above] (B) at (3,2) {B, $3.0\,$V};
				\node[above] (C) at (6,2) {C, $2.5\,$V};
				\node[above right] (D) at (9,2) {D, $x\,$V};
				\node[below right] (E) at (9,0) {E};
				\node[below] (F) at (6,0) {F};
				\node[below] (G) at (3,0) {G};
				\node[below left] (H) at (0,0) {H, $1.0\,$V};
				\draw[->,red] (C) -- (E) node[midway, above] {$U \St{EC}$};
				\draw[->,red] (H) -- (C) node[midway, above] {$U \St{CH}$};
				\draw[->,red] (E) -- (A) node[midway, below] {$U_{\rm AE}$};
			\end{circuitikz}
		}
	\end{figure}
	Ces tensions se calculent par différence de potentiel : $U_{\rm AB} = V_{\rm A} - V_{\rm B} = 2.0\,$V, $U_{\rm FG} = 0\,$V, $U_{\rm BG} = 2.0\,$V, $U_{\rm CB} = -0.5\,$V, $U_{\rm CF} = 1.5\,$V, $U_{\rm EC} = -1.5\,V$, $U_{\rm CH} = 1.5\,$V et $U_{\rm AE} = 4.0\,$V.
	\item
	\begin{NCdemo}{Outils du cours}
		Une tension peut être difficile à calculer au premier coup d'{\oe}il. Dans ces cas-là, il peut être utile d'utiliser des points intermédiaires dont on connaît le potentiel, c'est une forme de composition des vecteurs. Par exemple, la tension $U_{\rm AB}$ entre B et A peut être décomposée à l'aide d'un point tiers, C, par la formule $U_{\rm AB} = U_{\rm AC} + U_{\rm CB}$. On peut le démontrer facilement en remplaçant les tensions par des différences de potentiel.
	\end{NCdemo}
	\begin{NCexem}{Application}
		\begin{itemize}
			\item $U_{\rm AG} = U_{\rm AB} + U_{\rm BG}$
			\item $U_{\rm AD} = U_{\rm AB} + U_{\rm BD}$
		\end{itemize}
	\end{NCexem}
	\item Les points E, F, G et H sont reliés par des fils, sans dipôles intermédiaires. Ils sont donc au même potentiel.
\end{enumerate}
\begin{impo}{Potentiel dans un circuit}
	\label{def:potentiel}
	Dans un circuit, tous les points reliés par des fils sans dipôles intermédiaires sont au même potentiel. Ils sont en fait considérés comme un seul point dans un circuit. Il peut être utile de se servir de cette propriété pour redessiner un circuit sous une forme plus simple.
\end{impo}

\subsection{Schématisation}

\begin{defi}{Résistances équivalentes et associations de résistances}
	Très souvent, un circuit électrique contient de nombreuses résistances. Ces résistances peuvent être :
	\begin{itemize}
		\item en série : elles sont sur une même branche, aucune branche tierce ne part du point qui les sépare
			\begin{circuitikz}
				\draw (0,0) node {$\bullet$} to[R = \raisebox{-0.5cm}{$R_1$}]
				(2,0) to[R = \raisebox{-0.5cm}{$R_2$}]
				(4,0) node {$\bullet$};
			\end{circuitikz}
		\item en dérivation : elles sont situées sur deux branches connectées à leurs extrémités
			\begin{circuitikz}
				\draw (0,0) node {$\bullet$} --
				(1,0) --
				(1,0.5) to[R = \raisebox{-0.5cm}{$R_1$}]
				(3,0.5) --
				(3,0) --
				(4,0) node {$\bullet$};
				\draw (1,0) --
				(1,-0.5) to[R = \raisebox{-0.5cm}{$R_2$}]
				(3,-0.5) --
				(3,0);
			\end{circuitikz}
	\end{itemize}
	On note symboliquement que deux résistances sont en série par $R_1 + R_2$, elles sont alors équivalentes à une seule résistance de valeur :
	\begin{equation}
		R_{\rm série} = R_1 + R_2
	\end{equation}
	On note symboliquement que deux résistances sont en dérivation par $R_1 // R_2$, elles sont alors équivalentes à une seule résistance de valeur :
	\begin{equation}
		R_{\rm dérivation} = \dfrac{R_1\times R_2}{R_1 + R_2}
	\end{equation}
\end{defi}

\begin{NCcoro}{Conseils}
	Lors de la transformation d'une formule symbolique d'une association de résistance en schéma, il est conseillé de commencer par les parenthèses les plus intérieures, puis d'ajouter les éléments en allant vers l'extérieur des parenthèses. Comme pour la multiplication $\times$ et l'addition $+$, le symbole $//$ est prioritaire devant le symbole $+$. N'oubliez pas les branches de sortie afin de définir explicitement quel est le dipôle final.
\end{NCcoro}

\begin{NCexem}{Application}
	\begin{itemize}
		\item $[(R+r)//(r+r_2)]+r_1$ :\\[-0.5cm]
		\begin{center}
			\begin{circuitikz}
				\draw (0.5,0) node {$\bullet$} --
				(1,0) --
				(1,0.5) to[R = \raisebox{-0.5cm}{$R$}]
				(3,0.5) to[R = \raisebox{-0.5cm}{$r$}]
				(5,0.5) --
				(5,0) to[R = \raisebox{-0.5cm}{$r_1$}]
				(7,0) node {$\bullet$};
				\draw (1,0) --
				(1,-0.5) to[R = \raisebox{-0.5cm}{$r$}]
				(3,-0.5) to[R = \raisebox{-0.5cm}{$r_2$}]
				(5,-0.5) --
				(5,0);
			\end{circuitikz}
		\end{center}
		\item $\{ [ (R//r)+r_1 ]//(r + r_2) \} + (r_1//R)$ :\\[-0.5cm]
		\begin{center}
			\begin{circuitikz}
				\draw (0,0) node {$\bullet$} --
				(0.5,0) --
				(0.5,1) --
				(1,1) --
				(1,1.5) to[R = \raisebox{-0.5cm}{$R$}]
				(3,1.5) --
				(3,1) to[R = \raisebox{-0.5cm}{$r_1$}]
				(5,1) --
				(5,0) --
				(6,0) --
				(6,0.5) to[R = \raisebox{-0.5cm}{$r_1$}]
				(8,0.5) --
				(8,0) --
				(8.5,0) node {$\bullet$};
				\draw (0.5,0) --
				(0.5,-0.5) to[R = \raisebox{-0.5cm}{$r$}]
				(2.75,-0.5) to[R = \raisebox{-0.5cm}{$r_2$}]
				(5,-0.5) --
				(5,0);
				\draw (1,1) --
				(1,0.5) to[R = \raisebox{-0.5cm}{$r$}]
				(3,0.5) --
				(3,1);
				\draw (6,0) --
				(6,-0.5) to[R = \raisebox{-0.5cm}{$R$}]
				(8,-0.5) --
				(8,0);
			\end{circuitikz}
		\end{center}
	\end{itemize}
\end{NCexem}

\subsection{Association de résistances}

\begin{NCcoro}{Conseils}
	Dans cet exercice, on va exprimer la résistance équivalente d'un circuit, symboliquement, à l'aide des signes $//$ et $+$. Le plus simple est de procéder par étapes afin d'identifier les couples de résistances en série et en dérivation, et de les remplacer par leur résistance équivalente. Pour ce faire, on se rappelle que d'après la définition \ref{def:potentiel}, on peut déplacer les points le long de fils tant qu'on ne traverse pas de dipôle.
\end{NCcoro}

\begin{enumerate}
	\item 	\begin{itemize}
				\item Dans le schéma 1, $R_1$ et $R$ ne sont ni en série ni en dérivation, il y a un n{\oe}ud entre les deux, qui part sur une autre branche. $r$ et $r_2$ ne sont pas en parallèle, il y a une résistance ($R$) sur la branche transverse. $R_2$ et $R$ : aucun. $R_3$ et $r_2$ : aucun. $R_3$ et $R_2$ sont en série.
				\item Dans le schéma 2, $R_1$ et $R$ : aucun. $r$ et $r_2$ : aucun. $R_3$ et $r_2$ sont en parallèle, ce qui est bien visible si on déplace le point B en bas à gauche du schéma.
				\item Dans le schéma 3, $R_1$ et $R$ sont en série. $R_2$ et $R$ : aucun. $r$ et $r^\prime$ sont en parallèle, si on prend les deux résistances en bas à gauche du schéma.
			\end{itemize}
	\item On va faire en détails le cas du schéma 1.
	\begin{center}
		\begin{circuitikz}
			\draw (0,0) node {$\bullet$} to[R = \raisebox{-0.5cm}{$R_1$}]
			(2,0) to[R = \raisebox{-0.5cm}{$R$}]
			(4,0) to[R = \raisebox{-0.5cm}{$R_3$}]
			(6,0) to[R = \shifttext{-1.15cm}{$R_2$}]
			(6,-2) --
			(0,-2) node {$\bullet$};
			\draw (2,0) to[R = \shifttext{-0.9cm}{$r$}]
			(2,-2);
			\draw (4,0) to[R = \shifttext{-1cm}{$r_2$}]
			(4,-2);
			\node[left] (A) at (0,0) {A};
			\node[left] (B) at (0,-2) {B};
		\end{circuitikz}
	\end{center}
	Sur ce circuit, $R_3$ et $R_2$ sont en série. Nous les remplaçons par une résistance équivalente $R_{\rm eq,1} = R_3 + R_2$. Le nouveau schéma est :
	\begin{center}
		\begin{circuitikz}
			\draw (0,0) node {$\bullet$} to[R = \raisebox{-0.5cm}{$R_1$}]
			(2,0) to[R = \raisebox{-0.5cm}{$R$}]
			(4,0) --
			(6,0) to[R = \shifttext{-1.15cm}{$R_{\rm eq,1}$}]
			(6,-2) --
			(0,-2) node {$\bullet$};
			\draw (2,0) to[R = \shifttext{-0.9cm}{$r$}]
			(2,-2);
			\draw (4,0) to[R = \shifttext{-1cm}{$r_2$}]
			(4,-2);
			\node[left] (A) at (0,0) {A};
			\node[left] (B) at (0,-2) {B};
		\end{circuitikz}
	\end{center}
	Maintenant, on voit que $r_2$ et $R_{\rm eq,1}$ sont en parallèle. On les remplace par la résistance équivalente $R_{\rm eq,2} = r_2 // R_{\rm eq,1}$. Le nouveau schéma est :
	\begin{center}
		\begin{circuitikz}
			\draw (0,0) node {$\bullet$} to[R = \raisebox{-0.5cm}{$R_1$}]
			(2,0) to[R = \raisebox{-0.5cm}{$R$}]
			(4,0) to[R = \shifttext{-1.15cm}{$R_{\rm eq,2}$}]
			(4,-2) --
			(0,-2) node {$\bullet$};
			\draw (2,0) to[R = \shifttext{-0.9cm}{$r$}]
			(2,-2);
			\node[left] (A) at (0,0) {A};
			\node[left] (B) at (0,-2) {B};
		\end{circuitikz}
	\end{center}
	$R$ et $R_{\rm eq,2}$ sont en série, on les remplace par la résistance équivalente $R_{\rm eq,3} = R + R_{\rm eq,2}$ :
	\begin{center}
		\begin{circuitikz}
			\draw (0,0) node {$\bullet$} to[R = \raisebox{-0.5cm}{$R_1$}]
			(2,0) --
			(4,0) to[R = \shifttext{-1.15cm}{$R_{\rm eq,3}$}]
			(4,-2) --
			(0,-2) node {$\bullet$};
			\draw (2,0) to[R = \shifttext{-0.9cm}{$r$}]
			(2,-2);
			\node[left] (A) at (0,0) {A};
			\node[left] (B) at (0,-2) {B};
		\end{circuitikz}
	\end{center}
	$r$ et $R_{\rm eq,3}$ sont en parallèle, on les remplace par $R_{\rm eq,4} = r // R_{\rm eq,3}$ :
	\begin{center}
		\begin{circuitikz}
			\draw (0,0) node {$\bullet$} to[R = \raisebox{-0.5cm}{$R_1$}]
			(2,0) to[R = \raisebox{-0.5cm}{$R_{\rm eq,4}$}]
			(4,0);
			\node[left] (A) at (0,0) {A};
			\node[right] (B) at (4,0) {B};
		\end{circuitikz}
	\end{center}
	$R_1$ et $R_{\rm eq,4}$ sont en série. La résistance totale entre A et B est :
	\begin{align}
		R_{\rm eq} &= R_1 + R_{\rm eq,4} \nonumber \\ &= R_1 + (r//R_{\rm eq,3}) \nonumber \\
		&= R_1 + (r//[R + R_{\rm eq,2}]) \nonumber \\
		&= R_1 + (r//[R + \{ r_2 // R_{\rm eq,1} \}]) \nonumber \\
		&= R_1 + (r//[R + \{ r_2 // (R_3 + R_2) \}]) \label{eq:1.4.1}
	\end{align}
	Une fois ce calcul terminé, on peut vérifier par le chemin inverse que la schématisation de la résistance équivalente \ref{eq:1.4.1} donne bien le schéma de départ.
\end{enumerate}

\newpage

%\theendnotes

\begin{circuitikz}
    \draw (0,0) node {$\bullet$} to[R=\shifttext{1.15cm}{$R_1$}, i>^=$i_{11}$]
          (0,4) node {$\bullet$} --
          (2,4) node {$\bullet$} to[R=\shifttext{-1.15cm}{$R_2$},
                                    i>^=$i_2$,
                                    v^<=$U_{R_2}$]
          (2,0) node {$\bullet$} to[R=\raisebox{-.5cm}{$R_{40}$}]
          (4,0) node {$\bullet$} to[V_=$E_3$, i^>=$i_3$] 
          (4,2) to[R=\shifttext{1.15cm}{$R_3$},
                   v_<=$U_{R_3}$]
          (4,4) node {$\bullet$} --
          (2,4)
          (0,0) -- (2,0) ;
    \node[above left ] (A) at (0,4) {A};
    \node[above] (B) at (2,4) {B};
    \node[above right] (C) at (4,4) {C};
    \node[below left] (E) at (0,0) {E};
    \node[below] (F) at (2,0) {F};
    \node[below right] (G) at (4,0) {G};
    \draw[->] (E) -- (F) node[midway, below] {$U \St{FE}$};
    \draw[->] (B) -- (A) node[midway, above] {$U \St{AB}$};
    
\end{circuitikz}

\begin{circuitikz}
    \draw (0,0) to[R=$R_1$]
    (0,4) to[V=$E_3$, i^>=$i_1$]
    (4,4) to[R=\shifttext{-1.15cm}{$R_2$}, v^<=$U_{R_2}$]
    (4,0)
    ;
\end{circuitikz}

\begin{circuitikz}
    \draw (0,0) to[I=$I$]
    (0,2) -- (3,2) node[right] {A}
    (1,2) to[R=\shifttext{-1.15cm}{$R_1$}]
    (1,0)
    (2,2) to[R=\shifttext{-1.15cm}{$R_2$}]
    (2,0)
    (0,0) -- (3,0) node[right] {B}
    ;
\end{circuitikz}

\end{document}
