\documentclass[10pt,a5paper,notitlepage]{book}

\usepackage{preambule}
\usepackage[Rejne]{fncychap}
\fancyhead[L]{\scriptsize Nora \textsc{Nicolas}}
\fancyhead[R]{\scriptsize \textsc{UE PG -- Fiche de travaux dirigés}}

\titleformat{\section}{\color{blue}\bfseries\Large}{
\hspace{-.85em}}{.5em}{}{}
\titleformat{\subsection}{\color{brandeisblue}\bfseries\large}{Exercice
\arabic{subsection})}{.5em}{}{}
\titleformat{\subsubsection}{\color{capri}\bfseries}{
\arabic{subsection}) \arabic{subsubsection}-}{.5em}{}{}

\newlength\tindent
\setlength{\tindent}{\parindent}
\setlength{\parindent}{0pt}
\renewcommand{\indent}{\hspace*{\tindent}}

\begin{document}

\begin{center}
\Huge Physique générale : Électricité\smallbreak\vspace*{-14pt}
\rule[11pt]{5cm}{0.5pt}\smallbreak\vspace*{-14pt}
\huge Chapitres 1 à 4
\end{center}

\toccontents

\setcounter{chapter}{-1}
\chapter{Conseils généraux}
\vspace*{-47pt}
Ce document à pour but de rappeler et résumer les conseils, arguments et astuces
qui ont pu être vues et énoncées durant les TDs. Il ne remplace ni les séances
en elles-mêmes, où votre participation active est nécessaire (c'est en se
trompant qu'on sait comment ne pas faire, et donc comment bien faire), ni les CM
de votre professeur-e. J'espère néanmoins qu'il saura vous être utile.
\smallbreak

La première partie comporte quelques éléments généraux sur l'électricité.
D'autres conseils et éléments importants sont mis en valeur quand ils sont
pertinents : le code couleur reste le même, dans le but d'avoir une structure
facilement navigable. Les bases de réflexion, données ou définitions, sont en
vert. Les résultats importants, propriétés ou résultats à trouver, sont en
rouge. Les points pivots de réflexion, démonstration ou outils à choisir
judicieusement, sont en bleu. Les côtés pratiques, exemples et applications,
sont en gris. \smallbreak

Les premiers exercice du chapitre 1 sont intégralement corrigés, et certains
mots importants (comme « divergent ») ont une note de fin du chapitre 1 avec une
brève définition. Ces exercices représentent la base de comment construire sa
réflexion face à un exercice de pĥysique (d'optique particulièrement), mais ils
ne sont pas tous corrigés ainsi. Ainsi, vous verrez qu'après quelques exemples,
je vous renvoie aux corrigés que vous avez à disposition sur \textit{Claroline}.
Les schémas y sont clairs et j'espère que ma retranscription écrite du
raisonnement derrière ces schémas suffiront à vous guider. \smallbreak

Bonne lecture, \hfill Nora NICOLAS --
\href{mailto:n.nicolas@ipnl.in2p3.fr}{n.nicolas@ipnl.in2p3.fr}\\

\begin{NCprop}{Principe des exercices de physique}
    Tout exercice de physique suit le schéma suivant :
    \begin{enumerate}
        \item Lecture de l'énoncé en français et relevé des données ;
        \item Traduction des données en schéma si pertinent, et en expression
            mathématique si pertinent ;
        \item Compréhension de la réponse attendue ;
        \item Traduction de la réponse attendue en schéma si pertinent, et en
            expression mathématique si pertinent ;
        \item Détermination d'un ou de plusieurs outils (relation mathématique,
            règle de construction...) du cours faisant le lien entre les données
            et la réponse : répéter si besoin d'une réponse intermédiaire ;
        \item Application.
    \end{enumerate}
    Un exemple est donné partie .
\end{NCprop}


\begin{NCcoro}{Conseils}
    Avant d'encadrer un résultat :
    \begin{enumerate}
        \item Vérifer la cohérence mathématique avec la ligne précédente : les
            signes devant les grandeurs, le nombre de grandeurs, ne pas oublier
            les fonctions inverses... ;
        \item Vérifier l'homogénéité de part et d'autre de l'équation pour les
            résultats littéraux ;
        \item Vérifier la cohérence physique de la valeur numérique, notamment à
            l'aide d'un schéma
    \end{enumerate}
\end{NCcoro}

\begin{NCimpo}{Important}
    L'erreur la plus simple mais la plus grave à faire est de se tromper sur une
    grandeur algébrique.
    \begin{center}
        \Ul{Toujours vérifier le sens des grandeurs algébriques} 
    \end{center}
\end{NCimpo}

\chapter{Grandeurs électriques}\label{ch:O1}
\vspace*{-24pt}
\section{Exercices d'application}
\subsection{Ordres de grandeur}

Cet exercice ce concentre sur la notion d'intensité en électricité. Faisons tout d'abord un petit rappel du cours.

\begin{defi}{Intensité électrique}
	L'intensité électrique est une grandeur physique décrivant la quantité de charges électriques (exprimées en Coulomb, C) passant par un point d'un circuit à chaque unité de temps (exprimé en seconde, s) :
	\begin{equation}
		I = \dfrac{Q}{t} \label{eq:1.1.intensite}
	\end{equation}
	L'intensité est ainsi exprimée en Coulomb par seconde, unité que l'on nomme l'Ampère (A). Si les charges sont des électrons se déplaçant dans un fil, le nombre de charges est :
	\begin{equation}
		Q = N\times e \label{eq:1.1.charge}
	\end{equation}
	où $e = 1.602\times 10^{-19}\,$C est la charge de l'électron (en valeur absolue).
\end{defi}

Nous voyons donc que le temps, l'intensité et le nombre de charges sont reliées par les formules \ref{eq:1.1.intensite} et \ref{eq:1.1.charge}.

\begin{NCprop}{Résultats attendus}
	Les trois questions de l'exercice donnent une grandeur électrique et attendent de vous le calcul d'une grandeur inconnue. Il va donc falloir utiliser les formules précédentes pour exprimer la grandeur inconnue en fonction des données du problème.
\end{NCprop}

\begin{NCdefi}{Données}
	\begin{enumerate}
		\item "Un générateur délivre une intensité $I = 3,0\,$A." : $I = 3\,$A ;
		\item "1000 électrons" : $N = 1000$ ;
		\item "faire circuler $1.10^{20}$ électrons chaque seconde" : $N = 1\times 10^{20}$, $t = 1\,$s.
	\end{enumerate}
\end{NCdefi}

\begin{NCexem}{Application}
	\begin{enumerate}
		\item Le nombre d'électrons émis chaque seconde est donné par :
		\begin{equation}
			N = \dfrac{I \times t}{e}
		\end{equation}
		Avec les données du problème, nous avons :
		\begin{equation}
			N = \dfrac{3.0 \times 1}{1.6\times 10^{-19}} = 1.9\times 10^{19}
		\end{equation}
		\item Le temps pour émettre 1000 électrons est donné par :
		\begin{equation}
			t = \dfrac{N\times e}{I} = \dfrac{1000 \times 1.6\times 10^{-19}}{3.0}\,{\rm s} = 5.3\times 10^{-17}\,{\rm s}
		\end{equation}
		\item L'intensité correspondante est :
		\begin{equation}
			I = \dfrac{N\times e}{t} = \dfrac{1.0\times 10^{20} \times 1.6\times 10^{-19}}{1}\,{\rm A} = 16\,{\rm A}
		\end{equation}
	\end{enumerate}
\end{NCexem}

\begin{impo}{Important}
	Dans cet exercice, nous avons dû faire des applications numériques. Il faut alors faire attention à deux choses :
	\begin{itemize}
		\item l'unité : dès que vous remplacez les grandeurs littérales par des valeurs numériques, votre calcul acquiert une unité, qui doit apparaître ;
		\item les chiffres significatifs : le résultat final doit comporter un nombre de chiffres significatifs cohérent avec la précision des données utilisées. Par exemple, l'intensité $I = 3.0\,$A a deux chiffres significatifs, ce qui va limiter la précision avec laquelle on va utiliser la charge de l'électron à deux chiffres : $e = 1.6\times 10^{-19}\,$C. Autre cas, quand on vous dit "par seconde", le temps $t$ a alors la valeur $t = 1\,$s, avec une précision arbitraire, qui sera limitée par la précision des autres données. Il en va de même pour le nombre $N = 1000$ électrons.
	\end{itemize}
\end{impo}

\newpage

%\theendnotes

\begin{circuitikz}
    \draw (0,0) node {$\bullet$} to[R=\shifttext{1.15cm}{$R_1$}, i>^=$i_{11}$]
          (0,4) node {$\bullet$} --
          (2,4) node {$\bullet$} to[R=\shifttext{-1.15cm}{$R_2$},
                                    i>^=$i_2$,
                                    v^<=$U_{R_2}$]
          (2,0) node {$\bullet$} to[R=\raisebox{-.5cm}{$R_{40}$}]
          (4,0) node {$\bullet$} to[V_=$E_3$, i^>=$i_3$] 
          (4,2) to[R=\shifttext{1.15cm}{$R_3$},
                   v_<=$U_{R_3}$]
          (4,4) node {$\bullet$} --
          (2,4)
          (0,0) -- (2,0) ;
    \node[above left ] (A) at (0,4) {A};
    \node[above] (B) at (2,4) {B};
    \node[above right] (C) at (4,4) {C};
    \node[below left] (E) at (0,0) {E};
    \node[below] (F) at (2,0) {F};
    \node[below right] (G) at (4,0) {G};
    \draw[->] (E) -- (F) node[midway, below] {$U \St{FE}$};
    \draw[->] (B) -- (A) node[midway, above] {$U \St{AB}$};
    
\end{circuitikz}

\begin{circuitikz}
    \draw (0,0) to[R=$R_1$]
    (0,4) to[V=$E_3$, i^>=$i_1$]
    (4,4) to[R=\shifttext{-1.15cm}{$R_2$}, v^<=$U_{R_2}$]
    (4,0)
    ;
\end{circuitikz}

\begin{circuitikz}
    \draw (0,0) to[I=$I$]
    (0,2) -- (3,2) node[right] {A}
    (1,2) to[R=\shifttext{-1.15cm}{$R_1$}]
    (1,0)
    (2,2) to[R=\shifttext{-1.15cm}{$R_2$}]
    (2,0)
    (0,0) -- (3,0) node[right] {B}
    ;
\end{circuitikz}

\end{document}
